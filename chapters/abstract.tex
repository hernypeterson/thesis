\chapter{Abstract}
The martian volcano Olympus Mons records billions of years of eruptive and intrusive magmatism, including considerable evidence for activity in the last few hundred million years. Ongoing studies of the well-preserved edifice promise untold insights into Martian interior and surface evolution, as well as major volcanic processes like caldera-forming eruptions that shape life on Earth. With only limited satellite-derived datasets available, researchers continually strive to gain new insight from prior observations. The motivating observation for this thesis is the presence of numerous ``discordant'' lava flows at the summit of Olympus Mons; features whose original downhill direction no longer aligns with their topographic environment. I develop and implement a new method for estimating magma reservoir pressure changes responsible for the surface deformation leading to these discordant features. Initial results are promising but the method requires additional refinement before assessing the subsurface activity of Olympus Mons.