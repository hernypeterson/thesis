\chapter{Abstract}
The martian shield volcano Olympus Mons records billions of years of eruptive and intrusive magmatism, with considerable evidence for activity within the last few hundred million years. Ongoing inquiries into the beautifully preserved edifice promise untold insights into Martian evolution, its planetary interior, and important volcanic processes like large caldera-forming eruptions that shape life on Earth. However, the only datasets currently available to study Olympus Mons are derived from satellites, such as imagery and topography. Therefore, it is crucial to make the most of this very limited data. In this thesis, my data set is the topographic discordance between the direction of lava flows (presumably once downhill) and their topographic surroundings. I develop and apply a new method for directly estimating reservoir position (both horizontal and vertical within the edifice), size, and pressurization to account for observed discordance.