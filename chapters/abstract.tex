\chapter{Abstract}
Numerous lava flows at the summit of the Martian volcano Olympus Mons are topographically ``discordant''---they point in a direction other than downhill. If they originally flowed downhill, the underlying surface must have subsequently deformed to its current state. Starting with this assumption, I develop a new method for estimating net surface attitude change from discordant features. I then compare the spatial distribution of attitude change implied by these flows with the surface response to pressure change in a modeled magma reservoir at an array of possible locations. After identifying the most plausible reservoir locations, I determine the combinations of depth, radius, aspect ratio, and pressure change most consistent with the discordant features at the surface. The distribution of discordant flows is consistent with caldera-collapse-associated deflation centered at the eastern caldera rim---not inflation south of the caldera complex as proposed by \textcite{mouginis-mark_late-stage_2019}. \textcite{chadwick_late_2015} estimate \qtyrange{1.33e5}{1.35e6}{\km\cubed} of volcanic material incorporated into the edifice in recent Martian history; plausible reservoirs identified here account for 0.5\% to 12\% of this volume.