\chapter{Conclusions}

In this thesis I develop a method for estimating the position, size, and shape of pressurized magma reservoirs necessary to explain the surface deformation implied by topographically discordant lava features at the summit of Olympus Mons. Using this method, I show that discordant lava flows southeast of the caldera complex are best explained by a highly oblate reservoir, similar in size to the caldera complex itself but offset laterally to the east. However, the depth and pressure change of this modeled reservoir prove difficult to assess independently using this method. These results suggest that a small but measurable fraction of recent magma incorporated within the Olympus Mons edifice is situated within or immediately below the edifice near the caldera complex. The methods developed here are suitable for significant expansion and refinement to assess the physical processes occurring within Olympus Mons and other volcanic systems across the terrestrial planets.