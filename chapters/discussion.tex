\chapter{Discussion}\label{cha:discussion}

\section{Limitations}

Only a fraction of the study area surface is covered in flows whose orientation can be reliably determined. These flows are of different ages. They record an incomplete picture of the topographic surface upon which they were emplaced. Only those which were subsequently deformed off-axis (i.e., by an offset center, or a more complex inflation system with the same net tilt axis) become discordant. In this case, the majority of pressure change appears to have been minimally offset, if at all.

The empirical basis of this thesis relies on inferring paths through attitude space.

% Beyond this point, however, several concrete choices are necessary to make forward progress in the analysis; exhaustively analyzing the effect of tweaking each numerical value is unrealistic. For example, I choose a paleo-azimuth uncertainty term of \ang{7} (Figure~\ref{fig:az1-uncertainty}) not by any measurement or calculation but simply because a choice needs to be made: the analysis had failed to produce any interpretable results without an uncertainty term here due to the sensitivity of Equation~\eqref{eq:paleo-slope}. The maximum tilt envelope (Figure~\ref{fig:envelope}) is borne similarly out of necessity. Without it, the vast majority of computed tilts vary widely across the unrealistic range of roughly \ang{-90} to \ang{90}. I explain why this wide variation is to be expected (it does not represent a mathematical error) but it does require precisely defining a single plausibility criterion which could just as easily be defined slightly differently. Finally, varying the initial guess and maximum number of iterations in the non-linear regression analysis can affect the results, including whether the solution converges at all for particularly messy tilt data. These concrete decisions, together with the unpredictable (at the spatial resolution of the center candidate array) results presented in Chapter~\ref{cha:results}, suggests that the method I present is more spatially sensitive than I anticipated. In my view, the results are broadly plausible enough with my expectations to rule out the possibility of a major mathematical error, but not yet refined enough to proceed with fitting actual numerical model results based on specific reservoir geometries and inflation energies.

\section{Future Directions}

\subsection{Crater Counting}
Important progress has been made toward constraining the timeline of Late Amazonian volcanic activity using crater counting \parencite{kneissl_map-projection-independent_2011,robbins_volcanic_2011,
robbins_large_2013,
platz_crater-based_2013}, although a detailed chronology is still out of reach. The difficulty inherent to the stochastic nature of this method is that small regions are difficult to date with high precision. One result from this thesis which could address this challenge is that discordant populations which are best explained by the same axisymmetric pressure change conditions are likely to have been deformed (if not emplaced) at the same time. Conversely, populations whose most likely centers differ in position and inflation energy are more likely to be stratigraphically distinct. While the method of this thesis cannot directly resolve relative ages, it could help to identify contemporaneous units and thus increase the precision of the crater counting method by expanding the area of interest.

\subsection{Morphological Constraints on Paleo-slope}

The fundamentally missing piece of attitude data (which the whole axisymmetric tilt framework is designed to estimate) is paleo-slope: the slope of the surface upon which a lava feature was originally emplaced. The paleo-slope values that I calculate could be compared with independent constraints on paleo-slope based on the morphology of lava flows \parencite{wadge_lobes_1991, peitersen_correlations_2000, peters_lava_2021} to assess plausibility of the axisymmetric assumption I use to derive these estimates.

\subsection{Evaluating Model Conditions}

Finally, it will be important to bring the tilt analysis beyond analytical regression fit to more careful consideration of a range of reservoir parameters. After making these more specific estimates, numerical modeling can provide more sophisticated criteria for physical plausibility than the simple cutoff envelope used here. For example, it will be important to determine whether estimated parameters would cause reservoir wall failure before estimated pressure would physically be able to accumulate. 