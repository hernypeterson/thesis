\chapter{Discussion}\label{cha:discussion}

\section{Inflation Center Results}

\subsection{Tiltable Fraction}\label{sec:tiltable-fraction}

% Butterfly pattern. As shown in Figure~\ref{fig:test-tilt}, the magnitude of tilt necessary to explain a discordant feature is minimized when the direction toward the inflation center is perpendicular to the paleo-azimuth direction. Since the physical plausibility envelope is a maximum cutoff threshold, it makes sense that the center candidates in the orientation that minimizes tilt result in the highest tiltable fraction. However, the distribution tiltable fraction for each population does not depend only on azimuth direction between population features and center location; there is clearly a distance-dependent component as well. Notice for each feature that the greenest regions tend to be roughly \qtyrange{10}{20}{\km} from the population center. This corresponds to the distance where the tilt envelope (Figure~\ref{fig:envelope}) has its maximum value of \ang{\sim3}. However, a number of additional factors influence the distribution for each population. As the center candidate location approaches the population midpoint, its orientation relative to the samples at extremes of each population approaches parallel to the paleo-azimuth direction, making the center less likely to explain those features. None of these considerations depend on the particular distribution of discordant samples, they are inherent to the method itself.

% Of course, the actual spatial distribution of discordance within the sample has the final word in the distribution of tiltable fraction score. For example, notice in Figures~\ref{fig:results-a} and \ref{fig:results-b} that the two regions of maximum tiltable fraction are rotated clockwise relative to their expected position along the perpendicular bisector of their respective populations.

% conceptual figure?

\subsection{Analytical Convergence}\label{sec:convergence}

\subsubsection{\acs{RMSE}}\label{sec:rmse}

\subsection{Numerical Refinements}

\section{Limitations}

The process I develop in this thesis involves a long sequence of calculations, the intermediate products of which can be difficult to interpret and physically verify. Additionally, I introduce computational and organization expense by iterating the sequence of calculations over an array of axisymmetric center candidate points. I contend that this complexity is a fair price for the flexibility it affords in the analysis. After all, the summit of Olympus Mons is \emph{not} axisymmetric at the surface or (presumably) in its interior, and if such a simplification is to be applied I should be confident that the choice of vertical axis position is appropriate to the question at hand.

% Beyond this point, however, several concrete choices are necessary to make forward progress in the analysis; exhaustively analyzing the effect of tweaking each numerical value is unrealistic. For example, I choose a paleo-azimuth uncertainty term of \ang{7} (Figure~\ref{fig:az1-uncertainty}) not by any measurement or calculation but simply because a choice needs to be made: the analysis had failed to produce any interpretable results without an uncertainty term here due to the sensitivity of Equation~\eqref{eq:paleo-slope}. The maximum tilt envelope (Figure~\ref{fig:envelope}) is borne similarly out of necessity. Without it, the vast majority of computed tilts vary widely across the unrealistic range of roughly \ang{-90} to \ang{90}. I explain why this wide variation is to be expected (it does not represent a mathematical error) but it does require precisely defining a single plausibility criterion which could just as easily be defined slightly differently. Finally, varying the initial guess and maximum number of iterations in the non-linear regression analysis can affect the results, including whether the solution converges at all for particularly messy tilt data. These concrete decisions, together with the unpredictable (at the spatial resolution of the center candidate array) results presented in Chapter~\ref{cha:results}, suggests that the method I present is more spatially sensitive than I anticipated. In my view, the results are broadly plausible enough with my expectations to rule out the possibility of a major mathematical error, but not yet refined enough to proceed with fitting actual numerical model results based on specific reservoir geometries and inflation energies.

Additionally, it will be necessary to probe every component of the setup carefully, since even small errors introduced by faulty assumptions that in another analysis would be negligible here prove to be important. For example, I suspect that the \hlss{Calculate Geometry Attributes} tool in ArcGIS Pro may be introducing noticeable error in my \acl{az1} measurement. As I discuss in the setup for the \acs{bearing} calculation (Equation~\eqref{eq:bearing}), bearing azimuth is not constant along a linear feature. Nonetheless, I assign a single value of \acs{az1} to each point sampled from a feature, which could introduce small but significant systematic error, especially for longer lava features.

In my view, the logical next step for this inquiry to address these concerns is to focus on a smaller region with much higher spatial resolution in the center candidate array. It will be clear when sufficient spatial resolution is achieved because the inflation center evaluation should approach spatial continuity the way the paleo-summit estimate (Figure~\ref{fig:summit-score}) already has.

\section{Future Directions}

Once this method is refined to produce more convincing analytical and numerical fits, it will be possible to derive a number of results for comparison with additional methods.

\subsection{Crater Counting}
Important progress has been made toward constraining the timeline of Late Amazonian volcanic activity using crater counting \parencite{kneissl_map-projection-independent_2011,robbins_volcanic_2011,
robbins_large_2013,
platz_crater-based_2013}, although a detailed chronology is still out of reach. The difficulty inherent to the stochastic nature of this method is that small regions are difficult to date with high precision. One result from this thesis which could address this challenge is that discordant populations which are best explained by the same axisymmetric pressure change conditions are likely to have been deformed (if not emplaced) at the same time. Conversely, populations whose most likely centers differ in position and inflation energy are more likely to be stratigraphically distinct. While the method of this thesis cannot directly resolve relative ages, it could help to identify contemporaneous units and thus increase the precision of the crater counting method by expanding the area of interest.

\subsection{Morphological Constraints on Paleo-slope}

The fundamentally missing piece of attitude data (which the whole axisymmetric tilt framework is designed to estimate) is paleo-slope: the slope of the surface upon which a lava feature was originally emplaced. The paleo-slope values that I calculate could be compared with independent constraints on paleo-slope based on the morphology of lava flows \parencite{wadge_lobes_1991, peitersen_correlations_2000, peters_lava_2021} to assess plausibility of the axisymmetric assumption I use to derive these estimates.

\subsection{Evaluating Model Conditions}

Finally, it will be important to bring the tilt analysis beyond analytical regression fit to more careful consideration of a range of reservoir parameters. After making these more specific estimates, numerical modeling can provide more sophisticated criteria for physical plausibility than the simple cutoff envelope used here. For example, it will be important to determine whether estimated parameters would cause reservoir wall failure before estimated pressure would physically be able to accumulate. 