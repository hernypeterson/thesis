\chapter{Discussion}

Differentiated lava lows and channels do not uniformly cover the summit of Olympus Mons. Immediately beyond the caldera rim, distinguishable flows are limited mostly to the south-east (e.g., Figure~\ref{fig:uphill-flows}) and north-west margins. Notably, there are essentially zero flows immediately south of the caldera, closer to the modern summit. [FIGURE of all mapped lobate lava flows]. Lava channels are more widespread than lobate flows, especially further from the caldera rim 

\section{Fundamental Considerations}

In Section~\ref{sec:considerations}, I introduce the important geometric considerations guiding this thesis. The first of these is topographic discordance, the simple observation that lava features at the summit of Olympus Mons are oriented differently from the local downhill direction. The spatial distribution of these features constitutes the single dataset from which I attempt to derive as much information as possible about the underlying processes responsible. To make progress with this interpretation, I show that certain assumptions must be made; the first and most important of these assumptions is one of axisymmetric geometry.

Wherever possible, I attempt to 

\section{Future Directions}

\subsection{Crater Counting}
\textcite{kneissl_map-projection-independent_2011,robbins_volcanic_2011,
robbins_large_2013,
platz_crater-based_2013}

\subsection{Morphological Constraints on Paleo-slope}
\textcite{wadge_lobes_1991, peitersen_correlations_2000, peters_lava_2021}.

\subsection{Evaluating Model Conditions}

Reservoir wall failure mechanisms.

