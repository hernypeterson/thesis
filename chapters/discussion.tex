\chapter{Discussion}\label{cha:discussion}

\section{Limitations}

Only a fraction of the study area surface is covered in flows whose orientation can be reliably determined. These flows are of different ages. They record an incomplete picture of the topographic surface upon which they were emplaced. Only those which were subsequently deformed off-axis (i.e., by an offset center, or a more complex inflation system with the same net tilt axis) become discordant. In this case, the majority of pressure change appears to have been minimally offset, if at all.

\section{Future Directions}

\subsection{Crater Counting}
Important progress has been made toward constraining the timeline of Late Amazonian volcanic activity using crater counting \parencite{kneissl_map-projection-independent_2011,robbins_volcanic_2011,
robbins_large_2013,
platz_crater-based_2013}, although a detailed chronology is still out of reach. The difficulty inherent to the stochastic nature of this method is that small regions are difficult to date with high precision. One result from this thesis which could address this challenge is that discordant populations which are best explained by the same axisymmetric pressure change conditions are likely to have been deformed (if not emplaced) at the same time. Conversely, populations whose most likely centers differ in position and inflation energy are more likely to be stratigraphically distinct. While the method of this thesis cannot directly resolve relative ages, it could help to identify contemporaneous units and thus increase the precision of the crater counting method by expanding the area of interest.

\subsection{Morphological Constraints on Paleo-slope}

The fundamentally missing piece of attitude data (which the whole axisymmetric tilt framework is designed to estimate) is paleo-slope: the slope of the surface upon which a lava feature was originally emplaced. The paleo-slope values that I calculate could be compared with independent constraints on paleo-slope based on the morphology of lava flows \parencite{wadge_lobes_1991, peitersen_correlations_2000, peters_lava_2021} to assess plausibility of the axisymmetric assumption I use to derive these estimates.

\subsection{Evaluating Model Conditions}

Finally, it will be important to bring the tilt analysis beyond analytical regression fit to more careful consideration of a range of reservoir parameters. After making these more specific estimates, numerical modeling can provide more sophisticated criteria for physical plausibility than the simple cutoff envelope used here. For example, it will be important to determine whether estimated parameters would cause reservoir wall failure before the estimated pressure needed to explain the tilt signature would physically be able to accumulate. 