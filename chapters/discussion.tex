\chapter{Discussion}\label{cha:discussion}

In this thesis, I focus on developing and I implementing a method which ultimately involves a fairly long sequence of calculations including several iterative elements. Most of the associated computational and organization expense is introduced with the large array of axisymmetric center candidate points. I contend that this complexity is a fair price to pay for the flexibility it affords in the analysis. After all, the summit of Olympus Mons is \emph{not} axisymmetric either at the surface or (presumably) in its interior, and if such a simplification is to be applied I should at least be confident that the choice of vertical axis position is appropriate to the question at hand.

Beyond this point, I need to make certain choices somewhat arbitrarily to make forward progress in the analysis. For example, I choose a paleo-azimuth uncertainty term of \ang{7} (Figure~\ref{fig:az1-uncertainty}) not by measurement or calculation but simply because a choice needs to be made: the analysis was failing to produce any results without an uncertainty term here. The maximum tilt envelope (Figure~\ref{fig:envelope}) is borne similarly out of necessity. Without it, the vast majority of computed tilts vary widely across the approximate range from \ang{-90} to \ang{90}. These decisions, together with the largely unpredictable (on the spatial scale of the center candidate spacing) and poorly-fitting results presented in Chapter~\ref{cha:results}, suggests that the method I present is much more sensitive than I originally anticipated. In my view, the results are consistent enough with my expectations to rule out the possibility of a major mathematical error, but not yet refined enough to proceed with fitting actual numerical model results based on specific reservoir geometries and inflation energies.

\section{Future Directions}

My first priority for the following weeks,

\subsection{Crater Counting}
\textcite{kneissl_map-projection-independent_2011,robbins_volcanic_2011,
robbins_large_2013,
platz_crater-based_2013}

\subsection{Morphological Constraints on Paleo-slope}
\textcite{wadge_lobes_1991, peitersen_correlations_2000, peters_lava_2021}.

\subsection{Evaluating Model Conditions}

Reservoir wall failure mechanisms.

