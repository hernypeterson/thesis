\chapter{Discussion}\label{cha:discussion}

In this thesis, I focus on developing and I implementing a method which ultimately involves a fairly long sequence of calculations including several iterative elements. Most of the associated computational and organization expense is introduced with the large array of axisymmetric center candidate points. I contend that this complexity is a fair price to pay for the flexibility it affords in the analysis. After all, the summit of Olympus Mons is \emph{not} axisymmetric either at the surface or (presumably) in its interior, and if such a simplification is to be applied I should at least be confident that the choice of vertical axis position is appropriate to the question at hand.

Beyond this point, I need to make certain choices somewhat arbitrarily to make forward progress in the analysis. For example, I choose a paleo-azimuth uncertainty term of \ang{7} (Figure~\ref{fig:az1-uncertainty}) not by measurement or calculation but simply because a choice needs to be made: the analysis was failing to produce any results without an uncertainty term here. The maximum tilt envelope (Figure~\ref{fig:envelope}) is borne similarly out of necessity. Without it, the vast majority of computed tilts vary widely across the approximate range from \ang{-90} to \ang{90}. These decisions, together with the largely unpredictable (on the spatial scale of the center candidate spacing) and poorly-fitting results presented in Chapter~\ref{cha:results}, suggests that the method I present is much more sensitive than I originally anticipated. In my view, the results are consistent enough with my expectations to rule out the possibility of a major mathematical error, but not yet refined enough to proceed with fitting actual numerical model results based on specific reservoir geometries and inflation energies.

In my view, the logical next step for this inquiry is to focus on a smaller region with much higher spatial resolution to probe more local pressure change effects. It will be clear when sufficient spatial resolution is achieved because the inflation center evaluation will approach spatial continuity the way the paleo-summit estimate (Figure~\ref{fig:summit-score}) already has.

\section{Future Directions}

Once this method is refined to produce more convincing analytical and numerical fits, it will be possible to derive a number of results for comparison with additional methods.

\subsection{Crater Counting}
Important progress has been made toward constraining the timeline of Late Amazonian volcanic activity using crater counting \parencite{kneissl_map-projection-independent_2011,robbins_volcanic_2011,
robbins_large_2013,
platz_crater-based_2013}, although a detailed chronology is still out of reach. One of the difficulties in this method is that small regions are difficult to date precisely due to the stochastic nature of the method. One result from this thesis which could address this challenge is that discordant populations which are best explained by the same axisymmetric pressure change conditions are likely to have been deformed at the same time. Conversely, populations whose most likely centers differ in position and inflation energy are likely to be stratigraphically distinct. While the method of this thesis cannot resolve relative ages, it could help to identify contemporaneous units and thus increase the precision of the crater counting method.

\subsection{Morphological Constraints on Paleo-slope}

The fundamentally missing piece of attitude data which the whole axisymmetric tilt framework is designed to solve is that of paleo-slope: the slope of the surface upon which a lava feature was originally emplaced. The paleo-slope values that I calculate could be compared with independent constraints on paleo-slope based on the morphology of lava flows \parencite{wadge_lobes_1991, peitersen_correlations_2000, peters_lava_2021}.

\subsection{Evaluating Model Conditions}

Reservoir wall failure mechanisms.



