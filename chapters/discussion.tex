\chapter{Discussion}\label{cha:discussion}

In this chapter I first return to the guiding questions presented in Section~\ref{sec:research-questions}, focusing on how each one informs the choice of methods and the interpretation of results. Next, I compare key findings from this thesis with previous interpretations of the volcanic system. Finally, I discuss possibilities for future work refining this method and applying it to other volcanic systems. 

\emph{Are discordant lava features consistent with surface deformation resulting from magma reservoir pressure change?} The majority of novel methods in this thesis are designed to address the nuances of this question. After showing that surface attitude change (tilt) links mapped discordant features to modeled reservoir pressure change, I encounter an issue inherent to the dataset: discordant features do not imply a unique tilt. A plausible set of simplifying assumptions (Section~\ref{sec:considerations}) reduces this intractable tilt problem to a calculation whose result depends only on the position of an inflation/deflation center relative to the discordant feature (Section~\ref{sec:tilt-from-map}, esp. Figure~\ref{fig:tilt-example}). If an individual tilt calculation depends on inflation center position, so too does the spatial distribution of tilt implied by a larger population of discordant features.

Thus to answer the originally posed question, one could simply assume an inflation center position based on independent measurements and calculate the implied tilt for all samples. However, the resulting spatial distribution of tilt can only be as confident as the assumptions used to determine the inflation center. Since inflation center position itself is a crucial variable to assess, I instead choose to test a large array of center candidates. I evaluate each candidate by the degree to which its associated distribution of surface tilt is consistent with magma reservoir pressure change. In other words, I address the posed question not with an immediate answer but with a \emph{process} for reaching the most confident affirmative answer possible.

The results presented in Chapter~\ref{cha:results} (esp. Figures~\ref{fig:tiltable-offset-all},~\ref{fig:tiltable-offset-abc}, and~\ref{fig:rmse-abc}) confirm the importance of this open-ended approach. The most obvious location to select as a site of subsurface pressure change is the caldera complex center---a dead giveaway for significant depressurization observed in countless volcanic systems on the terrestrial planets. Given the topography of Olympus Mons, an additional candidate to assess is the southern summit region. For instance, \textcite{mouginis-mark_late-stage_2019} hypothesizes inflation centered in this region to be responsible for nearby discordant features. However, the caldera complex center is unsuitable for explaining discordance due to the orientation of the flows radially away from it: pressure change from this center essentially cannot produce quantifiable discordance. And while the southern summit is also a plausible inflation center, \parencite[Figures~\ref{fig:tiltable-offset-abc} and~\ref{fig:rmse-abc}, also suggested by][]{mouginis-mark_late-stage_2019} it is ultimately less consistent with the discordance data than the regions east of the caldera center (Regions 1 and 2 in Figure~\ref{fig:rmse-abc}). This finding would likely have been missed under a more restricted \emph{a priori} inflation center selection. 

\emph{What subsurface pressure change event(s) best explain(s) the discordant features?} While the methods used in this thesis are largely new, this guiding question is not. \parencite{mouginis-mark_late-stage_2019} argues that inflation of the southern summit is the best explanation for the discordant features southeast of the caldera rim. Figure~\ref{fig:tiltable-offset-abc} lends some support to this interpretation: this southern summit region captures the southeast discordant features well, but the eastern caldera region does better. Results in Figure~\ref{fig:rmse-abc} initially appear even more favorable for the southern summit hypothesis: this region consistently shows the best fit (as inferred from low \acs{RMSE}) with a consistent inflation (positive \acs{epv}) signal. However, careful inspection of actual tilt-distance datasets (Figures~\ref{fig:region1-analytical},~\ref{fig:region2-analytical}, and~\ref{fig:region34-analytical}) reveals that this goodness of fit is largely due to the smaller population of tiltable/offset samples, not a more convincing tilt signature. Ultimately, the southern inflation hypothesis is less consistent with the discordant flows than a \emph{deflation} explanation centered in or near the eastern caldera rim.

Proceeding to fit numerical tilt results to a promising center candidate in the eastern caldera rim region quickly reveals what this method can and cannot say about the underlying reservoir. Depth and underpressure (whether expressed in \unit{\mega\Pa}, $pmult$, or \acs{epv}) are difficult to disentangle from one another; deep high-pressure-change events yield similar surface tilt signatures as shallow low-pressure-change events. However, reservoir size and shape are better constrained regardless of depth/underpressure: the reservoir size in map view is comparable to the caldera complex itself, and highly oblate. 

It's worth emphasizing that these model results, e.g. the footprints in Figure~\ref{fig:reservoir-footprint}, should not be interpreted literally as an actual reservoir, but instead as a way of representing the relationship between the actual reservoir and the discordant samples in question. Its center position on the eastern caldera rim may suggest that a greater volume of magma evacuated from this region of the reservoir than it did elsewhere. Also note that the radius of this reservoir footprint is likely not uniform in every direction, since it was calculated to explain only the discordant features which occur south of the inflation center.

% Applying this method to the reservoir/summit scale requires consideration of two complicating factors. First, the deformation signature of plausible explanatory processes (e.g., caldera collapse, reservoir inflation) varies sharply over the spatial scale of the discordant features themselves. For example, consider the numerical tilt solution shown in Figure~\ref{fig:mogi-test-shallow-oblate}. For a caldera-scale reservoir, surface tilt varies from \ang{0} to nearly \ang{-3} nearly back to \ang{0} all within \qty{50}{\km} of the reservoir center. Many lava flows and linear channels are of comparable or even greater lengths (Figure~\ref{fig:features}). Therefore, in this thesis I sample points along each feature to perform tilt calculations at finer spatial resolution than those necessary for regional lithospheric flexure.
% Second, the symmetry of the summit region itself poses a significant conundrum. The nested caldera complex provides direct evidence for the central location of major subsurface pressure change events. Additionally, the overwhelming tendency of oriented lava features (lobate flows and linear channels) is to point directly away from this caldera complex. The conundrum is in the coincident positions of inflation/deflation center and.


% The starting point is an array of tilt-distance datasets corresponding to multiple possible inflation centers, sorted by their tiltable/offset fraction (how well they capture the discordant features) and \acs{RMSE} (how well they capture a simplified analytical tilt solution). I justify this approach on the grounds that a wide range of numerical tilt solutions are qualitatively similar to this simplified analytical solution. Crucially, this includes numerical solutions whose reservoir configurations violate the analytical assumption of a deep, point-like reservoir. The tradeoff is that the quantitative parameters identified in the analytical regression are incorrect when analytical assumptions are violated. Therefore, subsequent numerical forward modeling is necessary to identify plausible reservoir configurations corresponding to a particular fit.

% Could be because my model assumes axisymmetry, but not likely because in almost all cases (except region 4) the discordant features occupy a limited angular scope relative to the modeled center. Hard to imagine a reservoir which is asymmetric enough to affect the calculations. What CAN be plausible is multiple rounds of inflation. This would violate teh assumptions of my method because the superposition of two inflation events (even if they are both roughly axisymmetric/horizontal tilt axis) would not necessarily have a horizontal tilt axis if the two events have different centers.

\section{Future Refinements}

This thesis is a proof-of-concept for an array of new methods, particularly the calculation of tilt necessary to explain a discordant sample about an axis imposed by a particular inflation center. To carry out this calculation, it is necessary to sample an array of points from each discordant linear feature. In map view, it is typically easy to visually interpolate the linear features simply by examining the sampled points (e.g., Figure~\ref{fig:samples}) with the sampling interval (\samplinginterval). However, this visual interpolation is more difficult in the tilt-distance datasets (e.g., Figure~\ref{fig:region1-analytical}). In future applications of this method, it would be useful to use a finer sampling interval to be able to examine individual linear features within tilt-distance space. One benefit of this approach would be the ability to perform analytical or numerical tilt-solution fitting on individual flows rather than populations of several flows. This would address an issue present in the current iteration: flows within the populations analyzed here may not be contemporaneous despite their spatial proximity and visual similarity.

Regardless of the sampling interval, there is also room for significant expansion in the method for calculating tilt from mapped discordant features. The upper-hemisphere orthographic projection used throughout this thesis conveniently depicts tilt-paths swept through attitude space (about any horizontal tilt axis) as straight lines within a unit circle. This reduces the tilt calculation from angular distance to cartesian distance. Crucially, the orientation of these straight lines within the projected diagrams corresponds to the direction toward/away from the inflation center responsible for tilt (Figures~\ref{fig:tilt-from-map} and~\ref{fig:tilt-example}). In this thesis I consider only a single tilt event, i.e., the net attitude change between emplacement and modern observation. However, the complex asymmetric history evident at the summit of Olympus Mons suggests that discordant features are unlikely to be fully explained by a single inflation event. The aforementioned attitude representation in this thesis is suitable for incorporating the superposition of multiple tilt events corresponding to different tilt axes, as long as each one is horizontal. Qualitatively, this extension is as simple as drawing two straight lines in the orthographic projection rather than just one (Figure~\ref{fig:multiple-tilt}). Quantitatively, it would be important to independently assess plausible inflation center positions, paleo-slopes, etc., but the central tilt calculation given by Equation~\eqref{eq:tilt-from-model} would still apply.

For this study area, one logical application of this method would be to assume two pressure change centers: one within the caldera complex and one within the southern summit as suggested by \textcite{mouginis-mark_late-stage_2019}. Future work could also target multiple pressure change centers corresponding to the centers of various patera floors within the caldera complex. Combined tilt estimates may confirm the dominance of the single center identified in this thesis or reveal additional insight into a complex asymmetric magmatic system in the subsurface. 

\begin{figure}
    \floatbox[{\capbeside\thisfloatsetup{floatwidth=sidefil,capbesideposition={left,center},capbesidewidth=.7\linewidth}}]{figure}
    {\caption[Multiple tilt events]{Upper hemisphere orthographic projection of attitude space for an example discordant feature, as used throughout. The methods used in this thesis could be adapted to assess the effect of two or more successive tilt events from different inflation centers. I have considered only single straight-line paths between paleo- and modern attitudes, but any sequence of straight-line segments (blue) represents the combination of a sequence of tilt events about horizontal tilt axes. 
    }\label{fig:multiple-tilt}}
    {\begin{tikzpicture}[scale=.7]

    \coordinate (orig) at (0,0);
    \coordinate (s1) at (70:0.5*\flatradius);
    \coordinate (s2) at (-10:2);
    \coordinate (s3) at (-60:1);

    \draw (orig) circle (\flatradius);

    \draw[arrow, ultra thick] (orig) -- (70:\flatradius) node[anchor=south] {\acs{az1}};
    \fill (s3) circle (1mm) node[anchor=north] {$(\acs{az2},\acs{sl2})$};

    \draw[arrow, red] (s1) -- (s3);
    \draw[arrow, blue] (s1) -- (s2) -- (s3);


\end{tikzpicture}%}
\end{figure}

While I use a simplified ellipsoidal reservoir in an elastic medium, the tilt-based calculations in this thesis are indifferent to the physical processes responsible in surface deformation. Any numerical surface displacement solution can be converted to a tilt solution via Equation~\eqref{eq:tilt-from-model}; the same is true for analytical displacement solutions via Equation~\eqref{eq:analytical-tilt}. Similarly, any analytical tilt solution or array numerical solutions can be fit to a particular map-derived tilt-distance dataset using the processes described in this thesis. For instance, recall that late Amazonian volcanism at Olympus Mons likely spans hundreds of millions of years \parencite{neukum_recent_2004}. It may therefore be particularly important to incorporate time- and temperature-dependent host rock response \parencite[e.g.,][]{gregg_catastrophic_2012} to Olympus Mons.

\section{Fundamental Limitations}

Despite the potential to apply these methods more widely, it is important to keep in mind the fundamental limitations of the discordance method which constrain its explanatory power. First, only a fraction of this or any other study area surface is covered in flows or channels whose orientation can be reliably measured.  Additionally, these flows record an incomplete picture of the topographic surface upon which they were emplaced---downhill azimuth but not slope. Recent progress has been made toward independent paleo-slope estimates from lava flow morphology \parencite{wadge_lobes_1991, peitersen_correlations_2000, peters_lava_2021}, but the unique physical conditions of each terrestrial planet and difficulty in observing lava flow emplacement makes it difficult to benchmark these models.

Only those which were subsequently deformed off-axis (i.e., by an offset center, or a more complex inflation system with the same net tilt axis) become discordant. In this case, the majority of pressure change appears to have been minimally offset, if at all.

These flows are of different ages.

Important progress has been made toward constraining the timeline of Late Amazonian volcanic activity using crater counting \parencite{kneissl_map-projection-independent_2011,robbins_volcanic_2011,
robbins_large_2013,
platz_crater-based_2013}, although a detailed chronology is still out of reach. The difficulty inherent to the stochastic nature of this method is that small regions are difficult to date with high precision. One result from this thesis which could address this challenge is that discordant populations which are best explained by the same axisymmetric pressure change conditions are likely to have been deformed (if not emplaced) at the same time. Conversely, populations whose most likely centers differ in position and inflation energy are more likely to be stratigraphically distinct. While the method of this thesis cannot directly resolve relative ages, it could help to identify contemporaneous units and thus increase the precision of the crater counting method by expanding the area of interest.

These limitations underscore the importance of incorporating complimentary methods.