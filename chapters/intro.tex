\chapter{Introduction}

Olympus Mons is the largest volcano in the solar system, with a diameter of roughly \qty{600}{\km} (Figure~\ref{fig:edifice}) and a summit height of over \qty{20}{\km}~\parencite{plescia_morphometric_2004}. It is nearby and morphologically similar to the three Tharsis Montes (Arsia, Pavonis, and Ascraeus Montes) but considered distinct as it lies outside the Tharsis Plateau~\parencite{bleacher_trends_2007}. The surrounding Tharsis region is home to numerous additional volcanic features at a wide range of spatial scales.

\section{Martian Geologic Context}

One notable differences is that the martian scale has finer gradations \textit{early} in its history, (within the first \qty{\sim1.5}{Gyr}) reflecting initial bombardment and dynamic volcanic, geophysical, \& sedimentary processes, while later eons have been more quiescent. 

Three broad periods of Martian history are identified~\parencite{carr_geologic_2010}. First, the Noachian period lasted roughly from \qtyrange{4.1}{3.7}{Ga}.\footnote{As is the case on Earth, no direct evidence is available for the earliest times which presumably predate complete resurfacing by impacts and magmatism.} During this period, volcanic activity was common across the planet's surface. The Tharsis Plateau and Olympus Mons edifices were likely already under construction by the end of the Noachian~\parencite[][etc.]{isherwood_volcanic_2013,broquet_gravitational_2019}. Next, the Hesperian period \qtyrange{3.7}{3.1}{Ga} saw the end of \ac{LHB}, intermittent flooding but overall reduced prevalence of liquid water, and initial concentration of volcanism within Tharsis and Elysium volcanic provinces. Finally, the ongoing Amazonian period (\qty{\sim3.1}{Ga}--present) has been characterized by intermittent localized volcanism within the major provinces \parencite[e.g.,][]{grott_long-term_2013}, limited fluvial activity, and slow sedimentary processes.

The most precise absolute dating methods using radioisotopes are largely unavailable because few martian samples exist on Earth. Nonetheless, radiochronometry of martian meteorites has revealed volcanic eruptions within the last \qty{\sim1.5}{Gyr}~\parencite[e.g.,][]{cohen_taking_2017}. For assigning dates to specific surfaces, researchers count accumulated craters, which can often constrain absolute ages to within a factor of \num{\sim2}~\parencite[e.g.,][]{kneissl_map-projection-independent_2011}.

\section{Evidence for Late Amazonian Volcanism}

\subsection{Tharsis Province}

Besides being home to many of the largest volcanoes in the solar system, the Tharsis volcanic province is of scientific interest for to its relatively recent Late Amazonian activity---within the last \qty{200}{Myr}.

Researchers have examined volcanic processes and features of the Tharsis province from a variety of angles. For instance, numerical models are used to understand deep magma supply in terms of mantle-lithosphere interactions. \textcite{roberts_plume-induced_2004} find that dynamic (i.e., convective) support is unrealistic given the high observed geoid/topography ratio in the region, favoring a volcanically thickened lithosphere and leaving open the question of magma supply. While \textcite{redmond_numerical_2004} concur that the topographic and geoid anomaly is best explained by volcanically modified lithosphere, they find plausible support for a convective plume contribution with partial melting sufficient to explain late Amazonian volcanism, a conclusion later corroborated by \textcite{plesa_thermal_2018}.

Direct evidence for Late Amazonian volcanism in the Tharsis province includes global deposits consistent with major explosive eruptions \parencite{hynek_explosive_2003}, distributed low-viscosity lava flows \parencite{hauber_very_2011}, smaller vents \parencite{wilson_fissure_2009,richardson_small_2021} low-relief shield volcanoes analogous to the terrestrial Snake River Plains \parencite{hauber_topography_2009}, and dike swarms \parencite{pieterek_late_2022}.

\subsection{Olympus Mons}

At the edifice scale, \textcite{mcgovern_state_1993} predict that flexural loading redirects magma horizontally as it ascends through the large volcanic edifices, gradually favoring flank eruptions. Their visco-elastic model is also consistent with observed circumferential normal faults as interpreted by \textcite{thomas_flank_1990}. At the summit, \textcite{zuber_caldera_1992} use a numerical model to constrain the physical characteristics of a magma reservoir within the Olympus Mons edifice based on mapped summit topography and tectonic features. Unlike the Tharsis Montes, whose prominent caldera systems are thought to span several hundred million years in age, all patera floors of Olympus Mons' caldera have been crater-dated to within the \qtyrange{100}{200}{Ma} range \parencite{neukum_recent_2004}.

Presumably associated with caldera formation, \textcite{basilevsky_geologically_2006} identified lava flows on the eastern flank with crater-based ages in the same \qty{<200}{Ma} interval.

Olympus Mons likely experienced two phases of eruptive activity: an earlier stage characterized by more stable tube-forming eruptions and a later stage characterized by more episodic channel-forming eruptions \parencite{bleacher_olympus_2007,peters_flank_2017}. 

Finally, it is worth pointing out that \textcite[e.g.,][]{black_eruptibility_2016} estimate a higher intrusive/extrusive magmatism ratio on Mars than on Earth, suggesting that observed eruptive products are likely associated with significant subsurface activity. 

\section{Topographic Discordance}

Numerous lava flows around the summit caldera of Olympus Mons appear to flow uphill~\parencite[Figure~\ref{fig:uphill-flows}; after][]{mouginis-mark_late-stage_2019}. In this thesis, the term \ac{disc} is defined as the angular difference between a lava flow's downhill azimuthal direction and the local dip direction of the underlying surface. For example, a flow pointing directly uphill would have $\acs{disc}=\ang{180}$. 

Discordant lava features have long been used to infer surface deformation under the assumption that lava flows record the downhill direction at the time of their emplacement. This method has been applied at a regional scale to constrain flexural loading of the lithosphere from large volcanic edifices \parencite{mouginis-mark_ancient_1982,isherwood_volcanic_2013,chadwick_late_2015} and more recently to reservoir-scale processes at the summit of Olympus Mons \parencite{mouginis-mark_late-stage_2019}.

\begin{figure}
    \centering
    \includegraphics[width=\textwidth]{edifice.pdf}
    \caption[Olympus Mons]{The Olympus Mons edifice is \qty{\sim600}{\km} in diameter and its summit on the southern flank is more than \qty{20}{\km} high. It features a nested caldera complex more than \qty{50}{\km} in diameter and \qty{\sim2}{\km} deep. Main map is a hillshade (shaded relief) derived from \acs{MOLA} elevation data; inset of entire martian surface is \acs{MOLA} elevation. Contours in \unit{\km}.}\label{fig:edifice}
\end{figure}

\begin{figure}
    \centering
    \includegraphics[width=\textwidth]{uphill-flows.pdf}
    \caption[Discordant lava flows]{Several lava flows (white) in the southern flank region appear to flow uphill toward the modern summit. \acs{CTX} mosaic basemap. Contours in \unit{\km}. After Figure 3 in \textcite{mouginis-mark_late-stage_2019}.}%
    \label{fig:uphill-flows}
\end{figure}

\section{Planetary Perspective}

The study of Olympus Mons contributes to a more comprehensive understanding of volcanic processes than could be achieved with a solely Earth-based inquiry. Compared to Earth, the unique challenges of studying martian volcanoes are obvious: the inability to make in-situ measurements and the lack of evidence for ongoing eruptive activity stand out. 

However, there are important advantages as well. Most notably, surface deformation signatures are well preserved in the martian environment over hundreds of millions of years, especially at high elevations where weathering and erosion are far less active. This allows for a more comprehensive picture of deformation history than is commonly available on Earth, where large caldera-forming edifices are rare and often poorly preserved. Additionally, the unique martian conditions that yield the largest volcanoes in the solar system broaden our perspective of how volcanism works in general.

\begin{figure}
    \centering
    \includegraphics[width=\textwidth]{summit.pdf}
    \caption[Study area: Olympus Mons summit]{Study area at the summit of Olympus Mons (inset). Sinusoidal Martian Projection. Contours in \unit{km}. Square is \qtyproduct{200 x 200}{\km}, centered at the midpoint of the outermost \qty{19}{\km} contour.}\label{fig:summit}
\end{figure}

\section{Research Questions}

The following questions will inform the research project: 
\begin{enumerate}
    \item can discordant lava flows be used as reliable planetary ``tiltmeters'' to provide insight into subsurface activity?
    \item what patterns of regional surface deformation do discordant lava flows imply?
    \item what combinations of reservoir processes and geometries best explain inferred surface deformation?
\end{enumerate}

\section{Hypotheses}

I suspect:
\begin{enumerate}
    \item Recognizing the incompleteness of underlying attitude information, discordant lava flows will be able to provide limited insight into subsurface activity.
    \item Discordant flows will reveal limited surface deformation restricted to the immediate vicinity of the arcuate crest south of the caldera complex.
    \item Inflation under the southern crest will account for observed tilt better than subsidence at the caldera complex itself. 
\end{enumerate}

% \section{Challenges \& Assumptions}
% The topographic and structural deformation implies a combination of underlying volcanic processes including a.\ caldera collapse (i.e., magmatic discharge) and b.\ subsequent reservoir inflation (i.e., magmatic recharge). Untangling the superposition of these competing factors is challenging in itself, although preliminary inspection of the study region (Figure~\ref{fig:summit}) indicates that these effects may be localized to a.\ the immediate circumferential region, and b.\ the southern flank, respectively. Additionally, \textit{independent}\footnote{That is, in addition to observed deformation.} stratigraphic constraints on lava flow emplacement relative to the timeline of deformation are unavailable---while crater-dating can be done on lava flows and caldera floors, the timing of reservoir inflation does not leave a temporal record observable from spacecraft. Therefore, certain simplifying assumptions are necessary to make progress with this inquiry. In particular, with only ``initial'' (at flow emplacement) and ``final'' (modern) data points available for each location, only net change can be inferred.

% \begin{figure}
    % \centering
    % \includegraphics[width=\textwidth]{mars.pdf}
    % \caption[Martian topography]{\acs{MOLA}-derived global topography of Mars. Olympus Mons lies exactly on the central meridian, between northern lowlands (green) and southern highlands (orange). Elevation in \unit{\m}. Horizontal scale distances computed at equator.}
    % \label{fig:mars}
% \end{figure}