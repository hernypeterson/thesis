\chapter{Introduction}

Olympus Mons (Figure~\ref{fig:edifice}) is the largest volcano in the solar system, with a diameter of roughly \qty{600}{\km} and a summit height of over \qty{20}{\km} \parencite[e.g.,][]{plescia_morphometric_2004}. At its summit is a caldera complex, \qtyrange{65}{80}{\km} in diameter, comprised of six overlapping ``patera'' floors. It is closely related spatially and morphologically to the Tharsis Montes (Arsia, Pavonis, and Ascraeus) but considered distinct because it lies outside the Tharsis Plateau~\parencite[e.g.,][]{carr_volcanism_1973}.

\begin{figure}
    \centering
    \includegraphics[width=\textwidth]{intro/edifice.pdf}
    \caption[Olympus Mons]{The Olympus Mons edifice is \qty{\sim600}{\km} in diameter and its summit on the southern flank is more than \qty{20}{\km} high. It features a nested caldera complex more than \qty{50}{\km} in diameter and \qty{\sim2}{\km} deep. Main map is a hillshade (shaded relief) derived from \acf{MOLA} elevation data\parencite{smith_mars_2001}; inset of entire martian surface is also \acs{MOLA} elevation. Contours in \unit{\km}.}\label{fig:edifice}
\end{figure}

\section{Significance of Planetary Volcanology}

Studying this volcano contributes to a more comprehensive understanding of volcanic processes than could be achieved with a solely Earth-based inquiry. Compared to Earth, the unique challenges of studying martian volcanoes are obvious: the inability to make in-situ measurements and the lack of evidence for ongoing eruptive activity stand out. 

However, there are important advantages as well. Most notably, the surface is well preserved in the martian environment over hundreds of millions of years, especially at high elevations where weathering and erosion are far less active than on Earth. This allows satellites to make precise measurements of surface characteristics including geologic contacts and topography. These measurements allow for important investigations into some of the best-preserved large caldera complexes anywhere in the solar system, an important subject for understanding the hazards associated with major eruptions on Earth. Additionally, the unique martian conditions that yield the largest volcanoes in the solar system broaden our understanding of general volcanic processes. Therefore I begin introducing some key methods for assessing this martian geologic context and the insights they have derived.

\section{Martian Geologic Context}

The history of Olympus Mons spans most of the martian geologic timescale, so it is worth introducing the three major divisions of martian deep time \parencite{carr_geologic_2010}. First, the Noachian period lasted roughly from \qtyrange{4.1}{3.7}{Ga}.\footnote{As is the case on Earth, no direct evidence is available for the earliest times which likely predate complete resurfacing by impacts and magmatism.} During this period, volcanic activity was common across the planet's surface. The Tharsis Plateau and Olympus Mons edifices were likely already under construction by the end of the Noachian~\parencite[cf.][]{isherwood_volcanic_2013,broquet_gravitational_2019}. Next, the Hesperian period (\qtyrange{3.7}{3.1}{Ga}) saw the end of \acl{LHB}, intermittent flooding, and incipient spatial concentration of volcanism within the Tharsis and Elysium provinces. Finally, the ongoing Amazonian period (\qty{\sim3.1}{Ga}--present) has seen intermittent localized volcanism within the major provinces \parencite[e.g.,][]{grott_long-term_2013}, limited fluvial activity, and slow sedimentary processes. These dates are mostly estimated by counting craters accumulated within a surface; this method can constrain absolute ages to within a factor of \num{\sim2}~\parencite[e.g.,][]{kneissl_map-projection-independent_2011}.

\section{Volcanism in the Tharsis Province}

Besides being home to many of the largest volcanoes in the solar system, the Tharsis volcanic province is of scientific interest due to its relatively recent Late Amazonian activity---within the last \qty{200}{Myr}.

Researchers have examined volcanic processes and features of the Tharsis province at a wide range of spatial and temporal scales. For instance, numerical models are used to understand deep magma sources of recent volcanism in terms of mantle-lithosphere interactions. \textcite{roberts_plume-induced_2004} find that dynamic (i.e., convective) mantle support for the topographic high is unrealistic given the high observed geoid/topography ratio in the region, favoring a volcanically thickened lithosphere and leaving open the question of magma supply. While \textcite{redmond_numerical_2004} concur that the topographic and geoid anomaly is best explained by volcanically modified lithosphere, they find plausible support for a convective plume contribution with partial melting sufficient to explain late Amazonian volcanism, a conclusion later corroborated by \textcite{plesa_thermal_2018}.

Despite the open question of magma supply, considerable direct evidence exists for Late Amazonian volcanism in the Tharsis province, including global deposits consistent with major explosive eruptions \parencite{hynek_explosive_2003}, distributed low-viscosity lava flows \parencite{hauber_very_2011}, smaller vents \parencite{wilson_fissure_2009,richardson_small_2021} low-relief shield volcanoes analogous to the terrestrial Snake River Plains \parencite{hauber_topography_2009}, and dike swarms \parencite{pieterek_late_2022}.

\section{Late Amazonian Activity of Olympus Mons}

At the Olympus Mons edifice scale, \textcite{mcgovern_state_1993} predict that flexural loading redirects magma horizontally as it ascends through the large volcanic edifices, gradually favoring flank eruptions. Their visco-elastic model is also consistent with observed circumferential normal faults as interpreted by \textcite{thomas_flank_1990}. Modern consensus generally holds that Olympus Mons experienced two phases of eruptive activity in the Late Amazonian: an earlier stage characterized by more stable tube-forming eruptions and a later stage characterized by more episodic channel-forming eruptions \parencite{bleacher_olympus_2007,peters_flank_2017}.

At the summit, \textcite{zuber_caldera_1992} use a numerical model to constrain the physical characteristics of a magma reservoir within the Olympus Mons edifice based on mapped summit topography and tectonic features. Unlike the Tharsis Montes, whose prominent caldera systems are thought to span several hundred million years in age, all patera floors of Olympus Mons' caldera have been crater-dated to within the \qtyrange{100}{200}{Ma} range \parencite{neukum_recent_2004}. The presence of several nested caldera floors dated to within such a narrow age range suggests a complex and dynamic magmatic system, the system of interest for this study.

Likely associated with this caldera formation, \textcite{basilevsky_geologically_2006} identified lava flows on the eastern flank with crater-based ages in the same \qty{<200}{Ma} interval. Finally, it is worth pointing out that \textcite[][and others]{black_eruptibility_2016} have estimated a higher intrusive/extrusive magmatism ratio on Mars than on Earth, suggesting that observed eruptive products are likely associated with significant subsurface activity. With all this evidence for dynamic magmatic systems in the relatively recent history of Olympus Mons, I introduce the key observation which sets the stage for the current inquiry.  

\section{Topographic Discordance}\label{sec:discordance}

Numerous lava flows around the summit caldera of Olympus Mons appear to flow uphill~\parencite[Figure~\ref{fig:uphill-flows}; after][]{mouginis-mark_late-stage_2019}. In this thesis, the term \ac{disc} is defined as the angular difference between a lava flow's downhill azimuthal direction and the local azimuth of the underlying surface. For example, a flow pointing directly uphill would have $\acs{disc}=\ang{180}$. 

Under the assumption that lava flows record the downhill direction at the time of their emplacement, discordant lava features have long been used to infer surface deformation responsible for this discrepancy. This method has been applied at a regional scale to constrain flexural loading of the lithosphere from large volcanic edifices \parencite{mouginis-mark_ancient_1982,isherwood_volcanic_2013,chadwick_late_2015} and more recently to reservoir-scale processes at the summit of Olympus Mons \parencite{mouginis-mark_late-stage_2019}. In this thesis, I use topographic discordance as a starting point but acknowledge the fundamental limitation that downhill azimuth of a lava feature is insufficient for fully describing the attitude of the surface upon which it was emplaced. A full description of paleo-attitude is crucial because it can be directly compared with the modern attitude to determine the degree of deformation that has taken place at the location of a discordant feature. Thus the overarching problem I set out to solve in Chapter~\ref{cha:methods} is completing this dataset in a physically plausible and volcanologically useful way. After solving this problem, I can compare the tilt between mapped paleo- and modern attitudes to the tilt resulting from modeled reservoir pressure change at depth.

\begin{figure}
    \centering
    \includegraphics[width=\textwidth]{intro/uphill-flows.pdf}
    \caption[Discordant lava flows]{Discordant lava flows (white) in the southern flank region. Basemap is the \acf{CTX} mosaic produced by \textcite{Dickson2018AGB}. Contours in \unit{\km}. After Figure 3 in \textcite{mouginis-mark_late-stage_2019}. Note: for the caldera contact basemap here (inset) and several subsequent figures I use a subset of the feature class from \textcite{mouginis-mark_geologic_2021}.}%
    \label{fig:uphill-flows}
\end{figure}

\section{Research Questions}

\begin{enumerate}
    \item Can discordant lava features serve as reliable planetary ``tiltmeters?''
    \item Will tilt inferred from discordant features reflect a coherent regional pattern?
    \item Can inferred tilt be explained in terms of underlying reservoir pressure change?
    \item If inferred tilt can be explained by reservoir pressure change, what is the most likely reservoir position vertically and horizontally? How much pressure change is likely to have occurred?
    \item What is the spatial range of influence for reservoir pressure change? Will a single large reservoir have comparable explanatory power to several smaller reservoirs? 
\end{enumerate}

\section{Hypotheses}

\begin{enumerate}
    \item Discordant lava features will serve as reliable ``tiltmeters'' if a choice of geometric assumptions can complete the incomplete attitude dataset.
    \item Tilt inferred from discordant features will be coherent for the most discordant flows but error-prone elsewhere.
    \item Within these coherent regions, tilt will be possible to explain under a simplified model of reservoir pressure change.  
    \item The specific parameter estimates will depend on the region of interest, but one likely candidate is a shallow reservoir beneath the modern southern summit to explain nearby discordant features.
    \item The influence of pressure change influence will be identifiable laterally within \qtyrange{10}{20}{\km} before being drowned out by noise from local topographic variations.
\end{enumerate}