\chapter{Methods}\label{cha:methods}

In this chapter, I first introduce key considerations which inform subsequent methods (Section~\ref{sec:considerations}) and define the study area (Section~\ref{sec:study-area}).

Next, I model the surface deformation signature of a magma reservoir pressure change event (Section~\ref{sec:modeling}). I pay particular attention to expressing surface displacement in terms of \emph{attitude} (orientation) change for direct comparison with topographically discordant features. The intermediate products of this section are two methods for computing tilt as a function of distance from the reservoir inflation center: an analytical solution to efficiently identify plausible inflation center locations and a numerical solution for estimating reservoir parameters like reservoir depth, size, shape, and pressure change.

Next, I construct tilt-distance datasets (Section~\ref{sec:mapping}) from discordant features in the study area to compare with model tilt solutions. I first compare the orientation of flow features (via imagery) with the surface attitude (via topography) of their surroundings. From these data I derive the first intermediate product of this section: a spatial distribution of discordant features in the study area. However, distance and tilt (for comparison with model solutions from Section~\ref{sec:modeling}) each depend on the position of the magma reservoir responsible for surface deformation. Since I do not know this position, I define an array of candidate center locations to test. For each center, I calculate a) distance from, and b) tilt implied at, each lava feature. The final intermediate product of this section is an array of tilt-distance datasets (one for each center candidate location) for analysis under the framework derived in Section~\ref{sec:modeling}.

Finally, I analyze and evaluate (Section~\ref{sec:evaluation}) each candidate center by the degree to which its associated tilt-distance dataset (Section~\ref{sec:mapping}) captures both a) the mapped discordant features and b) a plausible tilt-distance solution from Section~\ref{sec:modeling}. I use the spatial distribution of these evaluation results to assess the subsurface conditions most likely responsible for discordant features at the surface. After identifying the most likely reservoir position, I assess additional reservoir parameters most consistent with the tilt implied at the surface.

Unless otherwise stated, I perform all calculations in a Jupyter Notebook environment using a custom Python (v. 3.10.9) module \hltt{topodisc}, printed in Appendix~\ref{app:code}.

\section{Fundamental Considerations}\label{sec:considerations}

The ultimate goal of this thesis is to investigate the magma system below the summit of Olympus Mons. Practically speaking, this means explaining the peculiar geometry and spatial distribution of discordant flows in terms of subsurface magma reservoir pressure change. I approach this inquiry under the assumption that discordant lava features and subsurface pressurization each imply net change in surface attitude (orientation). In this section I lay the foundations for a quantitative treatment of surface attitude change.

\begin{figure}
    \begin{tikzpicture}[scale=1,tdplot_main_coords]
    % origin
    \coordinate (A1) at (8,8,0);
    \coordinate (B1) at (0,8,1);
    \coordinate (C1) at (0,0,1);
    \coordinate (D1) at (8,0,0);

    \coordinate (B0) at (0,8,0);
    \coordinate (C0) at (0,0,0);   

    \coordinate (flow1start) at ($(B1)!0.5!(C1)$);
    \coordinate (flow1stop) at ($(A1)!0.5!(D1)$);

    \coordinate (A2) at (8,8,2);
    \coordinate (B2) at (0,8,6);
    \coordinate (C2) at (0,0,6);
    \coordinate (D2) at (8,0,2);

    \coordinate (flow2start) at ($(B2)!0.5!(C2)$);
    \coordinate (flow2stop) at ($(A2)!0.5!(D2)$);

    \coordinate (A3) at (8,8,7);
    \coordinate (B3) at (0,8,7);
    \coordinate (C3) at (0,0,9);
    \coordinate (D3) at (8,0,9);

    \coordinate (flow3start) at ($(B3)!0.5!(C3)$);
    \coordinate (flow3stop) at ($(A3)!0.5!(D3)$);

    \draw (A1) -- (B0) -- (C0) -- (D1) -- (A1);
    \draw (A1) -- (B1) -- (C1) -- (D1);
    \draw (C0) -- (C1);
    \draw (B0) -- (B1);

    \fill[blue, opacity = 0.2] (A1) -- (B1) -- (C1) -- (D1);

    \foreach \x in {0,0.33333333333,...,1}{
        \draw[blue, ultra thick] ($(A1)!\x!(B1)$) -- ($(D1)!\x!(C1)$);
    };

    \draw[arrow, ultra thick] (flow1start) -- (flow1stop);
    
    \draw (A2) -- (B2) -- (C2) -- (D2) -- (A2);
    \draw[arrow, gray] (A1) -- (A2);
    \draw[arrow, gray] (B1) -- (B2);
    \draw[arrow, gray] (C1) -- (C2);
    \draw[arrow, gray] (D1) -- (D2);

    \draw[arrow, ultra thick] (flow2start) -- (flow2stop);

    \fill[gray, opacity = 0.2] (A2) -- (B2) -- (C2) -- (D2);

    \foreach \x in {0,0.33333333333,...,1}{
        \draw[gray, ultra thick] ($(A2)!\x!(B2)$) -- ($(D2)!\x!(C2)$);
    };

    \draw (A3) -- (B3) -- (C3) -- (D3) -- (A3);
    \draw[arrow, green!70!black] (A2) -- (A3);
    \draw[arrow, green!70!black] (B2) -- (B3);
    \draw[arrow, green!70!black] (C2) -- (C3);
    \draw[arrow, green!70!black] (D2) -- (D3);

    \draw[arrow, ultra thick] (flow2start) -- (flow2stop);

    \fill[green!70!black, opacity = 0.2] (A3) -- (B3) -- (C3) -- (D3);

    \foreach \x in {0,0.33333333333,...,1}{
        \draw[green!70!black, ultra thick] ($(A3)!\x!(D3)$) -- ($(B3)!\x!(C3)$);
    };

    \draw[arrow, ultra thick] (flow3start) -- (flow3stop);

\end{tikzpicture}%%
    \caption{Discordance Concept}%
    \label{fig:discordance-concept}
\end{figure}

% This would be a great place to add a figure that we've discussed for a while -- a conceptual depiction of how a reservoir change that introduces tilt variation can be measured using a difference in lava flow attitude. This would be a schematic diagram, not one that uses actual data. Part A would show a topographic surface upon which a lava flow is emplaced downhill. Part B would then show an event that changes the surface tilt such that the lava flow attitude is changed. If you wanted to make the figure more powerful, you could include instead B and C, where you illustrate a change that introduces only vertical tilt change (basically, one where the center of axisymmetry remains unadjusted) and one where both vertical and horizontal change has been induced (presumably because the details of the inflation are not axisymmetric or because the center of axisymmetry has shifted).

\subsection{Attitude Representation}\label{sec:attitude-representation}

The central role of surface attitude warrants a graphical representation suitable for both conceptual illustration and geometrically-grounded calculation. The most common such representation is the stereonet: planar surfaces are represented by their stereographically\footnote{from the top of the sphere onto a plane tangent to the bottom of the sphere.} projected intersections with the lower hemisphere of a unit sphere.

I instead use an orthographic\footnote{from directly above; orthogonal to the plane.} projection to send small circles\footnote{whose axis lies on the primitive (horizontal) great circle} to straight lines. These small circles are crucial for the derivation in Section~\ref{sec:tilt-from-map} because they represent the paths swept out by rotation about a horizontal axis. I also represent planes by their unique perpendicular ``poles,'' not their curvilinear intersections. Finally, I use an upper hemisphere projection so each pole has the same azimuth as the corresponding \emph{downhill} direction. I illustrate this surface attitude representation in Figure~\ref{fig:surface}.

\begin{figure}
    \begin{tikzpicture}[scale=4.4,tdplot_main_coords]

% origin
\coordinate (O) at (0,0,0);

% also defines (Pxy), (Pxz), (Pyz), etc.
\tdplotsetcoord{P}{\radius}{\ze}{\az}

% fill flat surface
\fill[color = gray!10!white] (0,0) circle (0.5*\radius);

% define tilted surface
\tdplotsetrotatedcoords{\az}{\ze}{0}

% fill tilted surface
\fill[tdplot_rotated_coords, color = green!40!black, opacity=0.4] (0,0) circle (0.4*\radius);

% downhill line
\draw[arrow, tdplot_rotated_coords] (0,0) -- (0:0.4*\radius);

% horizontal surface (front right)
\fill[color = gray!10!white, opacity=0.6] (\az:0.5*\radius) arc (\az:\az+90:0.5*\radius) -- (0,0);

% perpendicular corners
\draw[tdplot_rotated_coords] (0.25,0,0) -- (0.25,0,0.25) -- (0,0,0.25) -- (0,-0.25,0.25) -- (0,-0.25,0);

% horizontal surface (front left)
\fill[color = gray!10!white, opacity=0.6] (\az-90:0.5*\radius) arc (\az-90:\az:0.5*\radius) -- (0,0);

% z axis
\draw[axis] (O) -- (0,0,0.5*\axislength) node[anchor=south]{$z$};

% line az surface
\draw[very thin, dashed,green!40!black] (P) -- (Pxy) -- (O);
\draw[arrow, green!40!black] (O) -- (P) node[anchor = south west] {};

% north axis
\draw[axis] (O) -- (0,0.4*\axislength,0) node[anchor=west]{\acs{north}};

% az angle label
\tdplotdrawarc{(O)}{0.4*\radius}{\az}{90}{coordinate, pin={[pin edge={black},-]-60:\acs{az2}}}{}

% az surface
\tdplotsetthetaplanecoords{\az}

% ze angle label
\tdplotdrawarc[tdplot_rotated_coords]{(O)}{0.4*\radius}{0}{\ze}{coordinate, pin={[pin edge={black},-]80:\acs{sl2}}}{}

% ze angle label
\tdplotdrawarc[tdplot_rotated_coords]{(O)}{0.4*\radius}{90}{90+\ze}{coordinate, pin={[pin edge={black},-]180:\acs{sl2}}}{}

\fill[black] (O) circle (0.2pt);

\end{tikzpicture}%
    \caption[Orthographic pole to plane]{Pole to plane in an upper hemisphere orthographic projection. \textbf{Left:} A sloped green surface compared to a flat grey surface. A single normal vector within the upper hemisphere is perpendicular to the green surface. The zenith angle of this vector is the same as the slope of the surface $(\varphi)$. The unit vector in this direction is the pole (point). \textbf{Right:} When this pole is projected vertically (orthographic) onto the flat surface, the resulting point retains the downhill azimuth direction $(\theta)$ at a distance $\sin\varphi$ from the origin.}%
    \label{fig:surface}%
\end{figure}

\subsection{Simplifying Assumptions}

While I observe discordant features via satellite-derived imagery and topography, I can only infer subsurface pressure change. To introduce a hypothetical reservoir and model surface attitude change resulting from its activity, I need to answer several questions. To name just a few: Where is the reservoir in map view? How deep within/below the Olympus Mons edifice? How big? What shape? How much pressure change has occurred? Inflation or deflation? Several simplifying assumptions are necessary to make these questions tractable.

\subsubsection{Physical Properties}

I model the Olympus Mons edifice material as a single homogeneous basalt. The low profile of Olympus Mons (like most martian volcanoes) is analogous to terrestrial basaltic shield volcano such as Hawai'i. In the absence of geochemical or geophysical data pertaining to the internal edifice structure, I assume uniform basaltic flow material. I use a linear elastic rheology which allows for computationally inexpensive modeling and analytical surface deformation solutions \parencite{mogi_relations_1958} without sacrificing accuracy for a wide range of volcanic processes \parencite{grosfils_elastic_2015}. Similarly, I treat the magma reservoir as an ellipsoidal body exerting a uniform pressure on the surrounding edifice for similar reasons. % repeats. you could note that this geometry, though idealized, is defensible on 
  % - field (e.g., commonality of smooth margins, geometries seem in plan view),
  % - mechanical (e.g., to consistently feed central eruptions, to emplace radial dike swarms, and to explain caldera ring fault geometries) and 
  % - thermal (irregular shapes and protuberances grade toward a heat-conserving ellipsoidal geometry over time) grounds.
Despite the simplification, this configuration could describe a wide range of plausible and relevant circumstances (Figure~\ref{fig:reservoir-configs}).

\begin{figure}
    \includegraphics[width=\textwidth]{methods/caldera-collapse.pdf}\\
    \vspace{2cm}
    \includegraphics[width=\textwidth]{methods/flank-inflation.pdf}
    \caption[Ellipsoidal reservoir configurations]{Reservoir configurations whose surface deformation signature is roughly consistent with a homogeneous ellipsoidal reservoir experiencing pressure change within an elastic edifice. Note that caldera collapse (top) could be associated with elastic response outside the collapse region itself, which could be captured in this model.} % make the elastic rim thing more clear
    \label{fig:reservoir-configs}
\end{figure}

\subsubsection{Axisymmetric Geometry}

The assumption of axisymmetry (rotational symmetry about a vertical axis) is widely incorporated in models of magmatic systems across spatial and temporal scales \parencite[c.f.,][]{redmond_numerical_2004,ogawa_four-stage_2021,mogi_relations_1958,mctigue_elastic_1987}. As shown in Figure~\ref{fig:axisymmetry}, this geometry decreases computational and analytical complexity by reducing a 3D edifice to a representative 2D cross-section. In this thesis, I use the assumption of axisymmetry to model surface deformation resulting from subsurface pressure change. Numerical and analytical models with this geometry each appear throughout the analysis. Additionally, I show in Section~\ref{sec:tilt-from-map} that the axisymmetry assumption solves the attitude data incompleteness problem described in Section~\ref{sec:discordance}.

\begin{figure}
    \begin{tikzpicture}[scale=1.5,tdplot_main_coords]

% origin
\coordinate (orig) at (0,0,0);
\coordinate (summit) at (0,0,2.5);

\coordinate (bottomcorner) at (0,3*\radius,0);
\coordinate (topcorner) at (0,3*\radius,1);
\coordinate (topcorneropp) at (0,-3*\radius,1);
\coordinate (foot) at (0,\radius,1);
\coordinate (footopp) at (0,-\radius,1);

\coordinate (arrow) at (0,1.5*\radius,.5);
\coordinate (arrowopp) at (0,-1.5*\radius,.5);

% also defines (Pxy), (Pxz), (Pyz), etc.
\tdplotsetcoord{P}{\radius}{\ze}{\az}

% draw bottom surface
\draw (0,0,0) circle (3*\radius);
\fill[opacity=0.05] (0,0,0) circle (3*\radius);

% back half line
\draw[red,arrow] (arrow) arc (90:450:1.5*\radius);
\draw (foot) arc (90:450:\radius);

% back half fill
\foreach \y in {90,...,269}{
    \fill[opacity=0.05] (\y:\radius) + (0,0,1) arc (\y:\y+1:\radius) -- (summit);
    \fill[opacity=0.05] (\y:3*\radius) -- (\y+1:3*\radius) --+ (0,0,1) arc (\y-360:\y-361:3*\radius);
};

\fill[opacity = 0.05] (topcorner) arc (90:270:3*\radius);

% cross section
\fill[white] (orig) -- (summit) -- (foot) -- (topcorner) -- (bottomcorner);
\draw (orig) -- (summit) -- (foot) -- (topcorner) -- (bottomcorner) -- (orig);

% axes
\draw[axis] (orig) -- (0,0,2*\axislength) node[anchor=south]{$z$};
\draw[axis] (orig) -- (0,3*\axislength,0) node[anchor=west]{$r$};

% front line
\draw[red,arrow] (arrowopp) arc (270:450:1.5*\radius);
\draw (footopp) arc (270:450:\radius);

% front fill
\foreach \z in {270,...,449}{
    \fill[opacity=0.05] (\z:\radius) + (0,0,1) arc (\z:\z+1:\radius) -- (summit);
    \fill[opacity=0.05] (\z:3*\radius) -- (\z+1:3*\radius) --+ (0,0,1) arc (\z-360:\z-361:3*\radius);
};

\fill[opacity = 0.05] (topcorneropp) arc (270:450:3*\radius);

% draw middle surfaces
\draw (0,0,1) circle (3*\radius);

\end{tikzpicture}%
    \caption[Axisymmetry]{Schematic illustration of an axisymmetric model. A 3D feature with rotational symmetry about a vertical axis is fully described by a single cross-section (white). Conceptually this surface can be swept around the axis of symmetry (red arrow) to reconstruct the 3D shape. Radial $(r)$ and vertical $(z)$ coordinates describe position within the model.}%
    \label{fig:axisymmetry}
\end{figure}
\section{Study Area Definition}\label{sec:study-area}

\subsection{Preparation of Published Data}

The \acf{CTX}\footnote{aboard the \ac{MRO} spacecraft launched in 2005 by \acs{NASA}.} captures \qty{\sim30}{\km} swaths across the entire martian surface in visible $(\lambda=\qtyrange{500}{800}{\nm})$ greyscale at \qty{\sim6}{\m} spatial resolution. \textcite{Dickson2018AGB} blended these swaths to produce a raster mosaic product (hereafter, ``\ac{CTX} mosaic'') which I use to visually identify and map lava flows and flow channels.

The \acf{MOLA}\footnote{aboard the now-retired \ac{MGS} spacecraft launched in 1996 by \acs{NASA}.} returned topography data with horizontal resolution of \qtyproduct{300 x 1000}{\m} at the equator (better at high latitudes) and elevation uncertainty of \qty{\sim3}{\m}~\parencite{smith_mars_2001}. To improve spatial resolution, additional elevation data from the \ac{HRSC}\footnote{aboard the \ac{MEX} spacecraft launched in 2003 by the \ac{ESA}} was blended to product a \ac{DEM} with \qty{200}{\m} pixel resolution. Each pixel's vertical uncertainty is \qty{\sim1}{\m}, with an additional global uncertainty of \qty{\sim1.8}{\m} in the martian areoid (martian equivalent of Earth's geoid). In this project, the global areoid uncertainty is not a concern because only one region (the summit of Olympus Mons) is considered.

I load these two data sources in an equal-area sinusoidal Mars projection\footnote{I originally used an equal-area projection with a plan to perform crater-counting calculations, which are area-dependent. I discuss in Chapter~\ref{cha:discussion} why this projection may introduce minor reliability concerns for the analysis I end up performing in the current version of this project.} in ArcGIS Pro. I define the study area as a square (in this projection) \qtyproduct{200 x 200}{\km} around at the centroid of the outermost \qty{19}{\km} contour,\footnote{This is the highest integer \unit{km} which is roughly circular and completely encloses the caldera complex, implying that it largely records the conical shape of the shield edifice without influence from subsequent caldera collapse or reservoir inflation.} as seen in Figure~\ref{fig:summit}.

\begin{figure}
    \centering
    \includegraphics[width=\textwidth]{summit.pdf}
    \caption[Summit study area]{Study area at the summit of Olympus Mons (inset). Sinusoidal Martian Projection. Contours in \unit{km}. Square is \qtyproduct{200 x 200}{\km}, centered at the midpoint of the outermost \qty{19}{\km} contour.}\label{fig:summit}
\end{figure}

Figure~\ref{fig:summit} shows important topographic patterns at the summit of Olympus Mons. More than \qty{50}{\km} from the center of the figure, topographic contours (\qtyrange{12}{19}{\km}) are roughly concentric rings. Closer to the caldera, this rotational symmetry breaks down: the caldera complex itself consists of six intersecting collapse pits. On the southern flank, we see a prominent arcuate \qty{20}{\km} contour with the topographic summit (within the \qty{21}{\km} contour) over \qty{20}{\km} from the southern caldera rim. Thus, my first task is to determine the degree to which axisymmetric inflation can apply to an edifice which is not entirely axisymmetric. It is important to point out, however, that the \emph{asymmetry} inherent to the region is crucial for the discordant flow method to function, as I describe in subsequent sections. 
\section{Modeling Reservoir Pressure Change}\label{sec:modeling}

In this section, I explore \emph{how} exactly the axisymmetric reservoir pressure change model introduced in Section~\ref{sec:considerations} would deform (change the attitude of) the surface. I pay particular attention to the effect of varying key parameters like reservoir depth, size, shape, and magnitude of pressure change. Surface deformation is expressed in terms of attitude change for direct comparison with discordant features.

My first approach to quantifying surface attitude change is a numerical, which computes of \emph{displacement} for a large array of discrete positions along an axisymmetric model surface. This approach has the benefit of incorporating nearly the full array of geometric considerations (Section~\ref{sec:considerations}), including reservoir size, aspect ratio, pressure change, and depth within/below the Olympus Mons edifice. The surface displacement solution derived from this model is easily converted to a tilt solution, i.e., angular change in surface attitude.

However, this numerical solution is really a family of solutions---each one corresponding to some combination of the aforementioned reservoir parameters. The primary goal is to find the best match between one model solution and one comparable map-derived dataset. The full implications of this problem are described in Section~\ref{sec:mapping}, but the short version is this: there are far too many map-datasets and model-derived solutions to manually search for matches.

To solve this problem, I introduce an analytical tilt solution to complement the numerical one, derived analogously from differential displacement. to complement the numerical one. This approach necessarily reduces the geometric complexity. It assumes, for example, a spherical reservoir under a flat surface rather than a more general ellipsoid under an Olympus Mons-shaped edifice. It reduces to a single equation with only two parameters: depth (to center) and inflation energy (\acs{epv}), the product of reservoir volume and pressure change. In other words, this solution cannot distinguish between reservoir $A$ and $B$ if reservoir $A$ is ten times bigger but the pressure change in reservoir $B$ is ten times bigger.

Despite these drawbacks, the equation form of this analytical solution can be fit via a least-squares regression to an arbitrary dataset (Section~\ref{sec:evaluation}). Just as a linear least-squares regression determines the slope and intercept of a linear equation which best fits an arbitrary dataset, this non-linear regression using the determines the depth and inflation energy of the analytical tilt equation which best fits an arbitrary dataset. This regression method is an efficient and mathematically robust method for evaluating a large array of map-derived tilt datasets.

% Modeling Tilt by Distance:
    % numerical: can capture mechanical response to a wide array of initial conditions (any depth, size, aspect ratio, pressure, surface effects), output is a large _displacement_ dataset
        % displacement to tilt
    % analytical: can capture only a narrow range of initial conditions (depth, inflation energy (pressure times volume), no free surface effects) BUT the equation is much more convenient than an array of data.
        % displacement to tilt 
    % results from this section: tilt as a f(distance) for any combination of depth/energy (analytical), plus a way to refine these estimates (numerical)

% \subsubsection{Axisymmetric Elastic Model}

% Both numerical and analytical solutions are axisymmetric and elastic.

% Two important simplifying assumptions are worth introducing immediately to provide context for numerical modeling. Having already introduced one of these, the axisymmetric assumption, the only point to emphasize here is that each aspect of a numerically constructed model must share the same axis of symmetry. In other words, the ellipsoidal magma reservoir must be centered directly underneath the center of the modeled edifice. Importantly, this assumption requires a single axis to describe each aspect of the model; namely, the topographic surface and the magma reservoir. Thus the model construction is similar to the plane shown in Figure~\ref{fig:axisymmetry}.

\subsection{Numerical Tilt-Distance Solution}\label{sec:numerical-tilt-solution}

% REORDER - HAS NOT BEEN MENTIONED YET

I use the numerical modeling software COMSOL Multiphysics 6.1 (COMSOL) to construct a numerical finite element representation of Olympus Mons. The axis of symmetry for this model is the vertical line through the center point discussed previously as the center of the \qty{19}{\km} contour.

The \ac{FEM} is widely used to model phenomena ranging from fluid flow, heat transfer, and mechanical stress in engineering and geology. This method involves discretizing a continuous medium by constructing a \emph{mesh:} a network of nodes connected to their neighbors by edges. Polygons enclosed by these edges are called elements. Software then solves the chosen equations to compute the variables of interest only for the nodes, whose values can then be interpolated within each element to determine the solution for any point within the continuum. Relevant to this inquiry, the solution to an elastic model is a set of \emph{displacement} vectors by which to translate each node such that all forces are balanced and mechanical equilibrium is reached.

From this point, an elevation transect due south from the study area center point to the base of the edifice \qty{\sim240}{\km} away represents the paleo-surface topography. This particular center point is derived from the summit of a topographic spline: an interpolation of a hypothetical paleo-topography based on the existing topography outside the \qty{19}{\km} contour (Figure~\ref{fig:paleo-topo}; compare with Figure~\ref{fig:summit}). This point is effectively identical to the center point of the outer \qty{19}{\km} contour, providing additional evidence for approximate axisymmetry outside this region. The topography of this outer region is assumed to be relatively unaffected by recent caldera and reservoir activity. % Why? 

\begin{figure}
    \includegraphics[width=\textwidth]{methods/paleo-topo.pdf}
    \caption[Spline-derived paleo-topography]{Paleo-topography estimated to have existed prior to caldera formation; interpolated from topography outside the \qty{19}{\km} contour. Note that both centers---one defined by the \qty{19}{\km} contour (labeled), and one interpolated from the surrounding topography (inner red contour)---are close to one another but south of the caldera complex center.}%
    \label{fig:paleo-topo}
\end{figure}

\begin{figure}
    \includegraphics[width=\textwidth]{methods/model-section.pdf}\\
    \includegraphics[width=\textwidth]{methods/model-section-zoom.pdf}%
    \caption[Axisymmetric numerical model section]{Axisymmetric numerical model cross-section (i.e., white surface in Figure~\ref{fig:axisymmetry}) of Olympus Mons edifice with example ellipsoidal reservoir and surrounding lithosphere. \textbf{Top:} Full model section. \textbf{Bottom:} Top left corner of top image enlarged to show ellipsoidal reservoir.}%
    \label{fig:model-section}
\end{figure}

I use the \hlss{Parametric Sweep} tool to perform a sequence of analyses which are identical except for specifically controlled parameters: depth to center $d$, reservoir radius $R$ (in plan view), aspect ratio (height divided by width) and a \acf{mult}:
\begin{equation}
    \acs{dP}=\acs{mult}\times\acs{rhor}\times\acs{g}\times d,
\end{equation}
where \acs{dP} is the simulated over- (positive) or under-pressure in the reservoir. For each numerical model, I calculate surface displacement in the radial and vertical direction as a function of distance from the inflation center.

COMSOL produces displacement data corresponding to individual model nodes. Figure~\ref{fig:tilt-from-model} shows how to derive tilt from the surface edge of an element before (initial) and after (displaced) modeled pressure change. The corresponding tilt equation is:
\begin{equation}
    \acs{tilt} = \arctan\left({\acs{dz1}}/{\acs{dr1}}\right) - \arctan\left(\dfrac{\acs{dz1}+\acs{ddisp_z}}{\acs{dr1}+\acs{ddisp_r}}\right).\label{eq:tilt-from-model}
\end{equation}

I define the distance value associated with this calculated tilt as the midpoint of the displaced edge. In practice, the horizontal scale for this edge (both before and after displacement) is negligible compared to the scale of the edifice, so another choice of distance within the edge (e.g., the midpoint of the initial surface) would effectively produce identical results.

\begin{figure}
    \newcommand{\uar}{3.5cm}
\newcommand{\uaz}{11cm}
\newcommand{\ubr}{5.5cm}
\newcommand{\ubz}{4.5cm}
\newcommand{\dz}{2.5cm}
\newcommand{\dr}{8cm}

\begin{tikzpicture}
    \coordinate (O) at (0,0);
    \coordinate (A_1) at (0,\dz);
    \coordinate (B_1) at (\dr,0);
    
    \path (A_1) + (\uar, \uaz) coordinate (A_2);
    \path (B_1) + (\ubr, \ubz) coordinate (B_2);
    \path (A_1) + (\ubr, \ubz) coordinate (B_2_trans);

    \path (A_1) + (0, \uaz - \ubz) coordinate (A_1_trans);
    \path (B_1) + (\ubr - \uar, 0) coordinate (B_1_trans);
    \path (B_1) + (\uar, \ubz) coordinate (B_1_trans_helper);

    \path (A_1) + (\uar, \ubz) coordinate (O_disp);

    \draw[-latex] (O) --+ (-0.5,0) node[anchor=east] {\acs{center}};

    \draw[] (O) -- node[fill=white] {\acs{dr1}} (B_1);
    \draw[] (O) -- node[fill=white] {$-\acs{dz1}$} (A_1);

    \draw[ultra thick] (A_1) -- node[sloped,fill=white] {Original Surface} (B_1);

    \draw[arrow] (A_1) -- node[sloped, fill=white] {$\acs{disp_a}$} (A_2);
    \draw[arrow] (B_1) -- node[sloped, fill=white] {$\acs{disp_b}$} (B_2);
    \draw[arrow] (A_1) -- node[sloped, fill=white] {$\acs{disp_b}$} (B_2_trans);
    \draw[-latex] (A_2) -- node[sloped, fill=white] {$\acs{ddisp}$} (B_2_trans);
    
    \fill[blue, opacity = 0.15] (B_1) --++ (162.7:2.5) arc (162.7:180:2.5);
    \path (B_1) + (172:2) node {\acs{proj1}};
    \draw[thick, dashed] (B_2) -- (B_2_trans);
    
    \draw[] (A_1) -- node[fill=white] {$-\acs{ddisp_z}$} (A_1_trans);
    \draw[] (B_1) -- node[fill=white] {$\acs{ddisp_r}$} (B_1_trans);
    
    \draw[dashed, thin] (O_disp) -- (A_1);
    \draw[dashed, thin] (A_2) -- (A_1_trans);
    \draw[dashed, thin] (B_2) -- (B_1_trans);
    \draw (B_1_trans_helper) --  node[fill=white] {\acs{ddisp_r}} (B_2);
    
    \fill[red, opacity = 0.3] (B_2) --++ (138:2.75) arc (138:162.7:2.75);
    \path (B_2) + (150:2) node {\acs{radial-deform}};
    \draw[thick, dashed] (B_2) -- (B_2_trans);
    
    \fill[blue, opacity = 0.15] (B_1_trans) --++ (138:2.75) arc (138:180:2.75);
    \path (B_1_trans) + (160:2.2) node {\acs{proj2}};
    \draw[ultra thick] (B_2) -- (B_2_trans);
    
    
    \draw[ultra thick] (A_1_trans) -- (B_1_trans);
    
    \draw[] (A_2) -- node[fill=white] {$-\acs{ddisp_z}$} (O_disp) -- node[fill=white] {$\acs{ddisp_r}$} (B_2_trans);
    \draw[ultra thick] (A_2) -- node[sloped,fill=white] {Deformed Surface} (B_2);

    \draw[dashed] (O_disp) -- (B_1_trans_helper) -- (B_1);
    
    \fill[black] (A_1) circle (2pt) node[anchor = east] {$A_1$};
    \fill[black] (B_1) circle (2pt) node[anchor = north] {$B_1$};
    \fill[black] (A_2) circle (2pt) node[anchor = south] {$A_2$};
    \fill[black] (B_2) circle (2pt) node[anchor = west] {$B_2$};
    \fill[black] (A_1_trans) circle (2pt);
    \fill[black] (B_1_trans) circle (2pt);
    
\end{tikzpicture}%
    \caption[Tilt from numerical modeling]{Cross-sectional view of a surface edge in the axisymmetric numerical model. As in Figure~\ref{fig:tilt-from-map}, the $r-$axis points away from the inflation center. During the model, the two nodes $A_1$ and $B_1$ of a surface element are displaced along \acs{disp_a} and \acs{disp_b} to reach a final position at $A_2$ and $B_2$, respectively. I illustrate the difference $\acs{disp_b} - \acs{disp_a} = \acs{ddisp}$ in terms of its components \acs{ddisp_r} and \acs{ddisp_z}. Notice that $z-$component terms are negative to ensure that tilt away from the inflation center yields a positive tilt \acs{tilt}, as shown in red. To calculate \acs{tilt}, I determine the slopes of segments $\overline{A_1B_1}$ and $\overline{A_2'B_1}$ using the labeled horizontal and vertical segments, convert these slopes to angles, and take their difference.}%
    \label{fig:tilt-from-model}%
\end{figure}

% \subsubsection{Physical Considerations}

In Section~\ref{sec:modeling}, I explain that the axisymmetric numerical model edifice is constructed from a particular elevation transect measured from the summit of Olympus Mons. Relative to this point, surrounding topography is roughly axisymmetric, especially more than \qty{\sim50}{\km} away past the previously discussed \qty{19}{\km} contour.

However, I do not assume that the subsequent reservoir pressure change was centered at this same location. Instead, my goal is to  determine this inflation center location on the basis of discordant flow data, wherever they may point.

Importantly, placing a reservoir inflation center anywhere off the axis of symmetry in the numerical model introduces some inaccuracy. To make progress I need to determine whether this error is negligible; if it is not, I need to be aware of its magnitude as I interpret results.

Of course, an axisymmetric model by definition does not permit any non-axisymmetric elements to be introduced. Instead, I construct a flat (no edifice, horizontal surface) variant of the model as an end-member case for the topographic variation introduced by shifting the reservoir within the edifice.\footnote{This model does not directly address the issue of introducing uphill slope away from the inflation center, but the magnitude of any error introduced should be similar.}

Additionally, \textcite{grosfils_magma_2007} showed that incorporating gravitational loading is unnecessary for modeling surface displacement. To confirm this, I run a test model under three conditions:
\begin{enumerate}
    \item no gravitational loading with magma reservoir overpressure \label{g0p1}
    \item gravitational loading (lithostatic pre-stress) with no reservoir overpressure\label{g1p0}
    \item gravitational loading with reservoir overpressure \label{g1p1}
\end{enumerate}
If gravitational loading is insignificant, displacement in case~\ref{g0p1} should match the component remaining in case~\ref{g1p1} after the gravitational component from case~\ref{g1p0} is subtracted out. The reason I need to subtract out this component is that the modeled edifice is not perfectly flat. Therefore, vertical loading is not initially in equilibrium under the reservoir and some ``slumping'' occurs to accommodate this imbalance. Preliminary test results are shown in Figure~\ref{fig:grav-topo-test}.

% TODO: fix units in these plots

\begin{figure}
    \includegraphics[width=\textwidth]{methods/grav-topo-test.pdf}
    \includegraphics[width=\textwidth]{methods/grav-topo-test-zoom.pdf}%
    \caption[Numerical model sensitivity to topography and gravity]{Comparison between topographic and flat model with and without gravitational loading. Patterns are consistent across a range of model parameters tested; a single representative case is plotted above, with the peak shown in more detail below. Gravitational loading makes essentially no difference, and the flat model (topo = False) tends to underestimate tilt by at most a few percent.}%
    \label{fig:grav-topo-test}%
\end{figure}

As expected for the flat model, gravitational loading has no effect on surface displacement and thus the subsequently calculated tilt is identical. The topographically accurate model shows the same pattern with respect to gravitational loading. There is a small but noticeable difference in tilt between the flat and topographic models, which ranges between ($0\%-10\%$) for the parameters I examined. This result places a reassuring upper bound on the magnitude of error that could be introduced by varying the horizontal location of a modeled inflation center within the summit region---likely much less than error introduced elsewhere.

\subsection{Analytical Tilt-Distance Solution}\label{sec:analytical-tilt-solution}

The numerical solutions of Section~\ref{sec:numerical-tilt-solution} each predict a particular signature of surface attitude change resulting from a particular set of subsurface reservoir conditions. More convenient for this thesis would be to invert this process---estimating subsurface conditions from surface attitude change derived from observations at the surface. A simplified analytical solution helps to efficiently narrow down a large empirical tilt dataset (introduced in Section~\ref{sec:mapping}); subsequent refinements and interpretations can be made with the help of numerical solutions.

I derive this analytical tilt solution from the widely cited \emph{displacement} solution of \textcite{mogi_relations_1958}. This so-called Mogi model assumes a deep spherical reservoir within a flat (``topo = False'') elastic half-space. Equation~\eqref{eq:tilt-from-model} in such a half-space ($\acs{dz1} = 0$) reduces to:
\begin{equation}
    \acs{tilt} = 
    -\arctan\left(\dfrac{\acs{ddisp_z}}{\acs{dr1}+\acs{ddisp_r}}\right).\label{eq:tilt-from-flat-model}
\end{equation}
This discrete equation can be taken to the continuous limit by dividing each term in the numerator and denominator by the edge width \acs{dr1}:
\begin{equation}
\acs{tilt}
    = \lim_{\acs{dr1}\to0} 
    -\arctan\left(\dfrac{\acs{ddisp_z}/\acs{dr1}}{\acs{dr1}/\acs{dr1}
    + \acs{ddisp_r}/\acs{dr1}}\right) = 
    -\arctan\left(\dfrac{\acs{disp_z'}}{1+\acs{disp_r'}}\right),\label{eq:analytical-tilt}
\end{equation}
where $'$ denotes the derivative with respect to $r_1$. The \textcite{mogi_relations_1958} solution provides the following displacement components:
\begin{gather}
    \acs{disp_z} = kd{(d^2+r_1^2)}^{-1.5},\label{eq:uz_mogi}\\
    \acs{disp_r} = kr_1{(d^2+r_1^2)}^{-1.5},\label{eq:ur_mogi}\\
    k = {3R^3\Delta P}/{4G},\label{eq:k}
\end{gather}
where $d$ is the depth to the center of the reservoir, $R$ is the reservoir radius, $\Delta P$ is the overpressure, and $G$ is the elastic shear modulus of the surrounding rock. Notice that Equation~\eqref{eq:k} can be written to solve for the product of reservoir volume and overpressure, which represents the energy associated with the reservoir pressure change:
\begin{equation}
    \acs{epv} = \frac{4}{3}\pi R^3\Delta P=\frac{16\pi G}{9} \cdot k.\label{eq:epv}
\end{equation}

Differentiating Equations~\eqref{eq:uz_mogi} and~\eqref{eq:ur_mogi}, substituting into Equation~\eqref{eq:analytical-tilt}, and simplifying:
\begin{equation}
    \acs{tilt} = \arctan\left(\frac{3kdr_1}{{(d^2+r_1^2)}^{2.5}+k(d^2-2r_1^2)}\right).\label{eq:mogi-tilt}
\end{equation}
This key equation relates the independently calculated variables \acs{tilt} and $r_1$ to physical parameters associated with reservoir pressure change: depth $d$ and energy \acs{epv}.

% \subsubsection{Physical Considerations}

The physical conditions assumed in this analytical model (deep, point-like reservoir in an elastic half-space) are not necessarily met even within numerical models, much less the physical edifice of Olympus Mons. However, this solution serves two important roles in the subsequent analysis. 

First, when the assumptions \emph{are} upheld\footnote{to the greatest extent possible; a reservoir of finite width and depth can never completely eliminate edge effects from the free surface above} in the numerical model, the analytical solution confirms that the model is working correctly. I show a representative example confirming this in Figure~\ref{fig:mogi-test}.

\begin{figure}
    \includegraphics[width=\textwidth]{methods/mogi-test.pdf}%
    \caption[Analytical solution verification]{Verification that the analytical tilt solution derived from \textcite{mogi_relations_1958} matches the numerical result for a deep, small, spherical reservoir in a flat half-space. ``Mogi (calc)'' uses parameters $d$ and \acs{epv} identical (or calculated directly from) those in the numerical model; ``Mogi (fit)'' uses a non-linear least squares regression to fit the numerical model data to the parameterized tilt function. All three results are essentially identical; the estimated parameters are very close to the true parameters.}%
    \label{fig:mogi-test}
\end{figure}

More importantly, I show in Figure~\ref{fig:mogi-test-shallow-oblate} that even conditions which violate the analytical solution assumptions produce tilt functions of similar qualitative shapes, although the associated parameters are incorrect. A non-linear least squares regression can be applied to find the combination of depth and inflation energy that best explains any tilt-distance dataset. 

\begin{figure}
    \includegraphics[width=\textwidth]{methods/mogi-test-shallow-oblate.pdf}%
    \caption[Analytical model sensitivity to reservoir geometry]{Illustration of the error introduced by violating the analytical assumption of a deep, point-like reservoir in a flat half-space. Specifically, this numerically modeled reservoir is oblate and close to the surface relative to its size; it also lies within a model of the Olympus Mons edifice rather than a flat half-space. Despite these factors, an analytical solution can fit the shape of this numerical data well, albeit by overestimating the depth and inflation energy responsible.}%
    \label{fig:mogi-test-shallow-oblate}
\end{figure}

\section{Mapping Discordant Features}\label{sec:mapping}

In this section, I map the spatial distribution of topographic discordance at the summit of Olympus Mons and define candidate axisymmetric center points to ultimately explain this discordance.

\subsection{Preparation of Published Data}

The \acf{CTX}\footnote{aboard the \ac{MRO} spacecraft launched in 2005 by \acs{NASA}.} captures \qty{\sim30}{\km} swaths across the entire martian surface in visible $(\lambda=\qtyrange{500}{800}{\nm})$ greyscale at \qty{\sim6}{\m} spatial resolution. \textcite{Dickson2018AGB} blended these swaths to produce a raster mosaic product (hereafter, ``\ac{CTX} mosaic'') which I use to visually identify and map lava flows and flow channels.

The \acf{MOLA}\footnote{aboard the now-retired \ac{MGS} spacecraft launched in 1996 by \acs{NASA}.} returned topography data with horizontal resolution of \qtyproduct{300 x 1000}{\m} at the equator (better at high latitudes) and elevation uncertainty of \qty{\sim3}{\m}~\parencite{smith_mars_2001}. To improve spatial resolution, additional elevation data from the \ac{HRSC}\footnote{aboard the \ac{MEX} spacecraft launched in 2003 by the \ac{ESA}} was blended to product a \ac{DEM} with \qty{200}{\m} pixel resolution. Each pixel's vertical uncertainty is \qty{\sim1}{\m}, with an additional global uncertainty of \qty{\sim1.8}{\m} in the martian areoid (martian equivalent of Earth's geoid). In this project, the global areoid uncertainty is not a concern because only one region (the summit of Olympus Mons) is considered.

\subsection{Study Area Definition}

I load these two data sources in an equal-area sinusoidal Mars projection\footnote{I originally used an equal-area projection with a plan to perform crater-counting calculations, which are area-dependent. I discuss in Chapter~\ref{cha:discussion} why this projection may introduce minor reliability concerns for the analysis I end up performing in the current version of this project.} in ArcGIS Pro. I define the study area as a square (in this projection) \qtyproduct{200 x 200}{\km} around at the centroid of the outermost \qty{19}{\km} contour,\footnote{This is the highest integer \unit{km} which is roughly circular and completely encloses the caldera complex, implying that it largely records the conical shape of the shield edifice without influence from subsequent caldera collapse or reservoir inflation.} as seen in Figure~\ref{fig:summit}.

\begin{figure}
    \centering
    \includegraphics[width=\textwidth]{summit.pdf}
    \caption[Summit study area]{Study area at the summit of Olympus Mons (inset). Sinusoidal Martian Projection. Contours in \unit{km}. Square is \qtyproduct{200 x 200}{\km}, centered at the midpoint of the outermost \qty{19}{\km} contour.}\label{fig:summit}
\end{figure}

\subsection{Preliminary Observations}

Figure~\ref{fig:summit} shows important topographic patterns at the summit of Olympus Mons. More than \qty{50}{\km} from the center of the figure, topographic contours (\qtyrange{12}{19}{\km}) are fairly regular concentric rings. Closer to the caldera, this axisymmetry breaks down: the caldera complex itself consists of six intersecting collapse pits. On the southern flank, we see a prominent arcuate \qty{20}{\km} contour with the topographic summit (within the \qty{21}{\km} contour) over \qty{20}{\km} from the southern caldera rim. Thus, my first task is to determine the degree to which axisymmetric inflation can apply to an edifice which is not entirely axisymmetric. It is important to point out, however, that the \emph{asymmetry} inherent to the region is crucial for the discordant flow method to function, as I describe in subsequent sections. 

\subsection{Mapping Lava Features}

I use the \ac{CTX} mosaic to visually identify lava flows near the summit of Olympus Mons. Following \textcite{mouginis-mark_geologic_2021}, I map lobate flow outlines as polygons where possible. From these polygons, I derive centerline features using the \hlss{Polygon To Centerline} tool, as shown in Figure~\ref{fig:linear-features}. Where flow margins are not visible, I map channels directly as linear features. I include discontinuous regions where I infer partial collapse of lava tubes yielding skylight chains,\footnote{This assumption of underlying continuity follows, e.g., \textcite{bleacher_olympus_2007,carr_geologic_2010,peters_lava_2021}.} as shown in Figure~\ref{fig:linear-features}.

\subsection{Sampling Paleo-Azimuth from Mapped Features}

While I maintain a consistent ``sense'' in my channel mapping (pointing away from rather than toward the caldera center), the \hlss{Polygon To Centerline} tool does not. Therefore, I use the \hlss{Flip Line} tool to reverse the orientation of any centerline features pointing in their paleo-uphill rather than paleo-downhill direction. Then I use the \hlss{Calculate Geometry Attributes} tool to find the azimuthal orientation from the start to the end of each linear feature. This result defines \acf{az1} for each feature.

\subsection{Sampling Modern Attitude from Topography}

\newcommand{\samplinginterval}{\qty{3}{\km}}

Along each linear feature, I use the \hlss{Generate Points Along Line} tool with sampling interval \samplinginterval\ to create a series of point features where further attitude data collection and analysis will take place. The reason for this choice is that while \acf{az1} is relatively uncertain being derived solely from flow features, modern topography can be measured to much higher precision using the \ac{MOLA} \ac{DEM}. More importantly, the analysis described later in Section~\ref{sec:tilt-from-map} is extremely sensitive to position (different locations within the same feature will yield different results even if they have the identical attitude variables). Note that features with length $<\samplinginterval$ are not sampled at all, on the grounds that especially short features are less likely than long ones to accurately record the regional paleo-topographic downhill azimuth.

\begin{figure}
    \includegraphics[width=\textwidth]{linear-features.pdf}
    \includegraphics[width=\textwidth]{linear-features-mapped.pdf}
    \caption[Mapping linear features]{\textbf{Top:} Lobate flows and linear channel features identified from the \acs{CTX} basemap. \textbf{Bottom:} Point samples derived along the linear channel and lobate flow centerlines.}%
    \label{fig:linear-features}
\end{figure}

\newcommand{\neighborhood}{\qty{2}{\km}}

Finally, I collect three attitude variables for each sampled point. The first  of these is \ac{az1}, which each point inherits directly from its parent linear feature. Note that since \ac{az1} is defined but \acf{sl1} is unknown, a graphical representation of this family of possible surfaces will be a line corresponding to a family of poles rather than a single pole.

Then, I use the \hlss{Surface Parameters} tool on the \ac{MOLA} \ac{DEM} to compute average topographic \hlss{Slope} and \hlss{Aspect} (downhill azimuth) rasters across the entire study area. To avoid capturing local topographic anomalies, these values are averaged over a ``neighborhood'' with radius \neighborhood. I use the \hlss{Extract Multi Values to Points} tool to assign \ac{sl2} and \ac{az2} to each sample point based on the value of the corresponding raster value at that location. Unlike the paleo-attitude, the modern attitude is fully defined by a single pole in attitude space, as in Figure~\ref{fig:surface}.

\subsection{Axisymmetric Center Candidate Locations}\label{sec:candidates}

Ultimately, I seek to identify one or more suitable vertical axes to explain the observed discordance in the summit region under an axisymmetric framework. I present the full discussion of this method in Section~\ref{sec:synthesis}, but the first step is generating a set of candidate center points to evaluate. To do this, I use the \hlss{Generate Tesselation} and \hlss{Feature to Point} tools to generate an evenly spaced array of points in the caldera vicinity as shown in Figure~\ref{fig:candidates}. I choose the extent of this array to capture most of the ``interesting'' topography, namely, the caldera and southern summit bulge, to capture discordance resulting from pressure change centered in this region. Each of these 781 points is less than \qty{4}{\km} from its six neighbors to ensure spatial resolution similar to the sampling interval (\samplinginterval) for \acl{az1} and modern topographic ``neighborhood'' (\neighborhood) for modern attitude measurements, without creating unnecessary computational expense.

\begin{figure}
    \includegraphics[width=\textwidth]{candidates.pdf}%
    \caption{Axisymmetric center location candidates}%
    \label{fig:candidates}
\end{figure}
\section{Evaluating Center Candidates}\label{sec:evaluation}

I illustrate the flow of this iterative approach abstractly in Figure~\ref{fig:eval-model} and present the Python implementation in Appendix~\ref{app:code}. In the following sections, I describe the specific criteria used to evaluate different aspects of the summit's geometry and history.

\begin{figure}
    \begin{tikzpicture}[scale=.95]

  \usetikzlibrary{positioning}

\coordinate (ceval) at (0,0);
\coordinate (centersample) at (0,5);
\coordinate (samples) at (-3,10);
\coordinate (centers) at (3,10);

\node [rotate=270] at (7, 10) {Input (from GIS)};
\node [rotate=270] at (7, 5) {Intermediate list};
\node [rotate=270] at (7, 0) {Output (back to GIS)};

\draw[arrow,line width=1mm] (6, 10) -- (6, 0);

\node (samplestable) [draw, shape=rectangle, align=center] at (samples) {\begin{tabular}{c|c}
    sID & LAT, LON, \acs{az1}, \acs{az2}, \acs{sl2}\\
    \hline\\
    \hline\\
    \hline\\
    \hline\\
  \end{tabular}};

\node[above=0mm of samplestable] {\hltt{samples.csv}};

\node (centerstable) [draw, shape=rectangle, align=center] at (centers) {\begin{tabular}{c|c}
    cID & LAT, LON\\
    \hline\\
    \hline\\
    \hline\\
    \hline\\
    \hline\\
    \hline\\
  \end{tabular}};

\node[above=0mm of centerstable] {\hltt{centers.csv}};

\node (frontcentersample) [shape=rectangle, align=center, rounded corners=0.2cm] at (centersample) {\begin{tabular}{c|c|c}
  sID & LAT, LON, \acs{az1}, \acs{az2}, \acs{sl2} & \acs{dist}, \acs{bearing}, \acs{beta1}, \acs{beta2}, \acs{sl1}, \acs{tilt}\\
  \hline
  &\\
  &copy of & calculated\\
  &\hltt{samples.csv} & from one cID\\
  &\\
\end{tabular}};

\node[above=3mm of frontcentersample] {\hltt{centers\_calc}};

\foreach \x in {0.3,0.25,...,-0.05}
    \path (centersample) + (-\x, \x) node[draw, shape=rectangle, align=center, fill=white, rounded corners=0.2cm] {\begin{tabular}{c|c|c}
        sID & LAT, LON, \acs{az1}, \acs{az2}, \acs{sl2} & \acs{dist}, \acs{bearing}, \acs{beta1}, \acs{beta2}, \acs{sl1}, \acs{tilt}\\
        \hline
        &\\
        &copy of & calculated\\
        &\hltt{samples.csv} & from cID\\
        &\\
      \end{tabular}};

  \node (centersevaltable) [draw, shape=rectangle, align=center] at (ceval) {\begin{tabular}{c|c|c}
    cID & LAT, LON & SCORES\\
    \hline
    &\\
    &\\
    &copy of&each row calculated from\\
    &\hltt{centers.csv}&one item in \hltt{centers\_calc} \\
    &\\
    &\\
  \end{tabular}};

\node[above=0mm of centersevaltable] {\hltt{centers\_eval.csv}};

\end{tikzpicture}%
    \caption[Center evaluation workflow]{Schematic illustration of candidate center evaluation process. For each center (unique cID) in \hltt{centers.csv}, I make a copy of \hltt{samples.csv} and calculate variables for each sample (unique sID). This results in the ``intermediate list'' where each element corresponds to one cID. Then I evaluate each item in this list and write the resulting scores to one row of \hltt{centers\_eval.csv}.}%
    \label{fig:eval-model}
\end{figure}

Discordant features at the summit of Olympus Mons have a particular spatial distribution which is independent of any surface deformation or subsurface pressure change history constructed to explain them. However, even the most basic questions involved in constructing such an interpretation require extensive transformations to the dataset. For example, each possible inflation center location ``sees'' discordant features only in terms of distance-tilt pairs.

The previous sections describe a method for expressing discordant features in terms of tilt-distance datasets which are specific to each center candidate, under the axisymmetric pressure change condition which imposes a particular horizontal tilt axis, etc. To make progress with these datasets, it is necessary to determine which ones are most consistent with volcanologically plausible processes.

This evaluation process is broken into two steps, each of which is broken into two broad questions asked of each dataset:
\begin{enumerate}
    \item How well does the dataset capture the mapped discordant features?
    \item How well does the dataset capture a plausible modeled reservoir change event?
\end{enumerate}
Each of these questions is divided into more specific components in its respective section.

\subsection{Does it capture Mapped Discordance?}

In the tilt-distance dataset corresponding to each center, discordant sample points are first expressed in terms of distance from the center. This calculation, shown in Equation~\eqref{eq:dist}, works for any pair of points given their coordinates. Next, Section~\ref{sec:tilt-from-map} discusses the series of calculations involved in expressing the tilt necessary to explain each discordant feature via rotation about the tilt axis imposed by the relationship between sample location and center locations. One important result from this discussion is that not every configuration can yield a tilt estimate at all. So the first important characteristic of a tilt-distance dataset is: what fraction of the samples can be explained by the corresponding center. I call this the ``tiltable'' fraction.

Another consideration is: can I be confident that a particular calculation is accurate? One way to see this visually is in Figure [NEW FIG].

\subsection{Inflation Center}\label{sec:inflation-center}


Instead, I need to consider a larger population of tilt-distance data points together. This increases the likelihood of recognizing a tilt pattern consistent with reservoir pressure change, i.e., a shape similar to those shown in Figures~\ref{fig:grav-topo-test},~\ref{fig:mogi-test}, and~\ref{fig:mogi-test-shallow-oblate}. % However, this alone does not address the second issue, that erroneously calculated tilt values will often be present alongside any true tilt patterns consistent with axisymmetric tilt from the inflation center in question. Of course, identifying these erroneous calculations as such would require independent knowledge of the axis and degree of tilt responsible, which is exactly what I am trying to determine.

Finally, I perform a non-linear least-squares regression using the \hltt{curve\_fit} function from the \hltt{scipy.optimize} Python module \parencite{2020SciPy-NMeth} to fit Equation~\eqref{eq:mogi-tilt} (the analytical tilt solution) to the tilt-distance dataset. Just as a linear regression computes the best-fitting slope and intercept, this process computes the best-fitting depth and energy parameters. In both cases, the best fit minimizes the mean square error:
\begin{equation}
    \frac{1}{n}\sum_{i}^{n}{\left[\acs{tilt}_i-\hat{\acs{tilt}}_i\right]}^2\label{eq:mse}
\end{equation}
for $n$ observations of the form $(\acs{dist}, \acs{tilt})$ where $\hat{\acs{tilt}}$ is evaluated at $\acs{dist}$ using the best-fit parameters. The details of the curve-fitting algorithm (Levenberg-Marquardt) are beyond the scope of this thesis, but one important difference from a simpler linear regression model is that the curve does not necessarily converge (successfully estimate a set of best-fitting parameters) for particularly messy datasets. Whether convergence occurs (and even the final parameter estimates) depends on initial parameter estimates and maximum number of iterations allowed before ``giving up'' on convergence.

Initial parameter estimates for inflation energy $\acs{epv} = \qty{7e19}{\J}$ if the mean tilt in the dataset is positive and \qty{-7e19}{\J} if the mean is negative. The initial depth estimate is $d = \qty{20}{\km}$ in either case. 500 iterations are allowed before a dataset is deemed non-convergent, i.e., not explainable by an analytical tilt model in the form of Equation~\eqref{eq:mogi-tilt}.

After performing each regression, I first assess the mathematical goodness of fit for each center's fitted tilt solution. First, I determine what fraction of the samples are. Recall that the regression only proceeds using this subset, so the fraction of the dataset involved provides important context for the second criterion: what is the \ac{RMSE} of the regression, if one converges at all? This statistic is the square root of Equation~\eqref{eq:mse}:
\begin{equation}
    \sqrt{\frac{1}{n}\sum_{i}^{n}{\left[\acs{tilt}_i-\hat{\acs{tilt}}_i\right]}^2}.\label{eq:rmse}
\end{equation}
It has the same units as \acs{tilt} (degrees) and provides a relative comparison of goodness-of-fit for regression results; smaller values indicate better fit. Taken together, the tiltable fraction and \ac{RMSE} associated with each center reflect the mathematical goodness-of-fit for each center candidate.

I use these fitted values and crucially, their associated error terms, to determine the likelihood that surface tilt actually occurred in the direction radial to the given center candidate. This approach is supported by Figure~\ref{fig:mogi-test-shallow-oblate}, which illustrates that even tilt resulting from shallow, oblate reservoirs within a topographic edifice can be fit well to the analytical Mogi tilt function. At this stage, I pay closer attention to the parameter error estimates (similar to an $R^2$ value for a linear regression) to determine whether \emph{any} axisymmetric tilt is likely to have occurred from this center, rather than the estimates themselves---although I treat any unrealistic parameter estimates with caution regardless of their goodness of fit.

This sequence of calculations is much more computationally expensive than the previous one, and a single realistic reservoir pressure change is unlikely to explain the entire dataset anyway. Therefore, I select smaller sample populations for this criterion, paying particular attention to the highly discordant flows near the southern summit region shown in Figure~\ref{fig:uphill-flows}.

\subsection{Evaluating Reservoir Pressure Change Parameters}

After narrowing down the set of plausible inflation center locations (map view) in Section~\ref{sec:evaluation}, I use numerical model results to refine estimates for reservoir size, depth, shape, and pressure change.

\subsubsection{Analytical Model Limitations}

Section~\ref{sec:evaluation} narrows down the plausible horizontal position of the reservoir center, but it can only play a limited role in identifying volcanologically plausible configurations.

Figure~\ref{fig:mogi-test-shallow-oblate} illustrates a key preliminary finding: map data which are well explained by an analytical Mogi tilt solutions are also likely to fit one or numerical results which approximate the deep, point-like reservoir assumed by \textcite{mogi_relations_1958}. Disagreement between numerical and analytical increases with the degree of analytical assumption violations.

The magmatic system below the summit of Olympus Mons is likely shallow and oblate. 

\subsubsection{Numerical Model Approach}

I use the same iterative optimization method (minimizing \acs{RMSE}) described in Section~\ref{sec:inflation-center}. In the absence of an explicit tilt equation parameterized by reservoir size, depth, shape, and pressure change, I perform a brute-force search through an array of numerical solutions at each iteration.

For each dataset (corresponding to a center candidate of interest) this process begins with a very coarse set of numerical model parameters. I use the \hlss{Parametric Sweep} tool in COMSOL to define a list of values to test for each parameter; the software produces and exports a displacement solution for each one. After

\section{Summary}

The final product of this section is a transformation from mapped attitude data (Figure~\ref{fig:attitude-data}) at a particular location $(\acs{lat}, \acs{lon})$ to a tilt-distance data point corresponding to a single inflation center candidate $(\acs{latC}, \acs{lonC})$. Figure~\ref{fig:tilt-example} illustrates the final results. Repeating the tilt and distance calculations for each sample (within a discordant population of interest) to produce a full tilt-distance dataset corresponding to a single center candidate. Finally, repeating this dataset construction for each center allows the full array of candidates to be evaluated and compared.