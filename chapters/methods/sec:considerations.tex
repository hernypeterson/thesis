\section{Fundamental Considerations}\label{sec:considerations}

The ultimate goal of this thesis is to investigate the magma system below the summit of Olympus Mons. Practically speaking, this means explaining the peculiar geometry and spatial distribution of discordant flows in terms of subsurface magma reservoir pressure change. I approach this inquiry under the assumption that discordant lava features and subsurface pressurization each imply net change in surface attitude (orientation). In this section I lay the foundations for a quantitative treatment of surface attitude change.

\begin{figure}
    \begin{tikzpicture}[scale=1,tdplot_main_coords]
    % origin
    \coordinate (A1) at (8,8,0);
    \coordinate (B1) at (0,8,1);
    \coordinate (C1) at (0,0,1);
    \coordinate (D1) at (8,0,0);

    \coordinate (B0) at (0,8,0);
    \coordinate (C0) at (0,0,0);   

    \coordinate (flow1start) at ($(B1)!0.5!(C1)$);
    \coordinate (flow1stop) at ($(A1)!0.5!(D1)$);

    \coordinate (A2) at (8,8,4);
    \coordinate (B2) at (0,8,6);
    \coordinate (C2) at (0,0,6);
    \coordinate (D2) at (8,0,4);

    \coordinate (flow2start) at ($(B2)!0.5!(C2)$);
    \coordinate (flow2stop) at ($(A2)!0.5!(D2)$);

    \coordinate (A3) at (8,8,8);
    \coordinate (B3) at (0,8,8);
    \coordinate (C3) at (0,0,10);
    \coordinate (D3) at (8,0,10);

    \coordinate (flow3start) at ($(B3)!0.5!(C3)$);
    \coordinate (flow3stop) at ($(A3)!0.5!(D3)$);

    \draw (A1) -- (B0) -- (C0) -- (D1) -- (A1);
    \draw (A1) -- (B1) -- (C1) -- (D1);
    \draw (C0) -- (C1);
    \draw (B0) -- (B1);

    \fill[blue, opacity = 0.2] (A1) -- (B1) -- (C1) -- (D1);

    \foreach \x in {0,0.33333333333,...,1}{
        \draw[blue, ultra thick] ($(A1)!\x!(B1)$) -- ($(D1)!\x!(C1)$);
    };

    \draw[arrow, ultra thick] (flow1start) -- (flow1stop);
    
    \draw (A2) -- (B2) -- (C2) -- (D2) -- (A2);
    \draw[arrow, gray] (A1) -- (A2);
    \draw[arrow, gray] (B1) -- (B2);
    \draw[arrow, gray] (C1) -- (C2);
    \draw[arrow, gray] (D1) -- (D2);

    \draw[arrow, ultra thick] (flow2start) -- (flow2stop);

    \fill[gray, opacity = 0.2] (A2) -- (B2) -- (C2) -- (D2);

    \foreach \x in {0,0.33333333333,...,1}{
        \draw[gray, ultra thick] ($(A2)!\x!(B2)$) -- ($(D2)!\x!(C2)$);
    };

    \draw (A3) -- (B3) -- (C3) -- (D3) -- (A3);
    \draw[arrow, green!70!black] (A2) -- (A3);
    \draw[arrow, green!70!black] (B2) -- (B3);
    \draw[arrow, green!70!black] (C2) -- (C3);
    \draw[arrow, green!70!black] (D2) -- (D3);

    \draw[arrow, ultra thick] (flow2start) -- (flow2stop);

    \fill[green!70!black, opacity = 0.2] (A3) -- (B3) -- (C3) -- (D3);

    \foreach \x in {0,0.33333333333,...,1}{
        \draw[green!70!black, ultra thick] ($(A3)!\x!(D3)$) -- ($(B3)!\x!(C3)$);
    };

    \draw[arrow, ultra thick] (flow3start) -- (flow3stop);

\end{tikzpicture}%%
    \caption{Discordance Concept}%
    \label{fig:discordance-concept}
\end{figure}

% This would be a great place to add a figure that we've discussed for a while -- a conceptual depiction of how a reservoir change that introduces tilt variation can be measured using a difference in lava flow attitude. This would be a schematic diagram, not one that uses actual data. Part A would show a topographic surface upon which a lava flow is emplaced downhill. Part B would then show an event that changes the surface tilt such that the lava flow attitude is changed. If you wanted to make the figure more powerful, you could include instead B and C, where you illustrate a change that introduces only vertical tilt change (basically, one where the center of axisymmetry remains unadjusted) and one where both vertical and horizontal change has been induced (presumably because the details of the inflation are not axisymmetric or because the center of axisymmetry has shifted).

\subsection{Attitude Representation}\label{sec:attitude-representation}

The central role of surface attitude warrants a graphical representation suitable for both conceptual illustration and geometrically-grounded calculation. The most common such representation is the stereonet: planar surfaces are represented by their stereographically\footnote{from the top of the sphere onto a plane tangent to the bottom of the sphere.} projected intersections with the lower hemisphere of a unit sphere.

I instead use an orthographic\footnote{from directly above; orthogonal to the plane.} projection to send small circles\footnote{whose axis lies on the primitive (horizontal) great circle} to straight lines. These small circles are crucial for the derivation in Section~\ref{sec:tilt-from-map} because they represent the paths swept out by rotation about a horizontal axis. I also represent planes by their unique perpendicular ``poles,'' not their curvilinear intersections. Finally, I use an upper hemisphere projection so each pole has the same azimuth as the corresponding \emph{downhill} direction. I illustrate this surface attitude representation in Figure~\ref{fig:surface}.

\begin{figure}
    \begin{tikzpicture}[scale=4.4,tdplot_main_coords]

% origin
\coordinate (O) at (0,0,0);

% also defines (Pxy), (Pxz), (Pyz), etc.
\tdplotsetcoord{P}{\radius}{\ze}{\az}

% fill flat surface
\fill[color = gray!10!white] (0,0) circle (.7*\radius);

% define tilted surface
\tdplotsetrotatedcoords{\az}{\ze}{0}

% fill tilted surface
\fill[tdplot_rotated_coords, color = green!40!black, opacity=0.4] (0,0) circle (0.4*\radius);

% downhill line
\draw[arrow, tdplot_rotated_coords, green!40!black] (0,0) -- (0:0.4*\radius);

% horizontal surface (front right)
\fill[color = gray!10!white, opacity=0.6] (\az:.7*\radius) arc (\az:\az+90:.7*\radius) -- (0,0);

% perpendicular corners (origin)
\draw[tdplot_rotated_coords] (0.25,0,0) -- (0.25,0,0.25) -- (0,0,0.25) -- (0,-0.25,0.25) -- (0,-0.25,0);

% perpendicular corners (projection) FIXME
% \draw[] (Pxy) + (0.25,0,0) --+ (0.25,0,0.25) --+ (0,0,0.25) --+ (0,-0.25,0.25) --+ (0,-0.25,0);


% horizontal surface (front left)
\fill[color = gray!10!white, opacity=0.6] (\az-90:.7*\radius) arc (\az-90:\az:.7*\radius) -- (0,0);

% z axis
\draw[axis] (O) -- (0,0,0.5*\axislength) node[anchor=south]{$z$};

% line az surface
\draw[very thin, dashed,green!40!black] (P) -- (Pxy) -- (O);
\fill (Pxy) circle (.2mm);

\draw[ultra thick, green!40!black] (O) -- (P) node[anchor = south west] {};
\fill[green!40!black] (P) circle (.2mm);

% north axis
\draw[axis] (O) -- (0,0.4*\axislength,0) node[anchor=west]{\acs{north}};

% az angle label
\tdplotdrawarc{(O)}{0.4*\radius}{\az}{90}{coordinate, pin={[pin edge={black},-]-60:$\theta$}}{}

% az surface
\tdplotsetthetaplanecoords{\az}

% ze angle label
\tdplotdrawarc[tdplot_rotated_coords]{(O)}{0.4*\radius}{0}{\ze}{}{}

% sl angle label
\tdplotdrawarc[tdplot_rotated_coords]{(P)}{0.4*\radius}{180}{180+\ze}{coordinate, pin={[pin edge={black},-]0:$\varphi$}}{}

% ze angle label
\tdplotdrawarc[tdplot_rotated_coords]{(O)}{0.4*\radius}{90}{90+\ze}{coordinate, pin={[pin edge={black},-]180:$\varphi$}}{}


\fill[black] (O) circle (0.2pt);

\end{tikzpicture}%
\hspace{5mm}%
\begin{tikzpicture}[scale=1]

    \coordinate (orig) at (0,0);
    \coordinate (s2) at (\az:2);

    \draw (orig) circle (\flatradius);
    \draw (\az:1.8) arc (\az:90:1.8);
    \path (65:2.1) node {$\theta$};

    \draw[arrow] (orig) -- (90:\flatradius) node[anchor = south] {\acs{north}};
    \fill (s2) circle (1mm);

    \draw[very thin] (s2) -- node[sloped, fill=white] {$\sin\varphi$} (orig);


\end{tikzpicture}%%
    \caption[Orthographic pole to plane]{Pole to plane in an upper hemisphere orthographic projection. \textbf{Left:} A sloped green surface compared to a flat grey surface. A single normal vector within the upper hemisphere is perpendicular to the green surface. The zenith angle of this vector is the same as the slope of the surface $(\varphi)$. The unit vector in this direction is the pole (point). \textbf{Right:} When this pole is projected vertically (orthographic) onto the flat surface, the resulting point retains the downhill azimuth direction $(\theta)$ at a distance $\sin\varphi$ from the origin.}%
    \label{fig:surface}%
\end{figure}

\subsection{Simplifying Assumptions}

While I observe discordant features via satellite-derived imagery and topography, I can only infer subsurface pressure change. To introduce a hypothetical reservoir and model surface attitude change resulting from its activity, I need to answer several questions. To name just a few: Where is the reservoir in map view? How deep within/below the Olympus Mons edifice? How big? What shape? How much pressure change has occurred? Inflation or deflation? Several simplifying assumptions are necessary to make these questions tractable.

\subsubsection{Physical Properties}

I model the Olympus Mons edifice material as a single homogeneous basalt. The low profile of Olympus Mons (like most martian volcanoes) is analogous to terrestrial basaltic shield volcano such as Hawai'i. In the absence of geochemical or geophysical data pertaining to the internal edifice structure, I assume uniform basaltic flow material. I use a linear elastic rheology which allows for computationally inexpensive modeling and analytical surface deformation solutions \parencite{mogi_relations_1958} without sacrificing accuracy for a wide range of volcanic processes \parencite{grosfils_elastic_2015}. Similarly, I treat the magma reservoir as an ellipsoidal body exerting a uniform pressure on the surrounding edifice for similar reasons. % repeats. you could note that this geometry, though idealized, is defensible on 
  % - field (e.g., commonality of smooth margins, geometries seem in plan view),
  % - mechanical (e.g., to consistently feed central eruptions, to emplace radial dike swarms, and to explain caldera ring fault geometries) and 
  % - thermal (irregular shapes and protuberances grade toward a heat-conserving ellipsoidal geometry over time) grounds.
Despite the simplification, this configuration could describe a wide range of plausible and relevant circumstances (Figure~\ref{fig:reservoir-configs}).

\begin{figure}
    \includegraphics[width=\textwidth]{methods/caldera-collapse.pdf}\\
    \vspace{2cm}
    \includegraphics[width=\textwidth]{methods/flank-inflation.pdf}
    \caption[Ellipsoidal reservoir configurations]{Reservoir configurations whose surface deformation signature is roughly consistent with a homogeneous ellipsoidal reservoir experiencing pressure change within an elastic edifice. Note that caldera collapse (top) could be associated with elastic response outside the collapse region itself, which could be captured in this model.} % make the elastic rim thing more clear
    \label{fig:reservoir-configs}
\end{figure}

\subsubsection{Axisymmetric Geometry}

The assumption of axisymmetry (rotational symmetry about a vertical axis) is widely incorporated in models of magmatic systems across spatial and temporal scales \parencite[c.f.,][]{redmond_numerical_2004,ogawa_four-stage_2021,mogi_relations_1958,mctigue_elastic_1987}. As shown in Figure~\ref{fig:axisymmetry}, this geometry decreases computational and analytical complexity by reducing a 3D edifice to a representative 2D cross-section. In this thesis, I use the assumption of axisymmetry to model surface deformation resulting from subsurface pressure change. Numerical and analytical models with this geometry each appear throughout the analysis. Additionally, I show in Section~\ref{sec:tilt-from-map} that the axisymmetry assumption solves the attitude data incompleteness problem described in Section~\ref{sec:discordance}.

\begin{figure}
    \begin{tikzpicture}[scale=1.5,tdplot_main_coords]

% origin
\coordinate (orig) at (0,0,0);
\coordinate (summit) at (0,0,2.5);

\coordinate (bottomcorner) at (0,3*\radius,0);
\coordinate (topcorner) at (0,3*\radius,1);
\coordinate (topcorneropp) at (0,-3*\radius,1);
\coordinate (foot) at (0,\radius,1);
\coordinate (footopp) at (0,-\radius,1);

\coordinate (arrow) at (0,1.5*\radius,.5);
\coordinate (arrowopp) at (0,-1.5*\radius,.5);

% also defines (Pxy), (Pxz), (Pyz), etc.
\tdplotsetcoord{P}{\radius}{\ze}{\az}

% draw bottom surface
\draw (0,0,0) circle (3*\radius);
\fill[opacity=0.05] (0,0,0) circle (3*\radius);

% back half line
\draw[red,arrow] (arrow) arc (90:450:1.5*\radius);
\draw (foot) arc (90:450:\radius);

% back half fill
\foreach \y in {90,...,269}{
    \fill[opacity=0.05] (\y:\radius) + (0,0,1) arc (\y:\y+1:\radius) -- (summit);
    \fill[opacity=0.05] (\y:3*\radius) -- (\y+1:3*\radius) --+ (0,0,1) arc (\y-360:\y-361:3*\radius);
};

\fill[opacity = 0.05] (topcorner) arc (90:270:3*\radius);

% cross section
\fill[white] (orig) -- (summit) -- (foot) -- (topcorner) -- (bottomcorner);
\draw (orig) -- (summit) -- (foot) -- (topcorner) -- (bottomcorner) -- (orig);

% axes
\draw[axis] (orig) -- (0,0,2*\axislength) node[anchor=south]{$z$};
\draw[axis] (orig) -- (0,3*\axislength,0) node[anchor=west]{$r$};

% front line
\draw[red,arrow] (arrowopp) arc (270:450:1.5*\radius);
\draw (footopp) arc (270:450:\radius);

% front fill
\foreach \z in {270,...,449}{
    \fill[opacity=0.05] (\z:\radius) + (0,0,1) arc (\z:\z+1:\radius) -- (summit);
    \fill[opacity=0.05] (\z:3*\radius) -- (\z+1:3*\radius) --+ (0,0,1) arc (\z-360:\z-361:3*\radius);
};

\fill[opacity = 0.05] (topcorneropp) arc (270:450:3*\radius);

% draw middle surfaces
\draw (0,0,1) circle (3*\radius);

\end{tikzpicture}%
    \caption[Axisymmetry]{Schematic illustration of an axisymmetric model. A 3D feature with rotational symmetry about a vertical axis is fully described by a single cross-section (white). Conceptually this surface can be swept around the axis of symmetry (red arrow) to reconstruct the 3D shape. Radial $(r)$ and vertical $(z)$ coordinates describe position within the model.}%
    \label{fig:axisymmetry}
\end{figure}