\section{Fundamental Considerations}\label{sec:considerations}

\subsection{Attitude Representation}

Discordant features imply net change in surface attitude (orientation in space). The central role of attitude in this thesis warrants a graphical representation appropriate for both conceptual illustration and geometric calculation. The most common such representation is the stereonet, wherein features are represented by their stereographically\footnote{from one point on the sphere onto a plane tangent to the opposite point on the sphere.} projected intersections with the lower hemisphere of a unit sphere.

I use an orthographic\footnote{from directly above; orthogonal to the plane.} projection of the same sphere so small circles whose axis lies on the primitive (horizontal) great circle project to straight lines. This property is important for the derivation in Section~\ref{sec:paleo-slope} and elsewhere. Planes are typically described by their curvilinear intersection to distinguish them from lines, whose intersections are points. This thesis deals only with planes, which I represent by their unique perpendicular ``poles.'' Finally, I use an upper hemisphere projection so each pole has the same azimuth as the \emph{downhill} direction it represents. I illustrate each of these considerations in Figure~\ref{fig:surface}.

% FIGURE HERE OF ORTHOGRAPHIC VS STEREO?

\begin{figure}
    \begin{tikzpicture}[scale=4.4,tdplot_main_coords]

% origin
\coordinate (O) at (0,0,0);

% also defines (Pxy), (Pxz), (Pyz), etc.
\tdplotsetcoord{P}{\radius}{\ze}{\az}

% fill flat surface
\fill[color = gray!10!white] (0,0) circle (0.5*\radius);

% define tilted surface
\tdplotsetrotatedcoords{\az}{\ze}{0}

% fill tilted surface
\fill[tdplot_rotated_coords, color = green!40!black, opacity=0.4] (0,0) circle (0.4*\radius);

% downhill line
\draw[arrow, tdplot_rotated_coords] (0,0) -- (0:0.4*\radius);

% horizontal surface (front right)
\fill[color = gray!10!white, opacity=0.6] (\az:0.5*\radius) arc (\az:\az+90:0.5*\radius) -- (0,0);

% perpendicular corners
\draw[tdplot_rotated_coords] (0.25,0,0) -- (0.25,0,0.25) -- (0,0,0.25) -- (0,-0.25,0.25) -- (0,-0.25,0);

% horizontal surface (front left)
\fill[color = gray!10!white, opacity=0.6] (\az-90:0.5*\radius) arc (\az-90:\az:0.5*\radius) -- (0,0);

% z axis
\draw[axis] (O) -- (0,0,0.5*\axislength) node[anchor=south]{$z$};

% line az surface
\draw[very thin, dashed,green!40!black] (P) -- (Pxy) -- (O);
\draw[arrow, green!40!black] (O) -- (P) node[anchor = south west] {};

% north axis
\draw[axis] (O) -- (0,0.4*\axislength,0) node[anchor=west]{\acs{north}};

% az angle label
\tdplotdrawarc{(O)}{0.4*\radius}{\az}{90}{coordinate, pin={[pin edge={black},-]-60:\acs{az2}}}{}

% az surface
\tdplotsetthetaplanecoords{\az}

% ze angle label
\tdplotdrawarc[tdplot_rotated_coords]{(O)}{0.4*\radius}{0}{\ze}{coordinate, pin={[pin edge={black},-]80:\acs{sl2}}}{}

% ze angle label
\tdplotdrawarc[tdplot_rotated_coords]{(O)}{0.4*\radius}{90}{90+\ze}{coordinate, pin={[pin edge={black},-]180:\acs{sl2}}}{}

\fill[black] (O) circle (0.2pt);

\end{tikzpicture}%
    \caption[Orthographic pole to plane]{Pole to plane in an upper hemisphere orthographic projection. \textbf{Left:} A sloped green surface compared to a flat grey surface. A single normal vector within the upper hemisphere is perpendicular to the green surface. The zenith angle of this vector is the same as the slope of the surface $(\varphi)$. The unit vector in this direction is the pole (point). \textbf{Right:} When this pole is projected vertically (orthographic) onto the flat surface, the resulting point retains the downhill azimuth direction $(\theta)$ at a distance $\sin\varphi$ from the origin.}%
    \label{fig:surface}%
\end{figure}

The most plausible explanation for surface attitude change at the summit of a volcano is activity in a subsurface magma reservoir. To induce mechanical response in the surrounding rock, a reservoir must experience an over- or under-pressure relative to its surroundings. The goal of this thesis is to provide a quantitative link between discordant features and specific parameters associated with subsurface reservoir activity. Where is the reservoir? How big is it? What shape? How much pressure change has occurred? Inflation or deflation?   

\subsection{Simplifying Assumptions}

To answer these questions, I introduce some simplifying assumptions.

\subsubsection{Physical Properties}

First, I treat the Olympus Mons edifice as a homogeneous linear elastic basaltic material. %plausibility
Similarly, I treat the magma reservoir as an ellipsoidal body exerting a uniform pressure on the surrounding edifice. This configuration could describe a wide range of plausible circumstances: a broad flat reservoir depressurization associated with caldera formation, a small vertical protrusion from a more complex system, etc. %plausibility

\subsubsection{Axisymmetric Geometry}

The assumption of axisymmetry (rotational symmetry about a vertical axis) is a useful simplification for volcanic and magmatic systems, which are often roughly circular in map view. As shown in Figure~\ref{fig:axisymmetry}, this geometry decreases computational and analytical complexity by reducing a 3D edifice to a 2D vertical cross-section. Similarly, it reduces a 2D surface to a single linear transect where position is completely described by distance from the center. More specific to this thesis, I show in Section~\ref{sec:tilt-from-map} how an axisymmetric geometry solves the attitude data incompleteness problem described in Section~\ref{sec:discordance}.

\begin{figure}
    \begin{tikzpicture}[scale=1.5,tdplot_main_coords]

% origin
\coordinate (orig) at (0,0,0);
\coordinate (summit) at (0,0,2.5);

\coordinate (bottomcorner) at (0,3*\radius,0);
\coordinate (topcorner) at (0,3*\radius,1);
\coordinate (topcorneropp) at (0,-3*\radius,1);
\coordinate (foot) at (0,\radius,1);
\coordinate (footopp) at (0,-\radius,1);

\coordinate (arrow) at (0,1.5*\radius,.5);
\coordinate (arrowopp) at (0,-1.5*\radius,.5);

% also defines (Pxy), (Pxz), (Pyz), etc.
\tdplotsetcoord{P}{\radius}{\ze}{\az}

% draw bottom surface
\draw (0,0,0) circle (3*\radius);
\fill[opacity=0.05] (0,0,0) circle (3*\radius);

% back half line
\draw[red,arrow] (arrow) arc (90:450:1.5*\radius);
\draw (foot) arc (90:450:\radius);

% back half fill
\foreach \y in {90,...,269}{
    \fill[opacity=0.05] (\y:\radius) + (0,0,1) arc (\y:\y+1:\radius) -- (summit);
    \fill[opacity=0.05] (\y:3*\radius) -- (\y+1:3*\radius) --+ (0,0,1) arc (\y-360:\y-361:3*\radius);
};

\fill[opacity = 0.05] (topcorner) arc (90:270:3*\radius);

% cross section
\fill[white] (orig) -- (summit) -- (foot) -- (topcorner) -- (bottomcorner);
\draw (orig) -- (summit) -- (foot) -- (topcorner) -- (bottomcorner) -- (orig);

% axes
\draw[axis] (orig) -- (0,0,2*\axislength) node[anchor=south]{$z$};
\draw[axis] (orig) -- (0,3*\axislength,0) node[anchor=west]{$r$};

% front line
\draw[red,arrow] (arrowopp) arc (270:450:1.5*\radius);
\draw (footopp) arc (270:450:\radius);

% front fill
\foreach \z in {270,...,449}{
    \fill[opacity=0.05] (\z:\radius) + (0,0,1) arc (\z:\z+1:\radius) -- (summit);
    \fill[opacity=0.05] (\z:3*\radius) -- (\z+1:3*\radius) --+ (0,0,1) arc (\z-360:\z-361:3*\radius);
};

\fill[opacity = 0.05] (topcorneropp) arc (270:450:3*\radius);

% draw middle surfaces
\draw (0,0,1) circle (3*\radius);

\end{tikzpicture}%
    \caption[Axisymmetry]{Schematic illustration of an axisymmetric model. A 3D feature with rotational symmetry about a vertical axis is fully described by a single cross-section (white). Conceptually this surface can be swept around the axis of symmetry (red arrow) to reconstruct the 3D shape. Radial $(r)$ and vertical $(z)$ coordinates describe position within the model.}%
    \label{fig:axisymmetry}
\end{figure}