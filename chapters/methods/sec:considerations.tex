\section{Fundamental Considerations}\label{sec:considerations}

The goal of this thesis is to explain discordant flows in terms of subsurface magma reservoir pressure change. I approach this inquiry under the assumption that discordant lava features and subsurface pressurization each imply net change in surface attitude (orientation). In this section I lay the foundations for a quantitative treatment of surface attitude change.

\subsection{Attitude Representation}\label{sec:attitude-representation}

The central role of surface attitude warrants a graphical representation suitable for both conceptual illustration and geometrically-grounded calculation. The most common such representation is the stereonet: planar surfaces are represented by their stereographically\footnote{from the top of the sphere onto a plane tangent to the bottom of the sphere.} projected intersections with the lower hemisphere of a unit sphere.

I instead use an orthographic\footnote{from directly above; orthogonal to the plane.} projection to send small circles\footnote{whose axis lies on the primitive (horizontal) great circle} to straight lines. These small circles are crucial for the derivation in Section~\ref{sec:paleo-slope} because they represent the paths swept out by rotation about a horizontal axis. I also represent planes by their unique perpendicular ``poles,'' not their curvilinear intersections. Finally, I use an upper hemisphere projection so each pole has the same azimuth as the corresponding \emph{downhill} direction. I illustrate this surface attitude representation in Figure~\ref{fig:surface}.

% FIGURE HERE OF ORTHOGRAPHIC VS STEREO?

\begin{figure}
    \begin{tikzpicture}[scale=4.4,tdplot_main_coords]

% origin
\coordinate (O) at (0,0,0);

% also defines (Pxy), (Pxz), (Pyz), etc.
\tdplotsetcoord{P}{\radius}{\ze}{\az}

% fill flat surface
\fill[color = gray!10!white] (0,0) circle (0.5*\radius);

% define tilted surface
\tdplotsetrotatedcoords{\az}{\ze}{0}

% fill tilted surface
\fill[tdplot_rotated_coords, color = green!40!black, opacity=0.4] (0,0) circle (0.4*\radius);

% downhill line
\draw[arrow, tdplot_rotated_coords] (0,0) -- (0:0.4*\radius);

% horizontal surface (front right)
\fill[color = gray!10!white, opacity=0.6] (\az:0.5*\radius) arc (\az:\az+90:0.5*\radius) -- (0,0);

% perpendicular corners
\draw[tdplot_rotated_coords] (0.25,0,0) -- (0.25,0,0.25) -- (0,0,0.25) -- (0,-0.25,0.25) -- (0,-0.25,0);

% horizontal surface (front left)
\fill[color = gray!10!white, opacity=0.6] (\az-90:0.5*\radius) arc (\az-90:\az:0.5*\radius) -- (0,0);

% z axis
\draw[axis] (O) -- (0,0,0.5*\axislength) node[anchor=south]{$z$};

% line az surface
\draw[very thin, dashed,green!40!black] (P) -- (Pxy) -- (O);
\draw[arrow, green!40!black] (O) -- (P) node[anchor = south west] {};

% north axis
\draw[axis] (O) -- (0,0.4*\axislength,0) node[anchor=west]{\acs{north}};

% az angle label
\tdplotdrawarc{(O)}{0.4*\radius}{\az}{90}{coordinate, pin={[pin edge={black},-]-60:\acs{az2}}}{}

% az surface
\tdplotsetthetaplanecoords{\az}

% ze angle label
\tdplotdrawarc[tdplot_rotated_coords]{(O)}{0.4*\radius}{0}{\ze}{coordinate, pin={[pin edge={black},-]80:\acs{sl2}}}{}

% ze angle label
\tdplotdrawarc[tdplot_rotated_coords]{(O)}{0.4*\radius}{90}{90+\ze}{coordinate, pin={[pin edge={black},-]180:\acs{sl2}}}{}

\fill[black] (O) circle (0.2pt);

\end{tikzpicture}%
    \caption[Orthographic pole to plane]{Pole to plane in an upper hemisphere orthographic projection. \textbf{Left:} A sloped green surface compared to a flat grey surface. A single normal vector within the upper hemisphere is perpendicular to the green surface. The zenith angle of this vector is the same as the slope of the surface $(\varphi)$. The unit vector in this direction is the pole (point). \textbf{Right:} When this pole is projected vertically (orthographic) onto the flat surface, the resulting point retains the downhill azimuth direction $(\theta)$ at a distance $\sin\varphi$ from the origin.}%
    \label{fig:surface}%
\end{figure}

\subsection{Simplifying Assumptions}

While I observe discordant features via satellite-derived imagery and topography, I can only infer subsurface pressure change. To introduce a hypothetical reservoir and model surface attitude change resulting from its activity, I need to answer several questions. To name just a few: Where is the reservoir in map view? How deep within/below the Olympus Mons edifice? How big? What shape? How much pressure change has occurred? Inflation or deflation? Several simplifying assumptions are necessary to make these questions tractable.

\subsubsection{Physical Properties}

I model the Olympus Mons edifice material as a single homogeneous basalt. The low profile of Olympus Mons (like most martian volcanoes) is analogous to terrestrial basaltic shield such as Hawai'i. %plausibility: compositional -- no evidence for extensive differentiation or major I treat the edifice rock material as an elastic material only. An elastic model can capture a major component of the rheologic property of rock, with one notable exception being time-dependent behavior such as viscous relaxation. This is an especially important limitation to keep in mind for Olympus Mons which has been under construction for billions of years; even the most recent episodes of intrusive and eruptive activity are only constrained to the order of a few hundred million years.
I use a linear elastic rheology which allows for computationally inexpensive modeling and analytical surface deformation solutions \parencite{mogi_relations_1958} without sacrificing accuracy for a wide range of volcanic processes \parencite{grosfils_elastic_2015}. Similarly, I treat the magma reservoir as an ellipsoidal body exerting a uniform pressure on the surrounding edifice for similar reasons. Despite the simplification, this configuration could describe a wide range of plausible and relevant circumstances (Figure~\ref{fig:reservoir-configs}). %plausibility

\begin{figure}
    \includegraphics[width=\textwidth]{caldera-collapse.pdf}\\
    \vspace{2cm}
    \includegraphics[width=\textwidth]{flank-inflation.pdf}
    \caption{Reservoir configurations whose surface deformation signature is roughly consistent with a homogeneous ellipsoidal reservoir.}
    \label{fig:reservoir-configs}
\end{figure}

\subsubsection{Axisymmetric Geometry}

% plausibility.

The assumption of axisymmetry (rotational symmetry about a vertical axis) is widely used to model volcanic and magmatic systems at a range of spatial and temporal scales \parencite[c.f.,][]{redmond_numerical_2004,ogawa_four-stage_2021,mogi_relations_1958,mctigue_elastic_1987}. As shown in Figure~\ref{fig:axisymmetry}, this geometry decreases computational and analytical complexity by reducing a 3D edifice to a 2D vertical cross-section. Additionally, I show in Section~\ref{sec:tilt-from-map} that the axisymmetry assumption solves the attitude data incompleteness problem described in Section~\ref{sec:discordance}.

\begin{figure}
    \begin{tikzpicture}[scale=1.5,tdplot_main_coords]

% origin
\coordinate (orig) at (0,0,0);
\coordinate (summit) at (0,0,2.5);

\coordinate (bottomcorner) at (0,3*\radius,0);
\coordinate (topcorner) at (0,3*\radius,1);
\coordinate (topcorneropp) at (0,-3*\radius,1);
\coordinate (foot) at (0,\radius,1);
\coordinate (footopp) at (0,-\radius,1);

\coordinate (arrow) at (0,1.5*\radius,.5);
\coordinate (arrowopp) at (0,-1.5*\radius,.5);

% also defines (Pxy), (Pxz), (Pyz), etc.
\tdplotsetcoord{P}{\radius}{\ze}{\az}

% draw bottom surface
\draw (0,0,0) circle (3*\radius);
\fill[opacity=0.05] (0,0,0) circle (3*\radius);

% back half line
\draw[red,arrow] (arrow) arc (90:450:1.5*\radius);
\draw (foot) arc (90:450:\radius);

% back half fill
\foreach \y in {90,...,269}{
    \fill[opacity=0.05] (\y:\radius) + (0,0,1) arc (\y:\y+1:\radius) -- (summit);
    \fill[opacity=0.05] (\y:3*\radius) -- (\y+1:3*\radius) --+ (0,0,1) arc (\y-360:\y-361:3*\radius);
};

\fill[opacity = 0.05] (topcorner) arc (90:270:3*\radius);

% cross section
\fill[white] (orig) -- (summit) -- (foot) -- (topcorner) -- (bottomcorner);
\draw (orig) -- (summit) -- (foot) -- (topcorner) -- (bottomcorner) -- (orig);

% axes
\draw[axis] (orig) -- (0,0,2*\axislength) node[anchor=south]{$z$};
\draw[axis] (orig) -- (0,3*\axislength,0) node[anchor=west]{$r$};

% front line
\draw[red,arrow] (arrowopp) arc (270:450:1.5*\radius);
\draw (footopp) arc (270:450:\radius);

% front fill
\foreach \z in {270,...,449}{
    \fill[opacity=0.05] (\z:\radius) + (0,0,1) arc (\z:\z+1:\radius) -- (summit);
    \fill[opacity=0.05] (\z:3*\radius) -- (\z+1:3*\radius) --+ (0,0,1) arc (\z-360:\z-361:3*\radius);
};

\fill[opacity = 0.05] (topcorneropp) arc (270:450:3*\radius);

% draw middle surfaces
\draw (0,0,1) circle (3*\radius);

\end{tikzpicture}%
    \caption[Axisymmetry]{Schematic illustration of an axisymmetric model. A 3D feature with rotational symmetry about a vertical axis is fully described by a single cross-section (white). Conceptually this surface can be swept around the axis of symmetry (red arrow) to reconstruct the 3D shape. Radial $(r)$ and vertical $(z)$ coordinates describe position within the model.}%
    \label{fig:axisymmetry}
\end{figure}