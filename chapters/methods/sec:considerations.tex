\section{Fundamental Considerations}\label{sec:considerations}

\subsection{Axisymmetry}

Physical volcanologists use a wide array of models to better understand volcanic processes in terms of their underlying mechanisms. Simplifications are always necessary to build a model; one such simplification which is especially convenient for analytical and numerical (as opposed to physical) models is the assumption of \emph{axisymmetry}, that is, rotational symmetry about a single axis (Figure~\ref{fig:axisymmetry}). This assumption is reasonable for many volcanic and magmatic systems which are roughly circular in plan view. Importantly, the assumption of axisymmetry significantly reduces the computational cost of numerical models and makes analytical solutions much more tractable.

\begin{figure}
    \begin{tikzpicture}[scale=1.5,tdplot_main_coords]

% origin
\coordinate (orig) at (0,0,0);
\coordinate (summit) at (0,0,2.5);

\coordinate (bottomcorner) at (0,3*\radius,0);
\coordinate (topcorner) at (0,3*\radius,1);
\coordinate (topcorneropp) at (0,-3*\radius,1);
\coordinate (foot) at (0,\radius,1);
\coordinate (footopp) at (0,-\radius,1);

\coordinate (arrow) at (0,1.5*\radius,.5);
\coordinate (arrowopp) at (0,-1.5*\radius,.5);

% also defines (Pxy), (Pxz), (Pyz), etc.
\tdplotsetcoord{P}{\radius}{\ze}{\az}

% draw bottom surface
\draw (0,0,0) circle (3*\radius);
\fill[opacity=0.05] (0,0,0) circle (3*\radius);

% back half line
\draw[red,arrow] (arrow) arc (90:450:1.5*\radius);
\draw (foot) arc (90:450:\radius);

% back half fill
\foreach \y in {90,...,269}{
    \fill[opacity=0.05] (\y:\radius) + (0,0,1) arc (\y:\y+1:\radius) -- (summit);
    \fill[opacity=0.05] (\y:3*\radius) -- (\y+1:3*\radius) --+ (0,0,1) arc (\y-360:\y-361:3*\radius);
};

\fill[opacity = 0.05] (topcorner) arc (90:270:3*\radius);

% cross section
\fill[white] (orig) -- (summit) -- (foot) -- (topcorner) -- (bottomcorner);
\draw (orig) -- (summit) -- (foot) -- (topcorner) -- (bottomcorner) -- (orig);

% axes
\draw[axis] (orig) -- (0,0,2*\axislength) node[anchor=south]{$z$};
\draw[axis] (orig) -- (0,3*\axislength,0) node[anchor=west]{$r$};

% front line
\draw[red,arrow] (arrowopp) arc (270:450:1.5*\radius);
\draw (footopp) arc (270:450:\radius);

% front fill
\foreach \z in {270,...,449}{
    \fill[opacity=0.05] (\z:\radius) + (0,0,1) arc (\z:\z+1:\radius) -- (summit);
    \fill[opacity=0.05] (\z:3*\radius) -- (\z+1:3*\radius) --+ (0,0,1) arc (\z-360:\z-361:3*\radius);
};

\fill[opacity = 0.05] (topcorneropp) arc (270:450:3*\radius);

% draw middle surfaces
\draw (0,0,1) circle (3*\radius);

\end{tikzpicture}%
    \caption[Axisymmetry]{Schematic illustration of an axisymmetric model. A 3D feature with rotational symmetry about a vertical axis is fully described by a single cross-section (white). Conceptually this surface can be swept around the axis of symmetry (red arrow) to reconstruct the 3D shape. Radial $(r)$ and vertical $(z)$ coordinates describe position within the model.}%
    \label{fig:axisymmetry}
\end{figure}    

In this thesis, I use this assumption to build and run a series of numerical models which represent the elastic response of the Olympus Mons edifice to pressure change in a hypothetical magma reservoir. Additionally, I use the same axisymmetry assumption to solve the map data incompleteness problem described in Section~\ref{sec:discordance}. Specifically, I show that constraining surface deformation under this assumption provides a unique and \emph{complete} estimate for the attitude of the paleo-surface upon which a discordant lava feature was emplaced.

\subsection{Attitude Representation}

As mentioned above, the data I collect, calculate, and analyze in this thesis refer to surface attitude: orientation within 3D space. It is important to use a suitable representation of these data both for conceptual illustration and associated geometric calculations. The most common such representation in geology is the lower-hemisphere stereographic projection, in which linear and planar features are represented by their intersections with a unit sphere.

In this thesis, I use an orthographic (rather than stereographic) projection of this same sphere. That means rather than projecting intersections from a point on the sphere, features are projected vertically (orthogonal to the map surface). The reason for this change is that small circles whose axis lies on the primitive (horizontal) circle project to straight lines, which is convenient for the derivation in Section~\ref{sec:paleo-slope} and elsewhere.

Additionally, planes are often described by their curvilinear intersection to help set them apart from linear features, whose intersections are points. However, I do not describe any linear attitudes in this thesis, and curvilinear intersections quickly become unwieldy for geometric calculations. Therefore, I define all planes by their poles, or the unique linear attitude normal to any plane.

Finally, I use an upper rather than lower hemisphere projection to ensure that the projected pole has the same azimuth angle as the \emph{downhill} direction of the surface. I illustrate each of these considerations in Figure~\ref{fig:surface}.

\begin{figure}
    \begin{tikzpicture}[scale=4.4,tdplot_main_coords]

% origin
\coordinate (O) at (0,0,0);

% also defines (Pxy), (Pxz), (Pyz), etc.
\tdplotsetcoord{P}{\radius}{\ze}{\az}

% fill flat surface
\fill[color = gray!10!white] (0,0) circle (0.5*\radius);

% define tilted surface
\tdplotsetrotatedcoords{\az}{\ze}{0}

% fill tilted surface
\fill[tdplot_rotated_coords, color = green!40!black, opacity=0.4] (0,0) circle (0.4*\radius);

% downhill line
\draw[arrow, tdplot_rotated_coords] (0,0) -- (0:0.4*\radius);

% horizontal surface (front right)
\fill[color = gray!10!white, opacity=0.6] (\az:0.5*\radius) arc (\az:\az+90:0.5*\radius) -- (0,0);

% perpendicular corners
\draw[tdplot_rotated_coords] (0.25,0,0) -- (0.25,0,0.25) -- (0,0,0.25) -- (0,-0.25,0.25) -- (0,-0.25,0);

% horizontal surface (front left)
\fill[color = gray!10!white, opacity=0.6] (\az-90:0.5*\radius) arc (\az-90:\az:0.5*\radius) -- (0,0);

% z axis
\draw[axis] (O) -- (0,0,0.5*\axislength) node[anchor=south]{$z$};

% line az surface
\draw[very thin, dashed,green!40!black] (P) -- (Pxy) -- (O);
\draw[arrow, green!40!black] (O) -- (P) node[anchor = south west] {};

% north axis
\draw[axis] (O) -- (0,0.4*\axislength,0) node[anchor=west]{\acs{north}};

% az angle label
\tdplotdrawarc{(O)}{0.4*\radius}{\az}{90}{coordinate, pin={[pin edge={black},-]-60:\acs{az2}}}{}

% az surface
\tdplotsetthetaplanecoords{\az}

% ze angle label
\tdplotdrawarc[tdplot_rotated_coords]{(O)}{0.4*\radius}{0}{\ze}{coordinate, pin={[pin edge={black},-]80:\acs{sl2}}}{}

% ze angle label
\tdplotdrawarc[tdplot_rotated_coords]{(O)}{0.4*\radius}{90}{90+\ze}{coordinate, pin={[pin edge={black},-]180:\acs{sl2}}}{}

\fill[black] (O) circle (0.2pt);

\end{tikzpicture}%
    \caption[Orthographic pole to plane]{Pole to plane in an upper hemisphere orthographic projection. \textbf{Left:} A sloped green surface compared to a flat grey surface. A single normal vector within the upper hemisphere is perpendicular to the green surface. The zenith angle of this vector is the same as the slope of the surface $(\varphi)$. The unit vector in this direction is the pole (point). \textbf{Right:} When this pole is projected vertically (orthographic) onto the flat surface, the resulting point retains the downhill azimuth direction $(\theta)$ at a distance $\sin\varphi$ from the origin.}%
    \label{fig:surface}%
\end{figure}