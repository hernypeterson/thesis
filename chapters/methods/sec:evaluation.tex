\section{Evaluating Center Candidates}\label{sec:evaluation}

I illustrate the flow of this iterative approach abstractly in Figure~\ref{fig:eval-model} and present the Python implementation in Appendix~\ref{app:code}. In the following sections, I describe the specific criteria used to evaluate different aspects of the summit's geometry and history.

\begin{figure}
    \begin{tikzpicture}[scale=.95]

  \usetikzlibrary{positioning}

\coordinate (ceval) at (0,0);
\coordinate (centersample) at (0,5);
\coordinate (samples) at (-3,10);
\coordinate (centers) at (3,10);

\node [rotate=270] at (7, 10) {Input (from GIS)};
\node [rotate=270] at (7, 5) {Intermediate list};
\node [rotate=270] at (7, 0) {Output (back to GIS)};

\draw[arrow,line width=1mm] (6, 10) -- (6, 0);

\node (samplestable) [draw, shape=rectangle, align=center] at (samples) {\begin{tabular}{c|c}
    sID & LAT, LON, \acs{az1}, \acs{az2}, \acs{sl2}\\
    \hline\\
    \hline\\
    \hline\\
    \hline\\
  \end{tabular}};

\node[above=0mm of samplestable] {\hltt{samples.csv}};

\node (centerstable) [draw, shape=rectangle, align=center] at (centers) {\begin{tabular}{c|c}
    cID & LAT, LON\\
    \hline\\
    \hline\\
    \hline\\
    \hline\\
    \hline\\
    \hline\\
  \end{tabular}};

\node[above=0mm of centerstable] {\hltt{centers.csv}};

\node (frontcentersample) [shape=rectangle, align=center, rounded corners=0.2cm] at (centersample) {\begin{tabular}{c|c|c}
  sID & LAT, LON, \acs{az1}, \acs{az2}, \acs{sl2} & \acs{dist}, \acs{bearing}, \acs{beta1}, \acs{beta2}, \acs{sl1}, \acs{tilt}\\
  \hline
  &\\
  &copy of & calculated\\
  &\hltt{samples.csv} & from one cID\\
  &\\
\end{tabular}};

\node[above=3mm of frontcentersample] {\hltt{centers\_calc}};

\foreach \x in {0.3,0.25,...,-0.05}
    \path (centersample) + (-\x, \x) node[draw, shape=rectangle, align=center, fill=white, rounded corners=0.2cm] {\begin{tabular}{c|c|c}
        sID & LAT, LON, \acs{az1}, \acs{az2}, \acs{sl2} & \acs{dist}, \acs{bearing}, \acs{beta1}, \acs{beta2}, \acs{sl1}, \acs{tilt}\\
        \hline
        &\\
        &copy of & calculated\\
        &\hltt{samples.csv} & from cID\\
        &\\
      \end{tabular}};

  \node (centersevaltable) [draw, shape=rectangle, align=center] at (ceval) {\begin{tabular}{c|c|c}
    cID & LAT, LON & SCORES\\
    \hline
    &\\
    &\\
    &copy of&each row calculated from\\
    &\hltt{centers.csv}&one item in \hltt{centers\_calc} \\
    &\\
    &\\
  \end{tabular}};

\node[above=0mm of centersevaltable] {\hltt{centers\_eval.csv}};

\end{tikzpicture}%
    \caption[Center evaluation workflow]{Schematic illustration of candidate center evaluation process. For each center (unique cID) in \hltt{centers.csv}, I make a copy of \hltt{samples.csv} and calculate variables for each sample (unique sID). This results in the ``intermediate list'' where each element corresponds to one cID. Then I evaluate each item in this list and write the resulting scores to one row of \hltt{centers\_eval.csv}.}%
    \label{fig:eval-model}
\end{figure}

Discordant features at the summit of Olympus Mons have a particular spatial distribution which is independent of any surface deformation or subsurface pressure change history constructed to explain them. However, even the most basic questions involved in constructing such an interpretation require extensive transformations to the dataset. For example, each possible inflation center location ``sees'' discordant features only in terms of distance-tilt pairs.

The previous sections describe a method for expressing discordant features in terms of tilt-distance datasets which are specific to each center candidate, under the axisymmetric pressure change condition which imposes a particular horizontal tilt axis, etc. To make progress with these datasets, it is necessary to determine which ones are most consistent with volcanologically plausible processes.

This evaluation process is broken into two steps, each of which is broken into two broad questions asked of each dataset:
\begin{enumerate}
    \item How well does the dataset capture the mapped discordant features?
    \item How well does the dataset capture a plausible modeled reservoir change event?
\end{enumerate}
Each of these questions is divided into more specific components in its respective section.

\subsection{Does it capture Mapped Discordance?}

In the tilt-distance dataset corresponding to each center, discordant sample points are first expressed in terms of distance from the center. This calculation, shown in Equation~\eqref{eq:dist}, works for any pair of points given their coordinates. Next, Section~\ref{sec:tilt-from-map} discusses the series of calculations involved in expressing the tilt necessary to explain each discordant feature via rotation about the tilt axis imposed by the relationship between sample location and center locations. One important result from this discussion is that not every configuration can yield a tilt estimate at all. So the first important characteristic of a tilt-distance dataset is: what fraction of the samples can be explained by the corresponding center. I call this the ``tiltable'' fraction.

Another consideration is: can I be confident that a particular calculation is accurate? One way to see this visually is in Figure [NEW FIG].

\subsection{Inflation Center}\label{sec:inflation-center}


Instead, I need to consider a larger population of tilt-distance data points together. This increases the likelihood of recognizing a tilt pattern consistent with reservoir pressure change, i.e., a shape similar to those shown in Figures~\ref{fig:grav-topo-test},~\ref{fig:mogi-test}, and~\ref{fig:mogi-test-shallow-oblate}. % However, this alone does not address the second issue, that erroneously calculated tilt values will often be present alongside any true tilt patterns consistent with axisymmetric tilt from the inflation center in question. Of course, identifying these erroneous calculations as such would require independent knowledge of the axis and degree of tilt responsible, which is exactly what I am trying to determine.

Finally, I perform a non-linear least-squares regression using the \hltt{curve\_fit} function from the \hltt{scipy.optimize} Python module \parencite{2020SciPy-NMeth} to fit Equation~\eqref{eq:mogi-tilt} (the analytical tilt solution) to the tilt-distance dataset. Just as a linear regression computes the best-fitting slope and intercept, this process computes the best-fitting depth and energy parameters. In both cases, the best fit minimizes the mean square error:
\begin{equation}
    \frac{1}{n}\sum_{i}^{n}{\left[\acs{tilt}_i-\hat{\acs{tilt}}_i\right]}^2\label{eq:mse}
\end{equation}
for $n$ observations of the form $(\acs{dist}, \acs{tilt})$ where $\hat{\acs{tilt}}$ is evaluated at $\acs{dist}$ using the best-fit parameters. The details of the curve-fitting algorithm (Levenberg-Marquardt) are beyond the scope of this thesis, but one important difference from a simpler linear regression model is that the curve does not necessarily converge (successfully estimate a set of best-fitting parameters) for particularly messy datasets. Whether convergence occurs (and even the final parameter estimates) depends on initial parameter estimates and maximum number of iterations allowed before ``giving up'' on convergence.

Initial parameter estimates for inflation energy $\acs{epv} = \qty{7e19}{\J}$ if the mean tilt in the dataset is positive and \qty{-7e19}{\J} if the mean is negative. The initial depth estimate is $d = \qty{20}{\km}$ in either case. 500 iterations are allowed before a dataset is deemed non-convergent, i.e., not explainable by an analytical tilt model in the form of Equation~\eqref{eq:mogi-tilt}.

After performing each regression, I first assess the mathematical goodness of fit for each center's fitted tilt solution. First, I determine what fraction of the samples are. Recall that the regression only proceeds using this subset, so the fraction of the dataset involved provides important context for the second criterion: what is the \ac{RMSE} of the regression, if one converges at all? This statistic is the square root of Equation~\eqref{eq:mse}:
\begin{equation}
    \sqrt{\frac{1}{n}\sum_{i}^{n}{\left[\acs{tilt}_i-\hat{\acs{tilt}}_i\right]}^2}.\label{eq:rmse}
\end{equation}
It has the same units as \acs{tilt} (degrees) and provides a relative comparison of goodness-of-fit for regression results; smaller values indicate better fit. Taken together, the tiltable fraction and \ac{RMSE} associated with each center reflect the mathematical goodness-of-fit for each center candidate.

I use these fitted values and crucially, their associated error terms, to determine the likelihood that surface tilt actually occurred in the direction radial to the given center candidate. This approach is supported by Figure~\ref{fig:mogi-test-shallow-oblate}, which illustrates that even tilt resulting from shallow, oblate reservoirs within a topographic edifice can be fit well to the analytical Mogi tilt function. At this stage, I pay closer attention to the parameter error estimates (similar to an $R^2$ value for a linear regression) to determine whether \emph{any} axisymmetric tilt is likely to have occurred from this center, rather than the estimates themselves---although I treat any unrealistic parameter estimates with caution regardless of their goodness of fit.

This sequence of calculations is much more computationally expensive than the previous one, and a single realistic reservoir pressure change is unlikely to explain the entire dataset anyway. Therefore, I select smaller sample populations for this criterion, paying particular attention to the highly discordant flows near the southern summit region shown in Figure~\ref{fig:uphill-flows}.

\subsection{Evaluating Reservoir Pressure Change Parameters}

After narrowing down the set of plausible inflation center locations (map view) in Section~\ref{sec:evaluation}, I use numerical model results to refine estimates for reservoir size, depth, shape, and pressure change.

\subsubsection{Analytical Model Limitations}

Section~\ref{sec:evaluation} narrows down the plausible horizontal position of the reservoir center, but it can only play a limited role in identifying volcanologically plausible configurations.

Figure~\ref{fig:mogi-test-shallow-oblate} illustrates a key preliminary finding: map data which are well explained by an analytical Mogi tilt solutions are also likely to fit one or numerical results which approximate the deep, point-like reservoir assumed by \textcite{mogi_relations_1958}. Disagreement between numerical and analytical increases with the degree of analytical assumption violations.

The magmatic system below the summit of Olympus Mons is likely shallow and oblate. 

\subsubsection{Numerical Model Approach}

I use the same iterative optimization method (minimizing \acs{RMSE}) described in Section~\ref{sec:inflation-center}. In the absence of an explicit tilt equation parameterized by reservoir size, depth, shape, and pressure change, I perform a brute-force search through an array of numerical solutions at each iteration.

For each dataset (corresponding to a center candidate of interest) this process begins with a very coarse set of numerical model parameters. I use the \hlss{Parametric Sweep} tool in COMSOL to define a list of values to test for each parameter; the software produces and exports a displacement solution for each one. After