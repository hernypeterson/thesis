\section{Mapping Discordant Features}\label{sec:mapping}

% Mapping Tilt by Distance
    % reducing mapped features to points
    % choosing a center: for any discordant feature, both distance and tilt depend on the location of a hypothetical inflation center point
        % distance obviously, looking at previous tilt = f(dist) functions we can see that 3 degrees of tilt for a given surface would have very different implications if that surface was 10 km or 100 km laterally from the inflation center.
        % more subtly, the amount of tilt calculated for a surface also depends on the location of the inflation center. That's because the axis of tilt has to be perpendicular to the direction away from the center to ensure that tilt occurs toward or away from the center.
        % but I can't tell where the inflation center is just by looking at the surface.
        % not only that, different centers might better explain different discordant feature populations
        % so therefore I make a large array of inflation center candidates.
    % calculating tilt for a single sample relative to a single inflation center

In this section, I first map the spatial distribution of topographic discordance at the summit of Olympus Mons and define candidate axisymmetric center points to ultimately explain this discordance. Next, I begin the process of translating discordance into tilt-distance datasets for comparison with solutions from Section~\ref{sec:modeling}. For each sampled location at the surface, both tilt and distance depend on the horizontal position of the axisymmetric inflation center. Since I don't know where this center is, I construct a large spatial array of candidate center points to test. For each pair of points (one center candidate and one discordant sample location) I calculate the tilt necessary to explain the discordance of the sample location about the horizontal axis imposed by the chosen center candidate. This results in a tilt-distance dataset for each center candidate.

\subsection{Mapping Lava Features}

I use the \ac{CTX} mosaic to visually identify lava flows near the summit of Olympus Mons. Following \textcite{mouginis-mark_geologic_2021}, I map lobate flow outlines as polygons where possible. From these polygons, I derive centerline features using the \hlss{Polygon To Centerline} tool, as shown in Figure~\ref{fig:linear-features}. Where flow margins are not visible, I map channels directly as linear features. I include discontinuous regions where I infer partial collapse of lava tubes yielding skylight chains,\footnote{This assumption of underlying continuity follows, e.g., \textcite{bleacher_olympus_2007,carr_geologic_2010,peters_lava_2021}.} as shown in Figure~\ref{fig:linear-features}.

\subsection{Sampling Paleo-Azimuth from Mapped Features}

While I maintain a consistent ``sense'' in my channel mapping (pointing away from rather than toward the caldera center), the \hlss{Polygon To Centerline} tool does not. Therefore, I use the \hlss{Flip Line} tool to reverse the orientation of any centerline features pointing in their paleo-uphill rather than paleo-downhill direction. Then I use the \hlss{Calculate Geometry Attributes} tool to find the azimuthal orientation from the start to the end of each linear feature. This result defines \acf{az1} for each feature.

\subsection{Sampling Modern Attitude from Topography}

\newcommand{\samplinginterval}{\qty{3}{\km}}

Along each linear feature, I use the \hlss{Generate Points Along Line} tool with sampling interval \samplinginterval\ to create a series of point features where further attitude data collection and analysis will take place. The reason for this choice is that while \acf{az1} is relatively uncertain being derived solely from flow features, modern topography can be measured to much higher precision using the \ac{MOLA} \ac{DEM}. More importantly, the analysis described later in Section~\ref{sec:tilt-from-map} is extremely sensitive to position (different locations within the same feature will yield different results even if they have the identical attitude variables). Note that features with length $<\samplinginterval$ are not sampled at all, on the grounds that especially short features are less likely than long ones to accurately record the regional paleo-topographic downhill azimuth.

\begin{figure}
    \includegraphics[width=\textwidth]{linear-features.pdf}
    \includegraphics[width=\textwidth]{linear-features-mapped.pdf}
    \caption[Mapping linear features]{\textbf{Top:} Lobate flows and linear channel features identified from the \acs{CTX} basemap. \textbf{Bottom:} Point samples derived along the linear channel and lobate flow centerlines.}%
    \label{fig:linear-features}
\end{figure}

\newcommand{\neighborhood}{\qty{2}{\km}}

Finally, I collect three attitude variables for each sampled point. The first  of these is \ac{az1}, which each point inherits directly from its parent linear feature. Note that since \ac{az1} is defined but \acf{sl1} is unknown, a graphical representation of this family of possible surfaces will be a line corresponding to a family of poles rather than a single pole.

Then, I use the \hlss{Surface Parameters} tool on the \ac{MOLA} \ac{DEM} to compute average topographic \hlss{Slope} and \hlss{Aspect} (downhill azimuth) rasters across the entire study area. To avoid capturing local topographic anomalies, these values are averaged over a ``neighborhood'' with radius \neighborhood. I use the \hlss{Extract Multi Values to Points} tool to assign \ac{sl2} and \ac{az2} to each sample point based on the value of the corresponding raster value at that location. Unlike the paleo-attitude, the modern attitude is fully defined by a single pole in attitude space, as in Figure~\ref{fig:surface}.

\subsection{Axisymmetric Center Candidate Locations}\label{sec:candidates}

Ultimately, I seek to identify one or more suitable vertical axes to explain the observed discordance in the summit region under an axisymmetric framework. I present the full discussion of this method in Section~\ref{sec:evaluation}, but the first step is generating a set of candidate center points to evaluate. To do this, I use the \hlss{Generate Tesselation} and \hlss{Feature to Point} tools to generate an evenly spaced array of points in the caldera vicinity as shown in Figure~\ref{fig:candidates}. I choose the extent of this array to capture most of the ``interesting'' topography, namely, the caldera and southern summit bulge, to capture discordance resulting from pressure change centered in this region. Each of these 781 points is less than \qty{4}{\km} from its six neighbors to ensure spatial resolution similar to the sampling interval (\samplinginterval) for \acl{az1} and modern topographic ``neighborhood'' (\neighborhood) for modern attitude measurements, without creating unnecessary computational expense.

\begin{figure}
    \includegraphics[width=\textwidth]{candidates.pdf}%
    \caption{Axisymmetric center location candidates}%
    \label{fig:candidates}
\end{figure}

\subsection{Deriving Tilt from Mapped Attitude Data}\label{sec:tilt-from-map}

\subsubsection{Spatial Description of Sampled Points}

Mars' rotation about an essentially fixed axis provides a convenient coordinate system of \ac{lat} and \ac{lon} for describing surface position. However, an \emph{axis}ymmetric model imposes a different coordinate system---one defined with respect to the axis of rotational symmetry.

Points mapped in GIS are automatically expressed in \ac{lat} and \ac{lon}; I need to convert this into polar coordinates $(\acs{dist},\acs{bearing})$ with the axisymmetric center as the origin. And because one goal of the project is to evaluate an array of candidate center points, I need an equation that takes any set of center coordinates $(\acs{latC}, \acs{lonC})$ and any set of sample point coordinates and returns a $(\acs{dist},\acs{bearing})$ pair.

To calculate \acf{dist}, I use the spherical law of cosines\footnote{Also known as the great circle distance formula, derived in Appendix~\ref{app:gcd}.} scaled by the martian equatorial radius since Olympus Mons is close to the equator:
\begin{equation}
    \acs{dist}=\arccos(\cos\acs{latC}\cos\acs{lat}\cos(\acs{lonC}-\acs{lon}) + \sin\acs{latC}\sin\acs{lat})\cdot\qty{3396.2}{\km}.
\end{equation}

It turns out that error in the calculated \acs{bearing} value is much more significant than error in \acs{dist}. That's because I use \acs{bearing} to define the axis of tilt and thus the value of tilt for a given center-sample pair, and this computation is very sensitive to minor variations.

Therefore, since all the calculations take place at the sampled point, I define \acs{bearing} as the geographic azimuth angle at that point directly \emph{away} from the center.\footnote{It is worth noting that this numerical bearing is different from the azimuth from the center to the sample, although they represent the same physical direction.} This is to ensure that the horizontal \acs{bearing}-axis points in the same direction as positive \acs{dist}. This equation \parencite{williams_aviation, veness_calculate} has the form:
\begin{equation}
    \acs{bearing} = \ang{180} + \arctan\left(\frac{\sin(\acs{lonC}-\acs{lon})\cos\acs{latC}} {\cos\acs{lat} \sin\acs{latC}-\sin\acs{lat}\cos\acs{latC}\cos(\acs{lonC}-\acs{lon})}\right).\label{eq:bearing}
\end{equation} 

\subsubsection{Paleo-Slope Equation}\label{sec:paleo-slope}

\begin{figure}
\begin{center}
    \begin{tikzpicture}[scale=5,tdplot_main_coords]
    % origin
    \coordinate (orig) at (0,0,0);
    
    % also defines (Pxy), (Pxz), (Pyz), etc.
    \tdplotsetcoord{P}{\radius}{\ze}{\az}
    \tdplotsetcoord{C}{-0.4*\axislength}{90}{\THETA}
    \tdplotsetcoord{P'}{\radius}{\zen}{\azi}
    \tdplotsetcoord{tiltintercept}{\deltalength}{90}{\THETA-90}
    \tdplotsetcoord{rintercept}{0.68}{90}{\THETA}
    
    % gray rectangle
    \fill[tdplot_main_coords, color = gray!30!white] (orig) -- (tiltintercept) -- (Pxy) -- (rintercept);
    \draw[tdplot_main_coords] (\THETA-90:\radius) arc (\THETA-90:\THETA:\radius);

    % z-axis
    \draw[arrow] (orig) -- (0,0,\axislength) node[anchor=south]{$z$};
    \draw[arrow] (orig) -- (\THETA:\axislength) node[anchor=north east]{$r$};
    \draw[arrow] (orig) -- node[sloped, fill=white, pos=.65]{Tilt Axis} (\THETA-90:\axislength);

    % az' surface
    \tdplotsetthetaplanecoords{\azi}

    \draw[tdplot_rotated_coords] (90:\radius) -- (0,0.76604444*\radius) -- (0,0);
    \tdplotdrawarc[tdplot_rotated_coords, line width = 1mm]{(orig)}{\radius}{0}{90}{anchor=east}{\acs{az1}}

    % \draw[very thin, dashed] (P') -- (P'xy) -- (orig);
    % \draw (orig) -- (P');

    % surface vertical from axis
    \tdplotsetthetaplanecoords{\THETA+90}

    % proj1 line
    \tdplotdrawarc[tdplot_rotated_coords]{(orig)}{\radius}{-90}{0}{}{}
    
    % r-z surface, fill, line
    \tdplotsetthetaplanecoords{\THETA}
    % \fill[tdplot_rotated_coords, color = red, opacity = 0.1] (\zenproj:\radius) arc (\zenproj:\zeproj:\radius) -- (0,0);     
    \tdplotdrawarc[arrow, tdplot_rotated_coords]{(orig)}{\radius}{0}{90}{}{}
    \tdplotdrawarc[arrow, tdplot_rotated_coords,red,line width=1mm]{(orig)}{\radius}{\zenproj}{\zeproj}{anchor=south west}{\textcolor{black}{\acs{tilt}}}

    % az2 surface
    \tdplotsetthetaplanecoords{\az}
    \draw[very thin, dashed] (P) -- (Pxy) -- (orig);
    % \draw (orig) -- (P);
    
    % az angle labels
    %\tdplotdrawarc{(orig)}{0.33}{\THETA}{\azi}{}{}%{fill=white, anchor=south east}{\acs{beta1}}
    %\tdplotdrawarc{(orig)}{0.37}{\THETA}{\az}{}{}%{anchor=north west}{\acs{beta2}}

    % translate THETA surface to small circle
    \tdplotsetthetaplanecoords{\THETA}
    \tdplotsetrotatedcoordsorigin{(tiltintercept)}
    \draw[tdplot_rotated_coords] (0:\smallcircleradius) arc (0:90:\smallcircleradius) -- (0,0.76604444*\smallcircleradius) -- (0,0) -- cycle;
    \tdplotdrawarc[arrow, tdplot_rotated_coords]{(tiltintercept)}{\smallcircleradius}{\zenproj}{\zeproj}{coordinate}{}
    
    % translate THETA surface to back to great circle
    \tdplotsetrotatedcoordsorigin{(orig)}

    % proj1 plane, line, fill
    \tdplotsetrotatedcoords{\THETA}{-90+\zenproj}{0}
    \tdplotdrawarc[tdplot_rotated_coords, blue, ultra thick]{(0,0,0)}{\radius}{\asindelta}{0}{}{}
    \fill[tdplot_rotated_coords, color = blue, opacity = 0.2] (\asindelta:\radius) arc (\asindelta:0:\radius) -- (0,0) -- (tiltintercept);

    % proj2 plane, fill, line,
    \tdplotsetrotatedcoords{\THETA}{-90+\zeproj}{0}
    \fill[tdplot_rotated_coords, color = green!70!black, opacity = 0.3] (\asindelta:\radius) arc (\asindelta:0:\radius) -- (0,0) -- (tiltintercept);
    \tdplotdrawarc[tdplot_rotated_coords, green!70!black, ultra thick]{(0,0,0)}{\radius}{\asindelta}{0}{}{}
    
    % (az2, sl2) point
    \fill (P) circle (0.2mm) node[anchor = north east] {$(\acs{az2},\acs{sl2})$};
\end{tikzpicture}%
\hspace{1cm}%
\newcommand{\dotradius}{.3pt}%
\begin{tikzpicture}[scale=8]%
    \coordinate (orig) at (0,0);

    \coordinate (tiltintercept) at (\deltalength,0);
    \coordinate (pole1) at (\deltalength,\poleoneflatx);
    \coordinate (pole2) at (\deltalength,\poletwoflatx);
    \coordinate (rendpoint) at (0,\poletwoflatx+0.2);
    \coordinate (rintercept2) at (0,\poletwoflatx);    
    \coordinate (rintercept1) at (0,\poleoneflatx);    

    \coordinate (tiltaxisendpoint) at (\deltalength,0);
    
    \draw[dashed] (rintercept2) -- node[fill=white] {$\delta$} (pole2);
    \draw[dashed] (rintercept1) -- (pole1);
    
    % az2, sl2 point
    \fill (pole2) circle (\dotradius);
    \draw[dashed] (orig) -- node[pos=0.55,sloped,fill=white]{$\sin\acs{sl2}$} (pole2);
    \draw (70.5:.25) arc (70.5:90:.25);
    \path (80:.32) node{\acs{beta2}};
    \draw[arrow] (orig) -- (rendpoint) node[anchor=south] {$r$};
    \draw[arrow] (orig) -- node[fill=white] {Tilt Axis}(tiltaxisendpoint);
    \draw (tiltaxisendpoint) -- (pole2);

    % colored great circle segments
    % first radius should be the value of \radius
    \begin{scope}
        \clip (tiltintercept) rectangle (rendpoint);
        \draw[blue, very thick] (\radius,0) arc(0:90:1.3 and 0.425);
        \draw[green!70!black, very thick] (\radius,0) arc(0:90:1.3 and 1.075);
    \end{scope}
    \draw[red, arrow, ultra thick] (0,.425) -- (0,1.075);
    
    % az1 line
    \path (68:.15cm) node[fill=white]{\acs{beta1}};
    \draw (45:.2) arc (45:90:.2);
    % \draw[arrow] (orig) -- (45:\stereoradius) node[anchor=south west] {\acs{az1}};
    \draw[ultra thick,-latex] (orig) -- node[fill=white,sloped,pos=.55] {$\sin\acs{sl1}$} (pole1);

    \fill (pole2) circle (\dotradius);
\end{tikzpicture}%%
    \caption[\Acl{tilt} from mapping]{\textbf{Left:} A quadrant of an upper hemisphere representing attitude space for a sampled point relative to a particular inflation center. The pole labeled $\left(\acs{az2},\acs{sl2}\right)$ represents the modern surface. The line of poles labeled \acs{az1} represents the family of possible paleo-surfaces with downhill direction given by the lava flow direction. The $r-$axis points in the azimuthal direction away from a modeled inflation center. The assumption of axisymmetry imposes a horizontal axis of tilt perpendicular to the $r-$axis. The red arrow labeled \acs{tilt} answers the question: how much tilt \emph{about this axis} bring some pole on line \acs{az1} to pole $\left(\acs{az2},\acs{sl2}\right)$? \textbf{Right:} Orthographic projection labeled with important angles and distances for the tilt calculation, building on Figure~\ref{fig:surface}. This region corresponds to horizontal grey rectangle on the left.}%
    \label{fig:tilt-from-map}%
\end{center}
\end{figure}
While the modern topographic surface attitude is fully characterized by \acs{az2} and \acs{sl2}, only the \acf{az1} can be inferred directly from mapped surface features. The axisymmetric model imposes a circumferential tilt axis, that is, modeled tilt is always directly toward or away from the inflation center. I denote the direction away from the inflation center by \acs{bearing}. Figure~\ref{fig:tilt-from-map} illustrates my method for calculating \acf{sl1} under this assumption. The horizontal segment labeled $\delta$ can be defined in two ways:
\begin{equation*}
    \delta = \sin\acs{beta1}\sin\acs{sl1};\qquad
    \delta = \sin\acs{beta2}\sin\acs{sl2},
\end{equation*}
where $\acs{beta1} = \acl{beta1}$ and $\acs{beta2} = \acl{beta2}$. Thus,
\begin{equation}
    \acs{sl1} = \arcsin\left(\frac{\sin\acs{beta2}\sin\acs{sl2}}{\sin\acs{beta1}}\right).\label{eq:sl1}
\end{equation}

Equation~\eqref{eq:sl1} can naturally handle some issues inherent to the geometry of the problem. First, for some combinations of surface characteristics and center point choice, it is impossible to tilt the once-downhill lava flow about the imposed axis and reach the observed surface attitude. In Figure~\ref{fig:tilt-from-map} a single quadrant contains both $\left(\acs{az2},\acs{sl2}\right)$ and \acs{az1}, but if these fall on opposing sides of the \acs{bearing} line (one on the left, one on the right) there will be no possible vertical translation between the two. Mathematically, this case will arise as a negative value for \acs{sl1}, which is unphysical and must therefore be removed from the subsequent analysis. When a large fraction of sampled points are subject to this error, it signals that the chosen axisymmetric inflation center is inaccurate.

Another problem occurs even in the single-handed (both left or both right) case when:
\begin{equation}
     |\sin\acs{beta2}\sin\acs{sl2}| > |\sin\acs{beta1}|,\label{eq:domain-error}
\end{equation}
because the resulting argument in Equation~\eqref{eq:sl1} is outside the domain of the arcsin function. Physically this case occurs when the lava flow is close to parallel to the radial direction for the modeled inflation center. Additionally, when $|\sin\acs{beta1}|$ is only slightly greater than $|\sin\acs{beta2}\sin\acs{sl2}|$, the calculated \acs{sl1} will be unrealistically large. Both of these cases reflect the difficulty of changing the downhill azimuth of a flow by tipping it roughly in the direction it is already pointing. The quantity of these undefined or unrealistic tilt values provide an additional test on the validity of each axisymmetric inflation center.

\subsubsection{Tilt Equation}

In Figure~\ref{fig:tilt-from-map}, the arrow which translates the \acs{az1} line onto $\left(\acs{az2},\acs{sl2}\right)$ is a segment of a small circle about the tilt axis. To determine the true angle of tilt I translate this small circle onto the corresponding great circle along the blue and green great circle segments, respectively. I include this derivation in Appendix~\ref{app:proj}. The difference between these two transformations is the tilt: 
\begin{equation}
    \acs{tilt} = \arctan(\tan\acs{sl2}\cos\acs{beta2}) - \arctan(\tan\acs{sl1}\cos\acs{beta1}).\label{eq:tilt-from-map}
\end{equation}
Unlike Equation~\eqref{eq:sl1}, this calculations will not raise any domain errors. The only regions of high numerical sensitivity are near the poles of the tilt axis, but this occurs only for near-vertical observed surfaces (to yield high values of $\delta$ in Figure~\ref{fig:tilt-from-map}) which is unphysical.

\subsubsection{Physical Considerations}

Equation~\eqref{fig:tilt-from-map} could in principle be applied directly to a given center-sample pair to calculate up to one unique tilt value. However, this calculation is extremely sensitive to error in collected data. Even if attitude data could be collected with perfect precision, there are several reasons why the resulting tilt would be inaccurate. For example, a lava flow may have been heading at a slight offset from the regional downhill azimuth due to local topography or physical processes occurring within the lava. Therefore, I introduce an uncertainly term for the \acs{az1} term, $\pm\ang{7}$. 

This uncertainty term provides some flexibility in the analysis, but at a cost. By definition, any flow feature whose modern underlying topography has a downhill azimuth within \ang{7} cannot be considered ``discordant'' and thus no non-zero tilt can be inferred here.

However, samples taken from discordant features can now be explained by a range of tilts, as shown in Figure~\ref{fig:az1-uncertainty}.

\begin{figure}
    \begin{tikzpicture}[scale=1]

    \coordinate (orig) at (0,0);
    \coordinate (s2) at (-60:1);

    \draw (orig) circle (\flatradius);

    \draw[arrow, ultra thick] (orig) -- (70:\flatradius);
    \fill (s2) circle (1mm);

    \draw[very thin] (s2) -- (orig);
    \draw[very thin] (s2) -- (70:\flatradius);


\end{tikzpicture}%
\hspace{1.5cm}%
\begin{tikzpicture}[scale=1]

    \coordinate (orig) at (0,0);
    \coordinate (s2) at (-60:1);

    \fill[opacity=0.1] (orig) -- (70+\uncert:\flatradius) arc (70+\uncert:70-\uncert:\flatradius) -- (orig);
    \fill (s2) circle (1mm);

    \draw (orig) circle (\flatradius);

    \draw[arrow, ultra thick] (orig) -- (70:\flatradius);
    \draw[arrow, blue] (orig) -- (70-\uncert:\flatradius);
    \draw[arrow, blue] (orig) -- (70+\uncert:\flatradius);
    \fill (s2) circle (1mm);

    \draw[very thin, blue] (s2) -- (orig);
    \draw[very thin, blue] (s2) -- (70-\uncert:\flatradius);
\end{tikzpicture}%%
    \caption[Paleo-azimuth uncertainty]{Paleo-azimuth uncertainty introduces tilt flexibility. \textbf{Left:} With \acs{az1} treated as an exact quantity, there are a narrow range of radial directions (colored green) which can translate the line to meet the point, and for many of those angles the tilt required is unrealistically large. \textbf{Right:} With a wider range of possible \acs{az1} values, more radial directions are possible to translate some point in the shaded region to the $(\acs{az2},\acs{sl2})$ point. Additionally, the tilt required to reach the boundary will always be less. However, points within the shaded region must have $\acs{tilt}=\ang{0}$. Therefore, the choice of uncertainty value is a tradeoff between incorporating more data points and explaining discordance using smaller, more realistic tilts.}%
    \label{fig:az1-uncertainty}
\end{figure}
