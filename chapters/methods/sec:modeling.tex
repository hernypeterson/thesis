\section{Modeling Reservoir Pressure Change}\label{sec:modeling}

In this section, I explore \emph{how} exactly the axisymmetric reservoir pressure change model introduced in Section~\ref{sec:considerations} would deform (change the attitude of) the surface. I pay particular attention to the effect of varying key parameters like reservoir depth, size, shape, and magnitude of pressure change. Surface deformation is expressed in terms of attitude change for direct comparison with discordant features.

My first approach to quantifying surface attitude change is a numerical, which computes of \emph{displacement} for a large array of discrete positions along an axisymmetric model surface. This approach has the benefit of incorporating nearly the full array of geometric considerations (Section~\ref{sec:considerations}), including reservoir size, aspect ratio, pressure change, and depth within/below the Olympus Mons edifice. The surface displacement solution derived from this model is easily converted to a tilt solution, i.e., angular change in surface attitude.

However, this numerical solution is really a family of solutions---each one corresponding to some combination of the aforementioned reservoir parameters. The primary goal is to find the best match between one model solution and one comparable map-derived dataset. The full implications of this problem are described in Section~\ref{sec:mapping}, but the short version is this: there are far too many map-datasets and model-derived solutions to manually search for matches.

To solve this problem, I introduce an analytical tilt solution to complement the numerical one, derived analogously from differential displacement. to complement the numerical one. This approach necessarily reduces the geometric complexity. It assumes, for example, a spherical reservoir under a flat surface rather than a more general ellipsoid under an Olympus Mons-shaped edifice. It reduces to a single equation with only two parameters: depth (to center) and inflation energy (\acs{epv}), the product of reservoir volume and pressure change. In other words, this solution cannot distinguish between reservoir $A$ and $B$ if reservoir $A$ is ten times bigger but the pressure change in reservoir $B$ is ten times bigger.

Despite these drawbacks, the equation form of this analytical solution can be fit via a least-squares regression to an arbitrary dataset (Section~\ref{sec:evaluation}). Just as a linear least-squares regression determines the slope and intercept of a linear equation which best fits an arbitrary dataset, this non-linear regression using the determines the depth and inflation energy of the analytical tilt equation which best fits an arbitrary dataset. This regression method is an efficient and mathematically robust method for evaluating a large array of map-derived tilt datasets.

% Modeling Tilt by Distance:
    % numerical: can capture mechanical response to a wide array of initial conditions (any depth, size, aspect ratio, pressure, surface effects), output is a large _displacement_ dataset
        % displacement to tilt
    % analytical: can capture only a narrow range of initial conditions (depth, inflation energy (pressure times volume), no free surface effects) BUT the equation is much more convenient than an array of data.
        % displacement to tilt 
    % results from this section: tilt as a f(distance) for any combination of depth/energy (analytical), plus a way to refine these estimates (numerical)

% \subsubsection{Axisymmetric Elastic Model}

% Both numerical and analytical solutions are axisymmetric and elastic.

% Two important simplifying assumptions are worth introducing immediately to provide context for numerical modeling. Having already introduced one of these, the axisymmetric assumption, the only point to emphasize here is that each aspect of a numerically constructed model must share the same axis of symmetry. In other words, the ellipsoidal magma reservoir must be centered directly underneath the center of the modeled edifice. Importantly, this assumption requires a single axis to describe each aspect of the model; namely, the topographic surface and the magma reservoir. Thus the model construction is similar to the plane shown in Figure~\ref{fig:axisymmetry}.

\subsection{Numerical Tilt-Distance Solution}\label{sec:numerical-tilt-solution}

% REORDER - HAS NOT BEEN MENTIONED YET

I use the numerical modeling software COMSOL Multiphysics 6.1 (COMSOL) to construct a numerical finite element representation of Olympus Mons. The axis of symmetry for this model is the vertical line through the center point discussed previously as the center of the \qty{19}{\km} contour.

The \ac{FEM} is widely used to model phenomena ranging from fluid flow, heat transfer, and mechanical stress in engineering and geology. This method involves discretizing a continuous medium by constructing a \emph{mesh:} a network of nodes connected to their neighbors by edges. Polygons enclosed by these edges are called elements. Software then solves the chosen equations to compute the variables of interest only for the nodes, whose values can then be interpolated within each element to determine the solution for any point within the continuum. Relevant to this inquiry, the solution to an elastic model is a set of \emph{displacement} vectors by which to translate each node such that all forces are balanced and mechanical equilibrium is reached.

From this point, an elevation transect due south from the study area center point to the base of the edifice \qty{\sim240}{\km} away represents the paleo-surface topography. This particular center point is derived from the summit of a topographic spline: an interpolation of a hypothetical paleo-topography based on the existing topography outside the \qty{19}{\km} contour (Figure~\ref{fig:paleo-topo}; compare with Figure~\ref{fig:summit}). This point is effectively identical to the center point of the outer \qty{19}{\km} contour, providing additional evidence for approximate axisymmetry outside this region. The topography of this outer region is assumed to be relatively unaffected by recent caldera and reservoir activity. % Why? 

\begin{figure}
    \includegraphics[width=\textwidth]{methods/paleo-topo.pdf}
    \caption[Spline-derived paleo-topography]{Paleo-topography estimated to have existed prior to caldera formation; interpolated from topography outside the \qty{19}{\km} contour. Note that both centers---one defined by the \qty{19}{\km} contour (labeled), and one interpolated from the surrounding topography (inner red contour)---are close to one another but south of the caldera complex center.}%
    \label{fig:paleo-topo}
\end{figure}

\begin{figure}
    \includegraphics[width=\textwidth]{methods/model-section.pdf}\\
    \includegraphics[width=\textwidth]{methods/model-section-zoom.pdf}%
    \caption[Axisymmetric numerical model section]{Axisymmetric numerical model cross-section (i.e., white surface in Figure~\ref{fig:axisymmetry}) of Olympus Mons edifice with example ellipsoidal reservoir and surrounding lithosphere. \textbf{Top:} Full model section. \textbf{Bottom:} Top left corner of top image enlarged to show ellipsoidal reservoir.}%
    \label{fig:model-section}
\end{figure}

I use the \hlss{Parametric Sweep} tool to perform a sequence of analyses which are identical except for specifically controlled parameters: depth to center $d$, reservoir radius $R$ (in plan view), aspect ratio (height divided by width) and a \acf{mult}:
\begin{equation}
    \acs{dP}=\acs{mult}\times\acs{rhor}\times\acs{g}\times d,
\end{equation}
where \acs{dP} is the simulated over- (positive) or under-pressure in the reservoir. For each numerical model, I calculate surface displacement in the radial and vertical direction as a function of distance from the inflation center.

COMSOL produces displacement data corresponding to individual model nodes. Figure~\ref{fig:tilt-from-model} shows how to derive tilt from the surface edge of an element before (initial) and after (displaced) modeled pressure change. The corresponding tilt equation is:
\begin{equation}
    \acs{tilt} = \arctan\left({\acs{dz1}}/{\acs{dr1}}\right) - \arctan\left(\dfrac{\acs{dz1}+\acs{ddisp_z}}{\acs{dr1}+\acs{ddisp_r}}\right).\label{eq:tilt-from-model}
\end{equation}

I define the distance value associated with this calculated tilt as the midpoint of the displaced edge. In practice, the horizontal scale for this edge (both before and after displacement) is negligible compared to the scale of the edifice, so another choice of distance within the edge (e.g., the midpoint of the initial surface) would effectively produce identical results.

\begin{figure}
    \newcommand{\uar}{1}
\newcommand{\uaz}{5}
\newcommand{\ubr}{3.5}
\newcommand{\ubz}{1.5}
\newcommand{\dz}{1.5}
\newcommand{\dr}{6}

\begin{tikzpicture}[scale=1]
    \coordinate (orig) at (0,0);
    \coordinate (A1) at (0,\dz);
    \coordinate (B1) at (\dr,0);
    
    \path (A1) + (\uar, \uaz) coordinate (A2);
    \path (B1) + (\ubr, \ubz) coordinate (B2);
    \path (A1) + (\uar - \ubr, 0) coordinate (ZA1);
    \path (orig) + (\uar - \ubr, 0) coordinate (ZB1);

    \path (A1) + (\uar - \ubr, \uaz - \ubz) coordinate (A2_trans);

    \draw[-latex] (A2_trans) --+ (0,1) node[anchor=south] {$z$};
    \draw[-latex] (B1) --+ (2.5,0) node[anchor=west] {$r$};

    % initial element shape
    \draw[] (orig) -- node[fill=white] {\acs{dr1}} (B1);
    \draw[] (orig) -- node[fill=white] {$-\acs{dz1}$} (A1);

    % % main displacement vectors
    \draw[arrow,gray] (A1) -- node[sloped, fill=white] {\acs{disp_a}} (A2);
    \draw[arrow,gray] (B1) -- node[sloped, fill=white] {\acs{disp_b}} (B2);
    
    \draw[arrow,gray,dashed] (A2_trans) -- node[sloped, fill=white] {$\acs{disp_b}$} (A2);
    \draw (A2_trans) -- node[fill=white] {$-\acs{ddisp_z}$} (ZA1) -- node[fill=white] {$-\acs{dz1}$} (ZB1) -- node[fill=white] {\acs{ddisp_r}} (orig);
    
    % tilt
    \draw[red,arrow,line width=1mm] (B1) + (166:5.3) arc (166:149.5:5.3);
    \path (B1) + (158:5.7) node {\acs{tilt}};

    % axsl1
    \draw[blue, arrow] (B1) + (180:4.8) arc (180:166:4.8);
    \path (B1) + (173.3:4.5) node {\acs{axsl1}};

    % axsl2
    \draw[green!70!black, arrow] (B1) + (180:5) arc (180:149.5:5);
    \path (B1) + (158:4.7) node {\acs{axsl2}};

    % element surface lines
    \draw[dashed, ultra thick, green!70!black] (A2_trans) -- (B1);
    \draw[ultra thick, blue] (A1) -- node[sloped,fill=white,pos=.55] {\textcolor{black}{Initial Surface}} (B1);
    \draw[ultra thick, green!70!black] (A2) -- node[sloped,fill=white] {\textcolor{black}{Displaced Surface}} (B2);
    
    % horizontal component of displacement
    \draw (ZA1) -- (A1);

    % A_1 label
    \fill (A1) node[fill=white, anchor = east] {$A_1$} circle (2pt);

    % ddisp vector
    \draw[arrow, gray] (A2_trans) -- node[sloped,fill=white] {$\acs{ddisp}$} (A1);

    % points
    \fill (B1) circle (2pt) node[anchor = north] {$B_1$};
    \fill (A2) circle (2pt) node[anchor = south] {$A_2$};
    \fill (B2) circle (2pt) node[anchor = west] {$B_2$};
    \fill (A2_trans) circle (2pt) node[anchor = east] {$A_2'$};

    % A_1 point
    \fill (A1) circle (2pt);
    
\end{tikzpicture}%
    \caption[Tilt from numerical modeling]{Cross-sectional view of a surface edge in the axisymmetric numerical model. As in Figure~\ref{fig:tilt-from-map}, the $r-$axis points away from the inflation center. During the model, the two nodes $A_1$ and $B_1$ of a surface element are displaced along \acs{disp_a} and \acs{disp_b} to reach a final position at $A_2$ and $B_2$, respectively. I illustrate the difference $\acs{disp_b} - \acs{disp_a} = \acs{ddisp}$ in terms of its components \acs{ddisp_r} and \acs{ddisp_z}. Notice that $z-$component terms are negative to ensure that tilt away from the inflation center yields a positive tilt \acs{tilt}, as shown in red. To calculate \acs{tilt}, I determine the slopes of segments $\overline{A_1B_1}$ and $\overline{A_2'B_1}$ using the labeled horizontal and vertical segments, convert these slopes to angles, and take their difference.}%
    \label{fig:tilt-from-model}%
\end{figure}

% \subsubsection{Physical Considerations}

In Section~\ref{sec:modeling}, I explain that the axisymmetric numerical model edifice is constructed from a particular elevation transect measured from the summit of Olympus Mons. Relative to this point, surrounding topography is roughly axisymmetric, especially more than \qty{\sim50}{\km} away past the previously discussed \qty{19}{\km} contour.

However, I do not assume that the subsequent reservoir pressure change was centered at this same location. Instead, my goal is to  determine this inflation center location on the basis of discordant flow data, wherever they may point.

Importantly, placing a reservoir inflation center anywhere off the axis of symmetry in the numerical model introduces some inaccuracy. To make progress I need to determine whether this error is negligible; if it is not, I need to be aware of its magnitude as I interpret results.

Of course, an axisymmetric model by definition does not permit any non-axisymmetric elements to be introduced. Instead, I construct a flat (no edifice, horizontal surface) variant of the model as an end-member case for the topographic variation introduced by shifting the reservoir within the edifice.\footnote{This model does not directly address the issue of introducing uphill slope away from the inflation center, but the magnitude of any error introduced should be similar.}

Additionally, \textcite{grosfils_magma_2007} showed that incorporating gravitational loading is unnecessary for modeling surface displacement. To confirm this, I run a test model under three conditions:
\begin{enumerate}
    \item no gravitational loading with magma reservoir overpressure \label{g0p1}
    \item gravitational loading (lithostatic pre-stress) with no reservoir overpressure\label{g1p0}
    \item gravitational loading with reservoir overpressure \label{g1p1}
\end{enumerate}
If gravitational loading is insignificant, displacement in case~\ref{g0p1} should match the component remaining in case~\ref{g1p1} after the gravitational component from case~\ref{g1p0} is subtracted out. The reason I need to subtract out this component is that the modeled edifice is not perfectly flat. Therefore, vertical loading is not initially in equilibrium under the reservoir and some ``slumping'' occurs to accommodate this imbalance. Preliminary test results are shown in Figure~\ref{fig:grav-topo-test}.

% TODO: fix units in these plots

\begin{figure}
    \includegraphics[width=\textwidth]{methods/grav-topo-test.pdf}
    \includegraphics[width=\textwidth]{methods/grav-topo-test-zoom.pdf}%
    \caption[Numerical model sensitivity to topography and gravity]{Comparison between topographic and flat model with and without gravitational loading. Patterns are consistent across a range of model parameters tested; a single representative case is plotted above, with the peak shown in more detail below. Gravitational loading makes essentially no difference, and the flat model (topo = False) underestimates tilt by at most a few percent.}%
    \label{fig:grav-topo-test}%
\end{figure}

As expected for the flat model, gravitational loading has no effect on surface displacement and thus the subsequently calculated tilt is identical. The topographically accurate model shows the same pattern with respect to gravitational loading. There is a small but noticeable difference in tilt between the flat and topographic models, which ranges between ($0\%-10\%$) for the parameters I examined. This result places a reassuring upper bound on the magnitude of error that could be introduced by varying the horizontal location of a modeled inflation center within the summit region---likely much less than error introduced elsewhere.

\subsection{Analytical Tilt-Distance Solution}\label{sec:analytical-tilt-solution}

The numerical solutions of Section~\ref{sec:numerical-tilt-solution} each predict a particular signature of surface attitude change resulting from a particular set of subsurface reservoir conditions. More convenient for this thesis would be to invert this process---estimating subsurface conditions from surface attitude change derived from observations at the surface. A simplified analytical solution helps to efficiently narrow down a large empirical tilt dataset (introduced in Section~\ref{sec:mapping}); subsequent refinements and interpretations can be made with the help of numerical solutions.

I derive this analytical tilt solution from the widely cited \emph{displacement} solution of \textcite{mogi_relations_1958}. This so-called Mogi model assumes a deep spherical reservoir within a flat (``topo = False'') elastic half-space. Equation~\eqref{eq:tilt-from-model} in such a half-space ($\acs{dz1} = 0$) reduces to:
\begin{equation}
    \acs{tilt} = 
    -\arctan\left(\dfrac{\acs{ddisp_z}}{\acs{dr1}+\acs{ddisp_r}}\right).\label{eq:tilt-from-flat-model}
\end{equation}
This discrete equation can be taken to the continuous limit by dividing each term in the numerator and denominator by the edge width \acs{dr1}:
\begin{equation}
\acs{tilt}
    = \lim_{\acs{dr1}\to0} 
    -\arctan\left(\dfrac{\acs{ddisp_z}/\acs{dr1}}{\acs{dr1}/\acs{dr1}
    + \acs{ddisp_r}/\acs{dr1}}\right) = 
    -\arctan\left(\dfrac{\acs{disp_z'}}{1+\acs{disp_r'}}\right),\label{eq:analytical-tilt}
\end{equation}
where $'$ denotes the derivative with respect to $r_1$. The \textcite{mogi_relations_1958} solution provides the following displacement components:
\begin{gather}
    \acs{disp_z} = kd{(d^2+r_1^2)}^{-1.5},\label{eq:uz_mogi}\\
    \acs{disp_r} = kr_1{(d^2+r_1^2)}^{-1.5},\label{eq:ur_mogi}\\
    k = {3R^3\Delta P}/{4G},\label{eq:k}
\end{gather}
where $d$ is the depth to the center of the reservoir, $R$ is the reservoir radius, $\Delta P$ is the overpressure, and $G$ is the elastic shear modulus of the surrounding rock. Notice that Equation~\eqref{eq:k} can be written to solve for the product of reservoir volume and overpressure, which represents the energy associated with the reservoir pressure change:
\begin{equation}
    \acs{epv} = \frac{4}{3}\pi R^3\Delta P=\frac{16\pi G}{9} \cdot k.\label{eq:epv}
\end{equation}

Differentiating Equations~\eqref{eq:uz_mogi} and~\eqref{eq:ur_mogi}, substituting into Equation~\eqref{eq:analytical-tilt}, and simplifying:
\begin{equation}
    \acs{tilt} = \arctan\left(\frac{3kdr_1}{{(d^2+r_1^2)}^{2.5}+k(d^2-2r_1^2)}\right).\label{eq:mogi-tilt}
\end{equation}
This key equation relates the independently calculated variables \acs{tilt} and $r_1$ to physical parameters associated with reservoir pressure change: depth $d$ and energy \acs{epv}.

% \subsubsection{Physical Considerations}

The physical conditions assumed in this analytical model (deep, point-like reservoir in an elastic half-space) are not necessarily met even within numerical models, much less the physical edifice of Olympus Mons. However, this solution serves two important roles in the subsequent analysis. 

First, when the assumptions \emph{are} upheld\footnote{to the greatest extent possible; a reservoir of finite width and depth can never completely eliminate edge effects from the free surface above} in the numerical model, the analytical solution confirms that the model is working correctly. I show a representative example confirming this in Figure~\ref{fig:mogi-test}.

\begin{figure}
    \includegraphics[width=\textwidth]{methods/mogi-test.pdf}%
    \caption[Analytical solution verification]{Verification that the analytical tilt solution derived from \textcite{mogi_relations_1958} matches the numerical result for a deep, small, spherical reservoir in a flat half-space. ``Mogi (calc)'' uses parameters $d$ and \acs{epv} identical (or calculated directly from) those in the numerical model; ``Mogi (fit)'' uses a non-linear least squares regression to fit the numerical model data to the parameterized tilt function. All three results are essentially identical; the estimated parameters are very close to the true parameters.}%
    \label{fig:mogi-test}
\end{figure}

More importantly, I show in Figure~\ref{fig:mogi-test-shallow-oblate} that even conditions which violate the analytical solution assumptions produce tilt functions of similar qualitative shapes, although the associated parameters are incorrect. A non-linear least squares regression can be applied to find the combination of depth and inflation energy that best explains any tilt-distance dataset. 

\begin{figure}
    \includegraphics[width=\textwidth]{methods/mogi-test-shallow-oblate.pdf}%
    \caption[Analytical model sensitivity to reservoir geometry]{Illustration of the error introduced by violating the analytical assumption of a deep, point-like reservoir in a flat half-space. Specifically, this numerically modeled reservoir is oblate and close to the surface relative to its size; it also lies within a model of the Olympus Mons edifice rather than a flat half-space. Despite these factors, an analytical solution can fit the shape of this numerical data well, albeit by overestimating the depth and inflation energy responsible.}%
    \label{fig:mogi-test-shallow-oblate}
\end{figure}
