\section{Synthesis: Axisymmetric Tilt}\label{sec:synthesis}

The previous sections introduce some broad considerations and describe basic mechanics of data collection methods as they exist in isolation. However, the methods for \emph{analyzing} those data in light of those considerations are difficult to partition into ``mapping'' and ``modeling.'' Therefore, the sequence of methods in this section frequently transitions between map- and model-related data, functions, etc.

To keep this process organized, I bring all data (map-derived attitude and position, model-derived displacement and position) into a Jupyter Notebook environment using the Python programming language (version 3.10.9). Python is an open-source, general purpose language whose large library of modules make it a popular choice for scientific computing applications.

In the body of this section, I describe conceptually and graphically the sequence of computations I would perform for a single analysis. The full source code I use to implement these operations for my entire dataset is included in Appendix~\ref{app:code}.

\subsection{Deriving Tilt from Mapped Attitude Data}\label{sec:tilt-from-map}

\subsubsection{Spatial Description of Sampled Points}

Mars' rotation about an essentially fixed axis provides a convenient coordinate system of \ac{lat} and \ac{lon} for describing surface position. However, an \emph{axis}ymmetric model imposes a different coordinate system---one defined with respect to the axis of rotational symmetry.

Points mapped in GIS are automatically expressed in \ac{lat} and \ac{lon}; I need to convert this into polar coordinates $(\acs{dist},\acs{bearing})$ with the axisymmetric center as the origin. And because one goal of the project is to evaluate an array of candidate center points, I need an equation that takes any set of center coordinates $(\acs{latC}, \acs{lonC})$ and any set of sample point coordinates and returns a $(\acs{dist},\acs{bearing})$ pair.

To calculate \acf{dist}, I use the spherical law of cosines\footnote{Also known as the great circle distance formula, derived in Appendix~\ref{app:gcd}.} scaled by the martian equatorial radius since Olympus Mons is close to the equator:
\begin{equation}
    \acs{dist}=\arccos(\cos\acs{latC}\cos\acs{lat}\cos(\acs{lonC}-\acs{lon}) + \sin\acs{latC}\sin\acs{lat})\cdot\qty{3396.2}{\km}.
\end{equation}

It turns out that error in the calculated \acs{bearing} value is much more significant than error in \acs{dist}. That's because I use \acs{bearing} to define the axis of tilt and thus the value of tilt for a given center-sample pair, and this computation is very sensitive to minor variations.

Therefore, since all the calculations take place at the sampled point, I define \acs{bearing} as the geographic azimuth angle at that point directly \emph{away} from \acs{center}.\footnote{It is worth noting that this numerical bearing is different from the azimuth from the center to the sample, although they represent the same physical direction.} This is to ensure that the horizontal \acs{bearing}-axis points in the same direction as positive \acs{dist}. This equation \parencite{williams_aviation, veness_calculate} has the form:
\begin{equation}
    \acs{bearing} = \ang{180} + \arctan\left(\frac{\sin(\acs{lonC}-\acs{lon})\cos\acs{latC}} {\cos\acs{lat} \sin\acs{latC}-\sin\acs{lat}\cos\acs{latC}\cos(\acs{lonC}-\acs{lon})}\right).\label{eq:bearing}
\end{equation} 

\subsubsection{Paleo-Slope Equation}\label{sec:paleo-slope}

\begin{figure}
\begin{center}
    \newcommand{\stereoradius}{7cm}
\newcommand{\dotradius}{3pt}

\begin{tikzpicture}[scale=1]
    \coordinate (orig) at (0,0);
    \coordinate (edge) at (0,0);
    \coordinate (pole) at (70:5.15cm);
    \draw (O) circle (\stereoradius);
    \fill (O) circle (\dotradius);

    % az1 line
    \draw[arrow] (orig) -- node[fill=white,sloped] {\acs{az1}} (45:\stereoradius);

    % \draw small circle between point and line;
    \draw[dashed] (pole) --+ (0,-3);

    % theta line
    \draw[arrow] (orig) -- node[fill=white] {\acs{bearing}} (0,\stereoradius);
    
    % half ellipse
    \draw[green!80!black, ultra thick] (\stereoradius,0) arc(0:180:7cm and 5cm) node[near end,sloped,fill=white] {\acs{proj2}};

    % half ellipse
    \draw[blue,ultra thick] (\stereoradius,0) arc(0:180:7cm and 1.83cm) node[near end,sloped,fill=white] {\acs{proj1}};

    % az2, sl2 point
    \fill (pole) circle (\dotradius) node[anchor=south] {$(\acs{az2}, \acs{sl2})$};

    % fill in 
    \fill[red,opacity=0.15] (\stereoradius,0) arc(0:180:7cm and 1.83cm) arc(180:0:7cm and 5cm);

    % points of rotation
    \fill (\stereoradius,0) circle (\dotradius);
    \fill (-\stereoradius,0) circle (\dotradius);
\end{tikzpicture}%
    \caption[\Acl{tilt} from mapping]{\textbf{Left:} A quadrant of an upper hemisphere representing attitude space for a sampled point relative to a particular inflation center. The pole labeled $\left(\acs{az2},\acs{sl2}\right)$ represents the modern surface. The line of poles labeled \acs{az1} represents the family of possible paleo-surfaces with downhill direction given by the lava flow direction. The $r-$axis points in the azimuthal direction away from a modeled inflation center \acs{center}. The assumption of axisymmetry imposes a horizontal axis of tilt perpendicular to the $r-$axis. The red arrow labeled \acs{tilt} answers the question: how much tilt \emph{about this axis} bring some pole on line \acs{az1} to pole $\left(\acs{az2},\acs{sl2}\right)$? \textbf{Right:} Orthographic projection labeled with important angles and distances for the tilt calculation, building on Figure~\ref{fig:surface}. This region corresponds to horizontal grey rectangle on the left.}%
    \label{fig:tilt-from-map}%
\end{center}
\end{figure}
While the modern topographic surface attitude is fully characterized by \acs{az2} and \acs{sl2}, only the \acf{az1} can be inferred directly from mapped surface features. The axisymmetric model imposes a circumferential tilt axis, that is, modeled tilt is always directly toward or away from the inflation center. I denote the direction away from the inflation center by \acs{bearing}. Figure~\ref{fig:tilt-from-map} illustrates my method for calculating \acf{sl1} under this assumption. The horizontal segment labeled $\delta$ can be defined in two ways:
\begin{equation*}
    \delta = \sin\acs{beta1}\sin\acs{sl1} = \sin\acs{beta2}\sin\acs{sl2},
\end{equation*}
where $\acs{beta1} = \acl{beta1}$ and $\acs{beta2} = \acl{beta2}$. Thus,
\begin{equation}
    \acs{sl1} = \arcsin\left(\frac{\sin\acs{beta2}\sin\acs{sl2}}{\sin\acs{beta1}}\right).\label{eq:sl1}
\end{equation}

Equation~\eqref{eq:sl1} can naturally handle some issues inherent to the geometry of the problem. First, for some combinations of surface characteristics and center point choice, it is impossible to tilt the once-downhill lava flow about the imposed axis and reach the observed surface attitude. In Figure~\ref{fig:tilt-from-map} a single quadrant contains both $\left(\acs{az2},\acs{sl2}\right)$ and \acs{az1}, but if these fall on opposing sides of the \acs{bearing} line (one on the left, one on the right) there will be no possible vertical translation between the two. Mathematically, this case will arise as a negative value for \acs{sl1}, which is unphysical and must therefore be removed from the subsequent analysis. When a large fraction of sampled points are subject to this error, it signals that the chosen axisymmetric inflation center is inaccurate.

Another problem occurs even in the single-handed (both left or both right) case when:
\begin{equation}
     |\sin\acs{beta2}\sin\acs{sl2}| > |\sin\acs{beta1}|,\label{eq:domain-error}
\end{equation}
because the resulting argument in Equation~\eqref{eq:sl1} is outside the domain of the arcsin function. Physically this case occurs when the lava flow is close to parallel to the radial direction for the modeled inflation center. Additionally, when $|\sin\acs{beta1}|$ is only slightly greater than $|\sin\acs{beta2}\sin\acs{sl2}|$, the calculated \acs{sl1} will be unrealistically large. Both of these cases reflect the difficulty of changing the downhill azimuth of a flow by tipping it roughly in the direction it is already pointing. The quantity of these undefined or unrealistic tilt values provide an additional test on the validity of each axisymmetric inflation center.

\subsubsection{Tilt Equation}

In Figure~\ref{fig:tilt-from-map}, the arrow which translates the \acs{az1} line onto $\left(\acs{az2},\acs{sl2}\right)$ is a segment of a small circle about the tilt axis. To determine the true angle of tilt I translate this small circle onto the corresponding great circle along the blue and green great circle segments, respectively. I include this derivation in Appendix~\ref{app:proj}. The difference between these two transformations is the tilt: 
\begin{equation}
    \acs{tilt} = \arctan(\tan\acs{sl2}\cos\acs{beta2}) - \arctan(\tan\acs{sl1}\cos\acs{beta1}).\label{eq:tilt-from-map}
\end{equation}
Unlike Equation~\eqref{eq:sl1}, this calculations will not raise any domain errors. The only regions of high numerical sensitivity are near the poles of the tilt axis, but this occurs only for near-vertical observed surfaces (to yield high values of $\delta$ in Figure~\ref{fig:tilt-from-map}) which is unphysical.

\subsubsection{Physical Considerations}

Equation~\eqref{fig:tilt-from-map} could in principle be applied directly to a given center-sample pair to calculate up to one unique tilt value. However, this calculation is extremely sensitive to error in collected data. Even if attitude data could be collected with perfect precision, there are several reasons why the resulting tilt would be inaccurate. For example, a lava flow may have been heading at a slight offset from the regional downhill azimuth due to local topography or physical processes occurring within the lava. Therefore, I introduce an uncertainly term for the \acs{az1} term, $\pm\ang{7}$. 

This uncertainty term provides some flexibility in the analysis, but at a cost. By definition, any flow feature whose modern underlying topography has a downhill azimuth within \ang{7} cannot be considered ``discordant'' and thus no non-zero tilt can be inferred here.

However, samples taken from discordant features can now be explained by a range of tilts, as shown in Figure~\ref{fig:az1-uncertainty}.

\begin{figure}
    \begin{tikzpicture}[scale=1.4]

    \coordinate (orig) at (0,0);
    \coordinate (s2) at (-60:1);

    \draw[green!70!black,ultra thick] (s2) + (83:.3) arc (83:120:.3);

    \draw (orig) circle (\flatradius);

    \draw[arrow, ultra thick] (orig) -- (70:\flatradius) node[anchor=south] {\acs{az1}};
    \fill (s2) circle (1mm) node[anchor=north] {$(\acs{az2},\acs{sl2})$};

    \draw[] (s2) -- (orig);
    \draw[] (s2) -- (70:\flatradius);


\end{tikzpicture}%
\hspace{5mm}%
\begin{tikzpicture}[scale=1.4]

    \coordinate (orig) at (0,0);
    \coordinate (s2) at (-60:1);

    \draw[dashed] (s2) -- (70:\flatradius);

    \draw[green!70!black,ultra thick] (s2) + (64:.3) arc (64:120:.3);

    \fill[opacity=0.1] (orig) -- (70+\uncert:\flatradius) arc (70+\uncert:70-\uncert:\flatradius) -- (orig);
    
    \draw (orig) circle (\flatradius);
    
    \draw[arrow, ultra thick] (orig) -- (70:\flatradius) node[anchor=south] {\acs{az1} range};
    \draw[arrow, blue] (orig) -- (70-\uncert:\flatradius);
    \draw[arrow, blue] (orig) -- (70+\uncert:\flatradius);
    \fill (s2) circle (1mm) node[anchor=north] {$(\acs{az2},\acs{sl2})$};
    
    \draw[blue] (s2) -- (orig);
    \draw[blue] (s2) -- (70-\uncert:\flatradius);

    \fill (s2) circle (1mm);
\end{tikzpicture}%%
    \caption[Paleo-azimuth uncertainty]{Paleo-azimuth uncertainty introduces tilt flexibility. \textbf{Left:} With \acs{az1} treated as an exact quantity, there are a narrow range of radial directions (colored green) which can translate the line to meet the point, and for many of those angles the tilt required is unrealistically large. \textbf{Right:} With a wider range of possible \acs{az1} values, more radial directions are possible to translate some point in the shaded region to the $(\acs{az2},\acs{sl2})$ point. Additionally, the tilt required to reach the boundary will always be less. However, points within the shaded region must have $\acs{tilt}=\ang{0}$. Therefore, the choice of uncertainty value is a tradeoff between incorporating more data points and explaining discordance using smaller, more realistic tilts.}%
    \label{fig:az1-uncertainty}
\end{figure}

\subsection{Deriving Tilt from Modeled Displacement Data}\label{sec:tilt-from-model}

It is much more straightforward to derive tilt from the numerical model, since this is a direct simulation of a surface being deformed from an initial configuration to a displaced configuration. The only issue is that COMSOL produces displacement data corresponding to individual model nodes. Tilt is derived from the attitude of a surface, the smallest unit of which in the axisymmetric model is an edge defined by two adjacent nodes.

\subsubsection{Tilt Equation}
Figure~\ref{fig:tilt-from-model} shows the surface edge of a single model element before (initial) and after (displaced) modeled pressure change. The corresponding tilt equation is:
\begin{equation}
    \acs{tilt} = \arctan\left({\acs{dz1}}/{\acs{dr1}}\right) - \arctan\left(\dfrac{\acs{dz1}+\acs{ddisp_z}}{\acs{dr1}+\acs{ddisp_r}}\right).\label{eq:tilt-from-model}
\end{equation}

I define the distance value associated with this calculated tilt as the midpoint of the displaced edge. In practice, the horizontal scale for this edge (both before and after displacement) is negligible compared to the scale of the edifice, so another choice of distance within the edge (e.g., the midpoint of the initial surface) would produce indistinguishable results.

\begin{figure}
    \newcommand{\uar}{3.5cm}
\newcommand{\uaz}{11cm}
\newcommand{\ubr}{5.5cm}
\newcommand{\ubz}{4.5cm}
\newcommand{\dz}{2.5cm}
\newcommand{\dr}{8cm}

\begin{tikzpicture}
    \coordinate (O) at (0,0);
    \coordinate (A_1) at (0,\dz);
    \coordinate (B_1) at (\dr,0);
    
    \path (A_1) + (\uar, \uaz) coordinate (A_2);
    \path (B_1) + (\ubr, \ubz) coordinate (B_2);
    \path (A_1) + (\ubr, \ubz) coordinate (B_2_trans);

    \path (A_1) + (0, \uaz - \ubz) coordinate (A_1_trans);
    \path (B_1) + (\ubr - \uar, 0) coordinate (B_1_trans);
    \path (B_1) + (\uar, \ubz) coordinate (B_1_trans_helper);

    \path (A_1) + (\uar, \ubz) coordinate (O_disp);

    \draw[-latex] (O) --+ (-0.5,0) node[anchor=east] {\acs{center}};

    \draw[] (O) -- node[fill=white] {\acs{dr1}} (B_1);
    \draw[] (O) -- node[fill=white] {$-\acs{dz1}$} (A_1);

    \draw[ultra thick] (A_1) -- node[sloped,fill=white] {Original Surface} (B_1);

    \draw[arrow] (A_1) -- node[sloped, fill=white] {$\acs{disp_a}$} (A_2);
    \draw[arrow] (B_1) -- node[sloped, fill=white] {$\acs{disp_b}$} (B_2);
    \draw[arrow] (A_1) -- node[sloped, fill=white] {$\acs{disp_b}$} (B_2_trans);
    \draw[-latex] (A_2) -- node[sloped, fill=white] {$\acs{ddisp}$} (B_2_trans);
    
    \fill[blue, opacity = 0.15] (B_1) --++ (162.7:2.5) arc (162.7:180:2.5);
    \path (B_1) + (172:2) node {\acs{proj1}};
    \draw[thick, dashed] (B_2) -- (B_2_trans);
    
    \draw[] (A_1) -- node[fill=white] {$-\acs{ddisp_z}$} (A_1_trans);
    \draw[] (B_1) -- node[fill=white] {$\acs{ddisp_r}$} (B_1_trans);
    
    \draw[dashed, thin] (O_disp) -- (A_1);
    \draw[dashed, thin] (A_2) -- (A_1_trans);
    \draw[dashed, thin] (B_2) -- (B_1_trans);
    \draw (B_1_trans_helper) --  node[fill=white] {\acs{ddisp_r}} (B_2);
    
    \fill[red, opacity = 0.3] (B_2) --++ (138:2.75) arc (138:162.7:2.75);
    \path (B_2) + (150:2) node {\acs{radial-deform}};
    \draw[thick, dashed] (B_2) -- (B_2_trans);
    
    \fill[blue, opacity = 0.15] (B_1_trans) --++ (138:2.75) arc (138:180:2.75);
    \path (B_1_trans) + (160:2.2) node {\acs{proj2}};
    \draw[ultra thick] (B_2) -- (B_2_trans);
    
    
    \draw[ultra thick] (A_1_trans) -- (B_1_trans);
    
    \draw[] (A_2) -- node[fill=white] {$-\acs{ddisp_z}$} (O_disp) -- node[fill=white] {$\acs{ddisp_r}$} (B_2_trans);
    \draw[ultra thick] (A_2) -- node[sloped,fill=white] {Deformed Surface} (B_2);

    \draw[dashed] (O_disp) -- (B_1_trans_helper) -- (B_1);
    
    \fill[black] (A_1) circle (2pt) node[anchor = east] {$A_1$};
    \fill[black] (B_1) circle (2pt) node[anchor = north] {$B_1$};
    \fill[black] (A_2) circle (2pt) node[anchor = south] {$A_2$};
    \fill[black] (B_2) circle (2pt) node[anchor = west] {$B_2$};
    \fill[black] (A_1_trans) circle (2pt);
    \fill[black] (B_1_trans) circle (2pt);
    
\end{tikzpicture}%
    \caption[Tilt from numerical modeling]{Cross-sectional view of a surface edge in the axisymmetric numerical model. As in Figure~\ref{fig:tilt-from-map}, the $r-$axis points away from the inflation center. During the model, the two nodes $A_1$ and $B_1$ of a surface element are displaced along \acs{disp_a} and \acs{disp_b} to reach a final position at $A_2$ and $B_2$, respectively. I illustrate the difference $\acs{disp_b} - \acs{disp_a} = \acs{ddisp}$ in terms of its components \acs{ddisp_r} and \acs{ddisp_z}. Notice that $z-$component terms are negative to ensure that tilt away from the inflation center yields a positive tilt \acs{tilt}, as shown in red. To calculate \acs{tilt}, I determine the slopes of segments $\overline{A_1B_1}$ and $\overline{A_2'B_1}$ using the labeled horizontal and vertical segments, convert these slopes to angles, and take their difference.}%
    \label{fig:tilt-from-model}%
\end{figure}

\subsubsection{Additional Model Considerations}

In Section~\ref{sec:modeling}, I explain that the axisymmetric numerical model edifice is constructed from a particular elevation transect measured from the summit of Olympus Mons. Relative to this point, surrounding topography is roughly axisymmetric, especially more than \qty{\sim50}{\km} away past the previously discussed \qty{19}{\km} contour.

However, I do not assume that the subsequent reservoir pressure change was centered at this same location. Instead, my goal is to  determine this inflation center location on the basis of discordant flow data, wherever they may point.

Importantly, placing a reservoir inflation center anywhere off the axis of symmetry in the numerical model introduces some inaccuracy. To make progress I need to determine whether this error is negligible; if it is not, I need to be aware of its magnitude as I interpret results.

Of course, an axisymmetric model by definition does not permit any non-axisymmetric elements to be introduced. Instead, I construct a flat (no edifice, horizontal surface) variant of the model as an end-member case for the topographic variation introduced by shifting the reservoir within the edifice.\footnote{This model does not directly address the issue of introducing uphill slope away from the inflation center, but the magnitude of any error introduced should be similar.}

Additionally, \textcite{grosfils_magma_2007} showed that incorporating gravitational loading is unnecessary for modeling surface displacement. To confirm this, I run a test model under three conditions:
\begin{enumerate}
    \item no gravitational loading with magma reservoir overpressure \label{g0p1}
    \item gravitational loading (lithostatic pre-stress) with no reservoir overpressure\label{g1p0}
    \item gravitational loading with reservoir overpressure \label{g1p1}
\end{enumerate}
If gravitational loading is in fact insignificant, displacement in case~\ref{g0p1} should match the component of remaining in case~\ref{g1p1} after the gravitational component from case~\ref{g1p0} is subtracted out. The reason I need to subtract out this component is that the modeled edifice is not perfectly flat. Therefore, vertical loading is not initially in equilibrium under the reservoir and some ``slumping'' occurs to accommodate this imbalance. I present preliminary test results in Figure~\ref{fig:grav-topo-test}.

\begin{figure}
    \includegraphics[width=\textwidth]{grav-topo-test.pdf}
    \includegraphics[width=\textwidth]{grav-topo-test-zoom.pdf}%
    \caption[Numerical model sensitivity to topography and gravity]{Comparison between topographic and flat model with and without gravitational loading. Patterns are consistent across a range of model parameters tested; a single representative case is plotted above, with the peak shown in more detail below. Gravitational loadings makes essentially no difference, while the flat model (topo = False) tends to underestimate tilt by a few percent.}%
    \label{fig:grav-topo-test}%
\end{figure}

As expected for the flat model, gravitational loading has no effect on surface displacement and thus the subsequently calculated tilt is identical. The topographically accurate model shows the same pattern with respect to gravitational loading. There is a small but noticeable difference in tilt between the flat and topographic models, which ranges between ($0\%-10\%$) for the parameters I examined. This result places a reassuring upper bound on the magnitude of error that could be introduced by varying the horizontal location of a modeled inflation center within the summit region---likely much less than error introduced elsewhere.

\subsection{Analytical Tilt Solution}

In this section, I draw on the widely cited analytical \emph{displacement} solution developed by \textcite{mogi_relations_1958} to derive an analytical \emph{tilt} solution as a counterpart to the numerical method described by Equation~\eqref{eq:tilt-from-model}. This so-called Mogi model assumes a deep spherical reservoir within a flat (``topo = False'') elastic half-space.

\subsubsection{Tilt Equation}

Equation~\eqref{eq:tilt-from-model} in such a half-space ($\acs{dz1} = 0$) reduces to:
\begin{equation}
    \acs{tilt} = 
    -\arctan\left(\dfrac{\acs{ddisp_z}}{\acs{dr1}+\acs{ddisp_r}}\right).\label{eq:tilt-from-flat-model}
\end{equation}
This discrete equation can be taken to the continuous limit by dividing each term in the numerator and denominator by the edge width \acs{dr1}:
\begin{equation}
\acs{tilt}
    = \lim_{\acs{dr1}\to0} 
    -\arctan\left(\dfrac{\acs{ddisp_z}/\acs{dr1}}{\acs{dr1}/\acs{dr1}
    + \acs{ddisp_r}/\acs{dr1}}\right) = 
    -\arctan\left(\dfrac{\acs{disp_z'}}{1+\acs{disp_r'}}\right),\label{eq:analytical-tilt}
\end{equation}
where $'$ denotes the derivative with respect to $r_1$. The \textcite{mogi_relations_1958} solution provides the following displacement components:
\begin{gather}
    \acs{disp_z} = kd{(d^2+r_1^2)}^{-1.5},\label{eq:uz_mogi}\\
    \acs{disp_r} = kr_1{(d^2+r_1^2)}^{-1.5},\label{eq:ur_mogi}\\
    k = {3R^3\Delta P}/{4G},\label{eq:k}
\end{gather}
where $d$ is the depth to the center of the reservoir, $R$ is the reservoir radius, $\Delta P$ is the overpressure, and $G$ is the elastic shear modulus of the surrounding rock. Notice that Equation~\eqref{eq:k} can be written to solve for the product of reservoir volume and overpressure, which represents the energy associated with the reservoir pressure change:
\begin{equation}
    \acs{epv} = \frac{4}{3}\pi R^3\Delta P=\frac{16\pi G}{9} \cdot k.\label{eq:epv}
\end{equation}

Differentiating Equations~\eqref{eq:uz_mogi} and~\eqref{eq:ur_mogi}, substituting into Equation~\eqref{eq:analytical-tilt}, and simplifying:
\begin{equation}
    \acs{tilt} = \arctan\left(\frac{3kdr_1}{{(d^2+r_1^2)}^{2.5}+k(d^2-2r_1^2)}\right).\label{eq:mogi-tilt}
\end{equation}
This key equation relates the measured\footnote{or independently calculated.} variables \acs{tilt} and $r_1$ to physical parameters associated with reservoir pressure change: depth $d$ and energy \acs{epv}.

\subsubsection{Physical Considerations}

The conditions assumed in this analytical model are not necessarily met even within my numerical models, much less the physical edifice of Olympus Mons. However, this solution serves two important roles in my analysis. 

First, when the assumptions \emph{are} upheld\footnote{to the greatest extent possible; a reservoir of finite width and depth can never completely eliminate edge effects from the free surface above} in the numerical model, the analytical solution confirms that the model is working correctly. I show a representative example confirming this in Figure~\ref{fig:mogi-test}.

\begin{figure}
    \includegraphics[width=\textwidth]{mogi-test.pdf}%
    \caption[Analytical solution verification]{Verification that the analytical tilt solution derived from \textcite{mogi_relations_1958} matches the numerical result for a deep, small, spherical reservoir in a flat half-space. ``Mogi (calc)'' uses parameters $d$ and \acs{epv} identical (or calculated directly from) those in the numerical model; ``Mogi (fit)'' uses a non-linear least squares regression to fit the numerical model data to the parameterized tilt function. All three results are essentially identical; the estimated parameters are very close to the true parameters.}%
    \label{fig:mogi-test}
\end{figure}

More importantly, I show in Figure~\ref{fig:mogi-test-shallow-oblate} that even conditions which violate the analytical solution assumptions produce tilt functions of similar qualitative shapes, although the associated parameters are incorrect. This is arguably the most significant methodological finding in this thesis because it provides the quantitative link between map- and model-derived datasets, which I describe in 

\begin{figure}
    \includegraphics[width=\textwidth]{mogi-test-shallow-oblate.pdf}%
    \caption[Analytical model sensitivity to reservoir geometry]{Illustration of the error introduced by violating analytical assumption of a deep, point-like reservoir in a flat half-space. Specifically, this numerically modeled reservoir is oblate and close to the surface relative to its size; it also lies within a model of the Olympus Mons edifice rather than a flat half-space. Despite these factors, an analytical solution can fit the shape of this numerical data well, albeit by overestimating the depth and inflation energy responsible.}%
    \label{fig:mogi-test-shallow-oblate}
\end{figure}

\subsection{Evaluating Center Candidates}

In this section, I define two broad criteria by which to evaluate the axisymmetric center candidates shown in Figure~\ref{fig:candidates} based on their spatial relationship with mapped sample data. Broadly speaking, these criteria are designed to answer the question ``is this candidate location reasonable as an axisymmetric center point?'' Importantly, two different locations may each individually explain different aspects of the study area. Additionally, the answer to ``where is the most likely center location'' may have a different answer even based on the same criterion if different sample populations are considered. After all, my goal is not to construct a single axisymmetric model but to determine the best explanations for all observed data by more flexibly applying the axisymmetric framework.

I illustrate the flow of this iterative approach abstractly in Figure~\ref{fig:eval-model} and present the Python implementation in Appendix~\ref{app:code}. In the following sections, I describe the specific criteria used to evaluate different aspects of the summit's geometry and history.

\begin{figure}
    \begin{tikzpicture}[scale=.95]

  \usetikzlibrary{positioning}

\coordinate (ceval) at (0,0);
\coordinate (centersample) at (0,5);
\coordinate (samples) at (-3,10);
\coordinate (centers) at (3,10);

\node [rotate=270] at (7, 10) {Input (from GIS)};
\node [rotate=270] at (7, 5) {Intermediate list};
\node [rotate=270] at (7, 0) {Output (back to GIS)};

\draw[arrow,line width=1mm] (6, 10) -- (6, 0);

\node (samplestable) [draw, shape=rectangle, align=center] at (samples) {\begin{tabular}{c|c}
    sID & LAT, LON, \acs{az1}, \acs{az2}, \acs{sl2}\\
    \hline\\
    \hline\\
    \hline\\
    \hline\\
  \end{tabular}};

\node[above=0mm of samplestable] {\hltt{samples.csv}};

\node (centerstable) [draw, shape=rectangle, align=center] at (centers) {\begin{tabular}{c|c}
    cID & LAT, LON\\
    \hline\\
    \hline\\
    \hline\\
    \hline\\
    \hline\\
    \hline\\
  \end{tabular}};

\node[above=0mm of centerstable] {\hltt{centers.csv}};

\node (frontcentersample) [shape=rectangle, align=center, rounded corners=0.2cm] at (centersample) {\begin{tabular}{c|c|c}
  sID & LAT, LON, \acs{az1}, \acs{az2}, \acs{sl2} & \acs{dist}, \acs{bearing}, \acs{beta1}, \acs{beta2}, \acs{sl1}, \acs{tilt}\\
  \hline
  &\\
  &copy of & calculated\\
  &\hltt{samples.csv} & from one cID\\
  &\\
\end{tabular}};

\node[above=3mm of frontcentersample] {\hltt{centers\_calc}};

\foreach \x in {0.3,0.25,...,-0.05}
    \path (centersample) + (-\x, \x) node[draw, shape=rectangle, align=center, fill=white, rounded corners=0.2cm] {\begin{tabular}{c|c|c}
        sID & LAT, LON, \acs{az1}, \acs{az2}, \acs{sl2} & \acs{dist}, \acs{bearing}, \acs{beta1}, \acs{beta2}, \acs{sl1}, \acs{tilt}\\
        \hline
        &\\
        &copy of & calculated\\
        &\hltt{samples.csv} & from cID\\
        &\\
      \end{tabular}};

  \node (centersevaltable) [draw, shape=rectangle, align=center] at (ceval) {\begin{tabular}{c|c|c}
    cID & LAT, LON & SCORES\\
    \hline
    &\\
    &\\
    &copy of&each row calculated from\\
    &\hltt{centers.csv}&one item in \hltt{centers\_calc} \\
    &\\
    &\\
  \end{tabular}};

\node[above=0mm of centersevaltable] {\hltt{centers\_eval.csv}};

\end{tikzpicture}%
    \caption[Center evaluation workflow]{Schematic illustration of candidate center evaluation process. For each center (unique cID) in \hltt{centers.csv}, I make a copy of \hltt{samples.csv} and calculate variables for each sample (unique sID). This results in the ``intermediate list'' where each element corresponds to one cID. Then I evaluate each item in this list and write the resulting scores to one row of \hltt{centers\_eval.csv}.}%
    \label{fig:eval-model}
\end{figure}

\subsubsection{Paleo-Summit}

The center location which best explains an axisymmetric paleo-edifice summit is the one for which flow features are pointing downhill away from the center as directly possible. In other words, this point is the one which minimizes the average value of $\beta_1$ for all the sampled points. For this criterion, I use both the full sample dataset and populations broken down by feature type (polygon flow and linear channel). Estimated paleo-summits differing by feature type could suggest that the features erupted under different topographic conditions and thus at different times, with regional deformation occurring between eruptive periods.

\subsubsection{Inflation Center}

Identifying and evaluating axisymmetric inflation (or deflation) centers is considerably more difficult for several reasons.

First, very little information can be derived from a single center-sample pair of points. That is because the single resulting tilt-distance data point could be consistent with any number of reservoir pressure conditions---or more likely, simply a mathematical calculation with no physical basis since the actual tilt at that point occurred about a different axis of rotation. Therefore, simply calculating an average value for some variable over all sample points (as in the previous case) will not be sufficient.

Instead, I need to consider a larger population of tilt-distance data points together. This increases the likelihood of recognizing a tilt pattern consistent with reservoir pressure change, i.e., a shape similar to those shown in Figures~\ref{fig:grav-topo-test},~\ref{fig:mogi-test}, and~\ref{fig:mogi-test-shallow-oblate}. However, this alone does not address the second issue, that erroneously calculated tilt values will often be present alongside any true tilt patterns consistent with axisymmetric tilt from the inflation center in question. Of course, identifying these erroneous calculations as such would require independent knowledge of the axis and degree of tilt responsible, which is exactly what I am trying to determine.

The next best option is to discard any calculations which \emph{must} be erroneous, recognizing that some will have to remain to have any chance of identifying real tilt signatures. To do this, I first calculate each tilt value using Equation~\eqref{eq:tilt-from-map} and the \acs{az1}-uncertainty correction illustrated in Figure~\ref{fig:az1-uncertainty}. Specifically, this correction involves setting the tilt for any points where $\acs{disc}<\ang{7}$ to $\acs{tilt}=\ang{0}$, and for the remaining points calculating \acs{tilt} for the two boundaries of the uncertainty region, then taking the lower of the two (if one exists at all).

Then, I compute an analytical tilt solution to a deep, high-\acs{epv} inflation (or deflation), as shown Figure~\ref{fig:envelope}. This is designed to place an upper bound on realistic tilt at each distance from the center, so any computed tilt values which exceed this envelope are discarded due to physical implausibility. This function needs to have an especially high value of \acs{epv} in light of Figure~\ref{fig:mogi-test-shallow-oblate}; shallow oblate reservoirs are best approximated by slightly deeper and \emph{much} more energetic inflations, and I want to be confident that the cutoff envelope exceeds any realistic conditions.

\begin{figure}
    \includegraphics[width=\textwidth]{envelope.pdf}%
    \caption[Physically plausible tilt envelope]{Upper and lower plausibility bounds on tilt as a function of distance. This function is computer using a relatively deep and extremely high-\acs{epv} Mogi function to ensure appropriate breadth and width is included, keeping in mind that the true reservoir is likely to be oblate. Any computed tilt values outside this envelope are discarded on physical grounds.}%
    \label{fig:envelope}
\end{figure}

During this process, I count the number of points discarded due either to mathematical or physical implausibility---this number itself is an important criterion for evaluating the inflation center candidate.

Finally, I perform a non-linear least squares regression to fit an analytical tilt solution (Equation~\eqref{eq:mogi-tilt}). Just as a linear regression computes the best-fitting slope and intercept, this process computes the best fit parameters $d$ and $k$, the latter of which is easily converted to \acs{epv} via Equation~\eqref{eq:epv}. I use these fitted values and crucially, their associated error terms, to determine the likelihood that surface tilt actually occurred in the direction radial to the given center candidate. This approach is supported by Figure~\ref{fig:mogi-test-shallow-oblate}, which illustrates that even tilt resulting from shallow, oblate reservoirs within a topographic edifice can be fit well to the analytical Mogi tilt function. At this stage, I pay closer attention to the parameter error estimates (similar to an $R^2$ value for a linear regression) to determine whether \emph{any} axisymmetric tilt is likely to have occurred from this center, rather than the estimates themselves---although I treat any unrealistic parameter estimates with caution regardless of their goodness of fit.

This sequence of calculations is much more computationally expensive than the previous one, and a single realistic reservoir pressure change is unlikely to explain the entire dataset anyway. Therefore, I select smaller sample populations for this criterion, paying particular attention to the highly discordant flows near the southern summit region shown in Figure~\ref{fig:uphill-flows}.

\subsection{Evaluating Reservoir Pressure Change Parameters}

Note: for reasons presented in Chapter~\ref{cha:discussion}, this is a time-permitting element for the final submission of this thesis.

Finally, I draw on the preliminary result shown in Figure~\ref{fig:mogi-test-shallow-oblate}: map data which are well fit by the analytical Mogi tilt solution are also likely to fit one or more numerical results. For the most promising center candidates, I examine these numerical results to identify which set of reservoir conditions most closely match the fitted analytical function, and more importantly the map-derived tilt dataset itself.