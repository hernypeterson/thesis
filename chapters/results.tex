\chapter{Results}\label{cha:results}

\section{Mapping Coverage}

Figure~\ref{fig:features} depicts all polygonal flows and linear channels I identify; Figure~\ref{fig:samples} shows the samples derived from these features symbolized by discordance.

\begin{figure}
    \includegraphics[width=\textwidth]{results/features.pdf}%
    \caption[Mapped lava features]{Mapped polygonal lava flows and linear channels.}%
    \label{fig:features}
\end{figure}

\begin{figure}
    \includegraphics[width=\textwidth]{results/samples.pdf}%
    \caption[Mapped topographic discordance]{Samples symbolized by topographic discordance.}%
    \label{fig:samples}
\end{figure}

\section{Inflation Center Evaluation}

Figure~\ref{fig:populations} shows three sample populations selected within a highly discordant region for inflation center evaluation.

\begin{figure}
    \includegraphics[width=\textwidth]{results/populations.pdf}%
    \caption[Sample populations for inflation center evaluation]{I use three nearby populations of lava features from which to evaluate axisymmetric inflation candidates. I chose these because they are among the most discordant features in the summit which are not at the immediate caldera rim. For reference, the flows shown in the center of Figure~\ref{fig:uphill-flows} are included in population B.}%
    \label{fig:populations}
\end{figure}

\begin{figure}
    \includegraphics[width=\textwidth]{results/tiltable-offset-abc.pdf}%
    \caption[Centers by tiltable/offset fraction]{These are the centers which yield a quantitative tilt estimate (tiltable) which is well constrained (offset) for the highest proportion of the discordant features in the southeast summit.}%
    \label{fig:tiltable-offset-abc}
\end{figure}

\begin{figure}
    \includegraphics[width=\textwidth]{results/rmse-abc.pdf}%
    \caption[Centers by analytical convergence]{These are the centers for which the tiltable/offset subset (Figure~\ref{fig:tiltable-offset-abc}) converge to an analytical tilt solution. Blank regions either do not converge at all or converge to a physically meaningless solution; see Figure [Next] for examples.}%
    \label{fig:rmse-abc}
\end{figure}