\chapter{Results}\label{cha:results}

\section{Mapping Coverage}

Figure~\ref{fig:features} depicts all mapped polygonal flows and linear channels; Figure~\ref{fig:samples} shows the samples derived from these features symbolized by discordance (\acs{disc}).

\begin{figure}
    \includegraphics[width=\textwidth]{results/features.pdf}%
    \caption[Mapped lava features]{Mapped polygonal lava flows and linear channels.}%
    \label{fig:features}
\end{figure}

\begin{figure}
    \includegraphics[width=\textwidth]{results/samples.pdf}%
    \caption[Mapped topographic discordance]{Samples symbolized by topographic discordance.}%
    \label{fig:samples}
\end{figure}

\section{Reservoir Center Location}

Figure~\ref{fig:populations} shows three sample populations selected within a highly discordant region for inflation center evaluation.

\begin{figure}
    \includegraphics[width=\textwidth]{results/populations.pdf}%
    \caption[Sample populations for inflation center evaluation]{I use three nearby populations of lava features from which to evaluate axisymmetric inflation candidates. I chose these because they are among the most discordant features in the summit which are not at the immediate caldera rim. For reference, the flows shown in the center of Figure~\ref{fig:uphill-flows} are included in population B.}%
    \label{fig:populations}
\end{figure}

\begin{figure}
    \includegraphics[width=\textwidth]{results/tiltable-offset-abc.pdf}%
    \caption[Centers by tiltable/offset fraction]{Centers which yield a well-constrained (offset) quantitative tilt estimate (tiltable) for the highest proportion of the discordant features in the southeast summit. Note the red streak roughly colinear with the lava features: centers in this region are not offset, i.e., tilt about the imposed axis tends to change slope but not downhill azimuth. Note also the asymmetry across this central red streak; centers north of the samples (near the eastern caldera rim) are more plausible than those to the west (southern summit, near Pangboche crater). Additional regions of high tiltable/offset fraction occur directly within the populations and in the far northwest corner of the caldera complex.}%
    \label{fig:tiltable-offset-abc}
\end{figure}

% flatten pdf annotations: https://tools.pdf24.org/en/flatten-pdf#s=1681704380798

\begin{figure}
    \includegraphics[width=\textwidth]{results/rmse-abc.pdf}%
    \caption[Centers by analytical convergence]{Centers for which an analytical tilt solution converges on the tiltable/offset subset (Figure~\ref{fig:tiltable-offset-abc}) of the southeast populations. Blank regions either do not converge at all or converge to a physically meaningless solution. Note that both inflation from the south (black circles) or deflation from the north (plain circles) are consistent with the same regional tilt pattern back toward the eastern caldera complex. Key regions numbered for further inspection of associated tilt-distance datasets. Regions 1-5 are promising with relatively low \acs{RMSE} and high tiltable/offset fraction; 6 \& 7 illustrate important limitations of the method.}%
    \label{fig:rmse-abc}
\end{figure}

\begin{enumerate}
    \item Southeast Caldera Rim
    % 598: 2.77
    % 650: 2.79,
    % 624: 2.81,
    % 698: 2.82,
    % 370: 2.92
    % 572: 3.72,
    % 695: 4.63,
    % 540: 5.74
    % 313,
    % 342,
    % 369,
    % 341,
    % 398,
    % 426
    % 454 next
    \item East of Caldera Rim
        432: 2.00
        403: 2.03
        549: 2.04
        520: 2.06
        490: 2.09
    \item Southern Summit
        17: 1.63 
        4: 1.65
        5: 1.73
        24: 3.13

    \item Within Discordant Features
    \item Southeast Caldera Rim (Low Tiltable/Offset Fraction; Poor Fit)
    \item Northwest Caldera Rim (High Tiltable/Offset Fraction; Poor Fit)
\end{enumerate}

\begin{figure}
    \vspace{-15pt}
    \includegraphics[width=\textwidth]{results/428-analytical.pdf}\\
    \includegraphics[width=\textwidth]{results/486-analytical.pdf}\\
    \includegraphics[width=\textwidth]{results/544-analytical.pdf}%
    \caption[Southeast Caldera Rim: analytical fit]{Representative datasets for centers in region 1 (southeast caldera rim). Vertical tick marks represent distances to all samples. Scatter points represent tilt computed for the tiltable/offset samples. The analytical fit for each dataset is labeled with its \acs{RMSE}, \ang{\sim2.7} for the best fits. Despite high error, note the tilt signature is consistently negative, i.e., represents deflation.}
    \label{fig:region1-analytical}
\end{figure}

\begin{figure}
    \vspace{-15pt}
    \includegraphics[width=\textwidth]{results/403-analytical.pdf}\\
    \includegraphics[width=\textwidth]{results/432-analytical.pdf}\\
    \includegraphics[width=\textwidth]{results/549-analytical.pdf}%
    \caption[East of Caldera Rim: analytical fit]{Representative datasets for centers in region 2 (east of caldera rim).}
    \label{fig:region2-analytical}
\end{figure}

\begin{figure}
    \includegraphics[width=\textwidth]{results/17-analytical.pdf}\\
    \includegraphics[width=\textwidth]{results/24-analytical.pdf}\\
    \includegraphics[width=\textwidth]{results/151-analytical.pdf}%
    \caption[East of Caldera Rim: analytical fit]{Representative datasets for centers in region 3 (southern summit).}
    \label{fig:region34-analytical}
\end{figure}

\section{Depth, Radius, Aspect Ratio, \& Pressure Change}

Can a plausible numerical tilt solution match or exceed the \acs{RMSE} for the dataset at the identified center(s)?