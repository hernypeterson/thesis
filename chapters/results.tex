\chapter{Results}

\section{Mapping Results}

Note: for map figures I translated the contacts feature class from SIM 3470 roughly \qty{700}{\m} to the North to align with the \ac{CTX} basemap.

Differentiated lava lows and channels do not uniformly cover the summit of Olympus Mons. Immediately beyond the caldera rim, distinguishable flows are limited mostly to the south-east (e.g., Figure~\ref{fig:uphill-flows}) and north-west margins. Notably, there are essentially zero flows immediately south of the caldera, closer to the modern summit. [FIGURE of all mapped lobate lava flows]. Lava channels are more widespread than lobate flows, especially further from the caldera rim 

\subsection{Mapping Coverage}
\begin{figure}
    \includegraphics[width=\textwidth]{results-coverage.pdf}%
    \caption[Mapping Coverage]{Polygon Flows and Linear Channels}%
    \label{fig:results-coverage}
\end{figure}
% map with polygons for entire study area
% smaller map with polygons, centerlines, and sampled points - points labels with arrows downhill and scaled by color

\subsection{Paleo-Summit Evaluation}
\begin{figure}
    \includegraphics[width=\textwidth]{summit-score.pdf}%
    \caption[Paleo-Summit Evaluation]{Paleo-summit Evaluation. Green regions with low ``Summit Score'' are most likely to represent the summit of an axisymmetric paleo-edifice. Summit Score is defined for each candidate point (large colored circle) as the average value of $\log_{10}\beta_1$ for all sampled points relative to that candidate point. In other words, linear features (both channels and lobate flow centerlines) collectively flow away from the green regions more than from the red regions. White contour lines are included to distinguish the true center of the green region, that is, the most likely paleo-summit.}%
    \label{fig:summit-score}
\end{figure}

\begin{figure}
    \includegraphics[width=\textwidth]{beta1.pdf}%
    \caption[Best Paleo-Summit Candidate]{Using the paleo-summit shown at the center of the green region in Figure~\ref{fig:summit-score}.}%
    \label{fig:beta1}
\end{figure}

\subsection{Inflation Center Evaluation}
% maps matching previous case in symbology but showing radial tilt instead (so all arrows pointing toward or away from the respective center point)

% scatter plots with radial distance against axisymmetric tilt for a. entire study area and b. sub-area

\section{Model Results}
% figures with radial tilt plotted against radial distance
    % scatter plot for each center point: channel points and flow points
    % line plot with parameter combinations that most closely match scatter plot