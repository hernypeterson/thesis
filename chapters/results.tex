\chapter{Results}

\section{Mapping Results}

Note: for map figures I translated the contacts feature class from SIM 3470 roughly \qty{700}{\m} to the North to align with the \ac{CTX} basemap.

Differentiated lava lows and channels do not uniformly cover the summit of Olympus Mons. Immediately beyond the caldera rim, distinguishable flows are limited mostly to the south-east (e.g., Figure~\ref{fig:uphill-flows}) and north-west margins. Notably, there are essentially zero flows immediately south of the caldera, closer to the modern summit. [FIGURE of all mapped lobate lava flows]. Lava channels are more widespread than lobate flows, especially further from the caldera rim 

\subsection{Mapping Coverage}
\begin{figure}
    \includegraphics[width=\textwidth]{results-coverage.pdf}%
    \caption[Mapping Coverage]{Polygon Flows and Linear Channels}%
    \label{fig:results-coverage}
\end{figure}
% map with polygons for entire study area
% smaller map with polygons, centerlines, and sampled points - points labels with arrows downhill and scaled by color

\subsection{Paleo-Summit Evaluation}
\begin{figure}
    \includegraphics[width=\textwidth]{summit-score.pdf}%
    \caption[Paleo-Summit Evaluation]{Summit Score is defined for each candidate point as the average value of \acs{beta1} for all sampled points relative to that candidate point. Since the scores vary smoothly over the candidates, I show only the central contour plots to identify the lowest (best) scoring region.}%
    \label{fig:summit-score}
\end{figure}

\begin{figure}
    \includegraphics[width=\textwidth]{beta1.pdf}%
    \caption[Best Paleo-Summit Candidate]{Using the paleo-summit shown at the center of the green region in Figure~\ref{fig:summit-score}.}%
    \label{fig:beta1}
\end{figure}

\subsection{Inflation Center Evaluation}

Figure~\ref{fig:cutoff} illustrates the degree to which estimated depth and energy parameters simply match those used to define the cutoff function.

\begin{figure}
    \includegraphics[width=\textwidth]{hugging-a.pdf}
    \includegraphics[width=\textwidth]{hugging-c.pdf}
    \caption[Check for envelope ``hugging'']{Evaluating inadvertent influence from the tilt envelope (Figure~\ref{fig:envelope}) on the regression. Essentially, I want to see whether calculated parameters closely ``hug'' the values set to define the shallow high-\acs{epv} Mogi tilt cutoff. This is a kernel density estimate where this ``hugging'' issue would show up as steep clusters right at 0, particularly for \acs{epv}, since depth $d$ is much more constrained. Seeing variation in \acs{epv} over several orders of magnitude is encouraging. This is also weighted by the number of samples that are actually ``tiltable.'' The best fits are hopefully on the lower end of the \acs{epv} distribution\dots} 
    \label{fig:cutoff}
\end{figure}

 \begin{figure}
    \begin{center}
     \includegraphics[width=.5\textwidth]{num-tiltable-a.pdf}\\
     \includegraphics[width=.5\textwidth]{num-tiltable-b.pdf}%
     \includegraphics[width=.5\textwidth]{num-tiltable-c.pdf}
     \caption[Inflation Center Candidates by number of ``tiltable'' samples]{Center candidates symbolized by the number of samples from the given population for which the computed tilt (using Equation~\eqref{eq:tilt-from-map} with the \acs{az1} corrections illustrated in Figure~\ref{fig:az1-uncertainty}) is within the maximum realistic tilt envelope (Figure~\ref{fig:envelope}). \textbf{Top:} Population A; $n=13$. \textbf{Left:} Population B; $n=14$. \textbf{Right:} Population C; $n=12$. Note the change in color ramp interval for Population B.} 
     \label{fig:num-tiltable}
    \end{center}
\end{figure}

\begin{figure}
    \includegraphics[width=\textwidth]{math-fit-a.pdf}%
    \caption[Population A: Goodness of Fit]{For population A, nonlinear regression using parameterized analytical function (Equation~\eqref{eq:tilt-from-flat-analytical-model}) converges, i.e., successfully reaches an estimate within 800 function calls, for 10 of the 781 center candidates, all within \qty{\sim20}{\km} of the samples. The four purple circles represent those estimates where the relative error (standard deviation divided by estimate) of both the energy \acs{epv} and depth $d$ parameters are within a factor of ten. Scaled parameter error for the light grey circles are much higher, indicating a poor fit. Center candidates are labeled by cID, for comparison with regression results in Figure~\ref{fig:scatter-fit-a}, and the fraction ``tiltable'' (proportional to displayed values in Figure~\ref{fig:num-tiltable})}%
    \label{fig:math-fit-a}
\end{figure}

\begin{figure}
    \includegraphics[width=\textwidth]{phys-fit-a.pdf}%
    \caption[Population A: Parameter Estimates]{Parameter estimates for the centers whose regression converges to explain sample population A. Colors represent the base-$10$ logarithm of the inflation energy \acs{epv} expressed in joules. Centers with black circles represent inflation (positive \acs{epv}); plain circles represent deflation (negative \acs{epv}). Estimated depths are labeled, with shallower estimates corresponding to larger circles. Nonsensical results such as negative depth reflect the capacity for a regression to converge mathematically to physically meaningless parameter estimates.}%
    \label{fig:phys-fit-a}
\end{figure}

\begin{figure}
    \includegraphics[width=\textwidth]{scatter-fit-a.pdf}%
    \caption[Scatter Plot for Select Center Candidates: Population A]{Scatter fit}%
    \label{fig:scatter-fit-a}
\end{figure}

\begin{figure}
    \includegraphics[width=\textwidth]{math-fit-c.pdf}
    \caption[Population C: Goodness of Fit]{For population C, the regression converges for a slim majority (422) of the 781 center candidates. Almost all depth estimates reach an absolute error under 10, while energy estimates are mixed. Circle sizes are proportional to the number of ``tiltable'' samples included in the regression, the higher values of which are tightly concentrated around the population as for population A (Figure~\ref{fig:math-fit-a}).}%
    \label{fig:math-fit-c}
\end{figure}

\begin{figure}
    \includegraphics[width=\textwidth]{phys-fit-c.pdf}%
    \caption[Population C: Parameter Estimates]{Parameter estimates for the centers whose regression converges to explain sample population C.}%
    \label{fig:phys-fit-c}
\end{figure}


\section{Model Results}
% figures with radial tilt plotted against radial distance
    % scatter plot for each center point: channel points and flow points
    % line plot with parameter combinations that most closely match scatter plot