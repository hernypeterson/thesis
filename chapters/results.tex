\chapter{Results}

\section{Mapping Results}

Note: for map figures I translated the contacts feature class from SIM 3470 roughly \qty{700}{\m} to the North to align with the \ac{CTX} basemap.

Differentiated lava lows and channels do not uniformly cover the summit of Olympus Mons. Immediately beyond the caldera rim, distinguishable flows are limited mostly to the south-east (e.g., Figure~\ref{fig:uphill-flows}) and north-west margins. Notably, there are essentially zero flows immediately south of the caldera, closer to the modern summit. [FIGURE of all mapped lobate lava flows]. Lava channels are more widespread than lobate flows, especially further from the caldera rim 

\subsection{Mapping Coverage}
\begin{figure}
    \includegraphics[width=\textwidth]{results-coverage.pdf}%
    \caption[Mapping Coverage]{Polygon Flows and Linear Channels}%
    \label{fig:results-coverage}
\end{figure}
% map with polygons for entire study area
% smaller map with polygons, centerlines, and sampled points - points labels with arrows downhill and scaled by color

\subsection{Paleo-Summit Evaluation}
\begin{figure}
    \includegraphics[width=\textwidth]{summit-score.pdf}%
    \caption[Paleo-Summit Evaluation]{Summit Score is defined for each candidate point as the average value of \acs{beta1} for all sampled points relative to that candidate point. Since the scores vary smoothly over the candidates, I show only the central contour plots to identify the lowest (best) scoring region.}%
    \label{fig:summit-score}
\end{figure}

\begin{figure}
    \includegraphics[width=\textwidth]{beta1.pdf}%
    \caption[Best Paleo-Summit Candidate]{Using the paleo-summit shown at the center of the green region in Figure~\ref{fig:summit-score}.}%
    \label{fig:beta1}
\end{figure}

\subsection{Inflation Center Evaluation}

\begin{figure}
    \includegraphics[width=\textwidth]{cutoff1.pdf}%
    \caption[Cutoff Effect]{Evaluating undue influence from the maximum tilt cutoff. Essentially, I want to see whether calculated parameters closely ``hug'' the values set to define the shallow high-\acs{epv} Mogi tilt cutoff. This is a kernel density estimate where this ``hugging'' issue would show up as steep clusters right at 0, particularly for \acs{epv}, since depth $d$ is much more constrained. Seeing variation in \acs{epv} over several orders of magnitude is encouraging. This is also weighted by the number of samples that are actually ``tiltable.'' The best fits are hopefully on the lower end of the \acs{epv} distribution\dots} 
\end{figure}

\section{Model Results}
% figures with radial tilt plotted against radial distance
    % scatter plot for each center point: channel points and flow points
    % line plot with parameter combinations that most closely match scatter plot